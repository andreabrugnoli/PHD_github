\documentclass[letterpaper, 10 pt, conference]{ieeeconf}

\IEEEoverridecommandlockouts
\overrideIEEEmargins 

\usepackage{graphicx}      % include this line if your document contains figures
\usepackage{amsmath,amssymb}
\usepackage{url}
\usepackage{diffcoeff}
\usepackage{bm}
\usepackage{cite}
\usepackage{subfig}
\usepackage{diffcoeff}
\usepackage{mathrsfs}
\usepackage{tikz}
\usetikzlibrary{shapes,arrows,positioning,calc}

\newtheorem{theorem}{Theorem}
\newtheorem{remark}{Remark}
\newtheorem{proposition}{Proposition}


\DeclareMathOperator*{\argmax}{arg\,max}
\DeclareMathOperator*{\argmin}{arg\,min}
\DeclareMathOperator{\Tr}{Tr}

\def\onedot{$\mathsurround0pt\ldotp$}
\def\cddot{% two dots stacked vertically
	\mathbin{\vcenter{\baselineskip.67ex
			\hbox{\onedot}\hbox{\onedot}}%
}}

\makeatletter \renewcommand\d[1]{\ensuremath{%
		\;\mathrm{d}#1\@ifnextchar\d{\!}{}}}
\makeatother

\title{\LARGE \bf
	Control by interconnection of the Kirchhoff plate
	within the Port-Hamiltonian framework*	
}


\author{Andrea Brugnoli$^{1}$ and Daniel Alazard$^{1}$ and Val\'erie Pommier-Budinger$^{1}$ and Denis Matignon$^{1}$% <-this % stops a space
	\thanks{*This work is supported by the project ANR-16-CE92-0028,
		entitled {\em Interconnected Infinite-Dimensional systems for Heterogeneous
			Media}, INFIDHEM, financed by the French National Research Agency (ANR) and the Deutsche Forschungsgemeinschaft (DFG). Further information is available at { {https://websites.isae-supaero.fr/infidhem/the-project} }.}% <-this % stops a space
	\thanks{$^{1}$Andrea Brugnoli, Daniel Alazard, Val\'erie Pommier-Budinger and Denis Matignon are with ISAE-SUPAERO, Universit\'e de Toulouse, France,
		10 Avenue Edouard Belin, BP-54032, 31055 Toulouse Cedex 4
		{\tt\small \{Andrea.Brugnoli, Daniel.Alazard, Valerie.Budinger, Denis.Matignon\}@isae.fr} }
}


\begin{document}


\maketitle
\thispagestyle{empty}
\pagestyle{empty}

% make the title area
\maketitle

% As a general rule, do not put math, special symbols or citations
% in the abstract
\begin{abstract}
	The Kirchhoff plate model is here detailed by using a tensorial port-Hamiltonian (pH) formulation. A structure-preserving discretization of this model is then achieved by using the partitioned finite element (PFEM). This methodology easily accounts for the boundary variables and the finite dimensional system can be interconnected to the surrounding environments in an simple and structured manner. The algebraic constraints to be considered are deduced by the boundary conditions, that may be homogeneous or defined by an interconnection with another dynamical system. \\
	The versatility of the proposed approach is presented by means of numerical simulations. A first illustration considers a rectangular plate clamped on one side and interconnected to a rod rigidly attached to the opposite side. A second example exploits the collocated output feature of pH systems to perform damping injection in a plate undergoing an external forcing. A stability proof is obtained effortlessly as the Hamiltonian is a Lyapunov function. 
\end{abstract}

% no keywords




% For peer review papers, you can put extra information on the cover
% page as needed:
% \ifCLASSOPTIONpeerreview
% \begin{center} \bfseries EDICS Category: 3-BBND \end{center}
% \fi
%
% For peerreview papers, this IEEEtran command inserts a page break and
% creates the second title. It will be ignored for other modes.
\IEEEpeerreviewmaketitle



\section{Introduction}
The port-Hamiltonian (pH) framework has proved to be a powerful framework for modeling and control multi-physics system \cite{bookPHs}. During the last years distributed systems, i.e. systems ruled by partial differential equations (PDEs) have attracted a lot of interest \cite{BookZwart}. The modularity of the pH paradigm is particularly appealing as it provides a structured and coherent way to build complex system. Infinite \cite{ShaftIntInfinite} and finite \cite{CerveraIntFinite} dimensional pH systems can be connected together giving rise to another pH system. In order to simulate and control such systems, a finite dimensional representation of the distributed system has to be found and it is advantageous to use a discretization procedure that preserves the port-Hamiltonian nature. \\

The first attempt to perform a structure-preserving discretization dates back to \cite{Golo}, where the authors proposed a mixed finite element spatial discretization for 1D hyperbolic system. Pseudo-spectral methods were studied in \cite{moulla:hal-01625008} .  A 2D finite difference method with staggered grids was used in \cite{Trenchant}. In \cite{WeakForm_Kot} the prototypical example of hyperbolic systems of two conservation laws was discretized by a weak formulation, leading to a Galerkin numerical approximations. All these methods require a specific implementation and cannot be related to standard numerical methods. An extension of the Mixed finite element method to pH system was proposed in \cite{CardosoRibeiro2018}. The main point of this methodology is that the integration by parts to be performed so that the symplectic structure is preserved. Several choice of the boundary control are possible. For this reason this method is referred to as partitioned finite element method (PFEM). If mixed boundary conditions have to be considered the discretized system is an algebraic differential one (pHDAEs), which can be analyzed by referring to \cite{beattie2018linear,vanderSchaft2013}. \\

In this paper the Kirchhoff plate model is presented in a pH fashion. The tensorial calculus is employed in order to clearly identify the skew-symmetric nature of the differential operator. The PFEM methodology is then used to obtain a finite dimensional system. Depending on the application the weak form may contain forces and momenta or linear and angular velocities as control inputs. Numerical applications are then carried out using the Firedrake platform \cite{firedrake}. First an interconnection along the boundary is then presented to model an cantilever plate welded to a rigid bar. A control application by damping injection follows. The plate is interconnected along part of the boundary with a dissipative system. The Hamiltonian for the interconnected system is a Lyapunov function, therefore the system will tend to the equilibrium point, i.e. the undeformed configuration. \\

In section \ref{sec:KirchhPH} the Kirchhoff Plate model in strong form as a port-Hamiltonian system is described. In section \ref{sec:WF} BLABLA 

\section{pH formulation of the Kirchoff plate}
\label{sec:KirchhPH}

In this section the classical formulation of the Kirchhoff plate is recalled. Then the tensorial pH formulation is illustrated. The boundary variables are highlighted thanks to the energy balance.

\subsection{Notations}
First, the differential operators needed for the following are recalled. For a scalar field $u: \mathbb{R}^d \rightarrow \mathbb{R}$ the gradient is defined as 
\begin{equation*}
\mathrm{grad}(u) =  \nabla u := \begin{pmatrix}
\partial_{x_1} u \dots \partial_{x_d} u \\
\end{pmatrix}^T.
\end{equation*}
For a vector field $\bm{v}: \mathbb{R}^d \rightarrow \mathbb{R}^d$ the symmetric part of the gradient operator $\mathrm{Grad}$ (i. e. the deformation gradient in continuum mechanics) is given by
\begin{equation*}
\mathrm{Grad}(\bm{v}) := \frac{1}{2} \left(\nabla \bm{v} + \nabla^T \bm{v} \right).
\end{equation*}
The Hessian operator of $u$ is then computed as follows
\begin{equation*}
\mathrm{Hess}(u) = \mathrm{Grad}(\mathrm{grad}(u)),
\end{equation*}
For a tensor field $\bm{U}: \mathbb{R}^d \rightarrow \mathbb{R}^{d \times d}$, with elements $u_{ij}$, the divergence is a vector defined column-wise as
\begin{equation*}
\mathrm{Div}(\bm U) = \nabla \cdot \bm{U} := \left( \sum_{i = 1}^d \partial_{x_i} u_{ij} \right)_{j = 1, \dots, d}.
\end{equation*}
The double divergence of a tensor field $\bm{U}$ is then a scalar field defined as
\begin{equation*}
\mathrm{div}(\mathrm{Div}(\bm U)):= \sum_{i, j = 1}^d \partial_{x_i} \partial_{x_j} u_{ij} .
\end{equation*}
The geometrical dimension of interest in this paper is $d=2$. In the following vectors (tensor) fields and numerical vectors (matrices) will be denote by a lower (upper) case bold letter. It will be clear by the context which mathematical objects is considered. Furthermore, $\mathbb{S}$ denotes the space of symmetric $d \times d$ tensors (matrices).
\subsection{Kirchhoff Model for Thin Plates}
\label{subsec:classMin}
The Kirchhoff Model is a generalization to the 2D case of the Euler-Bernoulli beam model and accounts for the shear deformation. Given an open and connected set $\Omega \in \mathbb{R}^2$, the classical equations for this model (\cite{timoshenko1959theory}) are 
\begin{equation}
\displaystyle \rho h \diffp[2]{w}{t} = -\mathrm{div}(\mathrm{Div}(\bm{M})), \\
\end{equation}
where $\rho$ is the material density, $h$ is the plate thickness, the scalar $w$ is the vertical displacement and $\bm{M}$ is the symmetric momenta tensor.  This tensor is related to the symmetric curvatures tensor $\bm{K}$ by the bending rigidity tensor, so that $\bm{M}_{ij} = \bm{D}_{ijkl} \bm{K}_{kl}$. The curvature tensor is defined as
\begin{equation*}
\bm{K} := \mathrm{Grad}(\mathrm{grad}(w)).
\end{equation*}
For an homogeneous isotropic material the components of $\bm{M}, \bm{K} \in \mathbb{S}$ are relatedby the relations ($x$ denotes index 1, $y$ index~2)
\begin{equation*}
\begin{aligned}
m_{xx} &= D\left(\kappa_{xx} + \nu \kappa_{yy}\right),\\
m_{yy} &= D\left(\kappa_{yy} + \nu \kappa_{xx}\right),\\
m_{xy} &= D(1 - \nu) \kappa_{xy}, \\
\end{aligned}
\end{equation*}
with $\nu$ the Poisson ratio, $D$ the bending module. The kinetic and potential energy density $\mathcal{K}$ and $\mathcal{U}$ read
\begin{equation}
\mathcal{K} =  \frac{1}{2}\rho h \left(\diffp{w}{t} \right)^2, \quad
\mathcal{U} = \frac{1}{2} \bm{M} \cddot \bm{K},
\end{equation} 
where $\bm{M} \cddot \bm{K} := \sum_{i,j} m_{ij} \kappa_{ij}$ is the tensor contraction. The Hamiltonian  is easily written as
\begin{equation} 
H = \int_{\Omega} \left( \mathcal{K} + \mathcal{U} \right)   \d\Omega. 
\end{equation}

\subsection{Tensorial Port-Hamiltonian formulation}
In order to rewrite the system as a port-Hamiltonian one, the energy variables have to be selected first. This  choice is analogous to that of the pH Euler-Bernoulli beam  model (\cite{BrugnoliKir}), but with the additional complication that the variable mome:
\begin{equation}
\begin{aligned}
\alpha_w &= \rho h \diffp{w}{t}, \quad &\text{Linear momentum}, \\
\bm{A}_{\kappa} &= \bm{K}, \quad &\text{Curvature tensor}.\\
\end{aligned}
\end{equation}
The co-energy variables are found by computing the variational derivative of the Hamiltonian
\begin{equation}
\begin{aligned}
e_w &:= \diffd{H}{\alpha_w} = \diffp{w}{t}, \quad &\text{Vertical Velocity}, \\
\bm{E}_{\kappa} &:= \diffd{H}{\bm{A}_{\kappa}} = \bm{M}, \quad &\text{Momenta tensor}.\\
\end{aligned} 
\end{equation}

The port-Hamiltonian system is expressed as follows 
\begin{equation}
\begin{cases}
\displaystyle\diffp{\alpha_w}{t} &= -\mathrm{div}(\mathrm{Div}(\bm{E}_{\kappa})), \vspace{1mm} \\
\displaystyle\diffp{\bm{A}_\kappa}{t} &= \mathrm{Grad}(\mathrm{Grad}(e_w)), ,
\end{cases}
\end{equation}


If the variables are concatenated together, the formally skew-symmetric operator $J$ can be highlighted 
\begin{equation}
\label{eq:PH_sys_Kir_Ten}
\diffp{}{t}
\begin{pmatrix}
\alpha_w \\
\bm{A}_\kappa \\
\end{pmatrix} = 
\underbrace{\begin{bmatrix}
	0  & -\mathrm{div} \circ \mathrm{Div} \\
	\mathrm{Grad} \circ \mathrm{grad}  & 0 \\
	\end{bmatrix}}_{\mathcal{J}}
\begin{pmatrix}
e_w \\
\bm{E}_{\kappa} \\
\end{pmatrix},
\end{equation}
\begin{remark}
	It can be observed that the interconnection structure given by $\mathcal{J}$ mimics that of the Euler-Bernoulli beam (in one spatial dimension the double divergence and the Hessian reduce to the second derivative).
\end{remark}


\begin{theorem}[\cite{BrugnoliKir}]
	The adjoint of the double divergence of a tensor $\mathrm{div} \circ \mathrm{Div}$ is $ - \mathrm{Grad} \circ  \mathrm{grad} = - \mathrm{Hess}$, the opposite of the Hessian operator.
\end{theorem}

The boundary values can be found by evaluating the time derivative of the Hamiltonian
\begin{equation}
\begin{aligned}
\dot{H}= \int_{\partial \Omega} \left\{ w_t \widetilde{q}_n  + \omega_n m_{nn}\right\} \d{s}.  \\
\end{aligned}
\end{equation}
where $s$ is the curvilinear abscissa and the result integral is obtained by applying the Green-Gauss theorem \cite{BrugnoliKir}.
The boundary variable are defined as follows
\begin{equation}
\label{eq:QnMnnMns}
\begin{aligned}
&\text{Effetive Shear Force}  \quad &\widetilde{q}_{n} &:=  -\mathrm{Div}(\bm{E}_{\kappa}) \cdot \bm{n} - \diffp{m_{ns}}{s},  \\
&\text{Flexural momentum} \quad  &m_{nn} &:=   \bm{E}_{\kappa} \cddot (\bm{n}\otimes{\bm{n}}),	
\end{aligned}
\end{equation}
where $m_{ns} :=  \bm{E}_{\kappa} \cddot (\bm{s} \otimes{\bm{n}})$ is the torsional momentum and $\bm{u} \otimes {\bm{v}}$ denotes the outer product of vectors equivalent to a matrix given by $\bm{u}\bm{v}^T$. Vectors $\bm{n}$ and $\bm{s}$ designate the normal and tangential unit vector to the boundary. The corresponding power conjugated variables are
\begin{equation}
\label{eq:wtwnws}
\begin{aligned}
\text{Vertical velocity}  \quad w_t &:= e_w, \\
\text{Flexural rotation} \quad \omega_{n} &:=  \diffp{e_w}{n}.
\end{aligned}
\end{equation}

\section{Structure preserving discretization}
\label{sec:SP_discr}
In this section the structure preserving discretization procedure is detailed. Three steps are needed: 
\begin{enumerate}
	\item put the system in weak form;
	\item second perform integrations by parts to get the boundary control of choice;
	\item select the finite element spaces to achieve a finite dimensional system.
\end{enumerate}


\subsection{Weak Form}
\label{subsec:WF}
In order to put the system into weak form the first line of \eqref{eq:PH_sys_Kir_Ten} is multiplied  by $v_w$ (multiplication by a scalar), the second line one by $\bm{V}_{\kappa}$ (tensor contraction).
\begin{align}
\int_{\Omega} v_w \diffp{\alpha_w}{t}  \d\Omega &=  -\int_{\Omega} v_w \mathrm{div}(\mathrm{Div}(\bm{E}_\kappa)) \, \d\Omega,  \label{eq:wf1_Kir_ten}\\
\int_{\Omega} \bm{V}_{\kappa} \cddot \diffp{\bm{A}_{\kappa}}{t}   \d\Omega &= +\int_{\Omega} \bm{V}_{\kappa} \cddot \mathrm{Grad}(\mathrm{grad}(e_w)) \, \d\Omega, \label{eq:wf2_Kir_ten} 
\end{align}
For sake of simplicity, all test and unknown functions can be collected in one variable
\begin{equation}
v := (v_{w}, \bm{V}_{\kappa}), \qquad \alpha := (\alpha_{w}, \bm{A}_{\kappa}), \qquad e := (e_{w}, \bm{E}_{\kappa}),
\end{equation}
so that the previous system is rewritten compactly as
\begin{equation}
\left(v, \diffp{\alpha}{t}\right) = (v, \mathcal{J}e),
\end{equation}
where the bilinear form $(v,u) = \int v \cdot u \d{\Omega}$, is the inner product on space $\mathscr{L}^2(\Omega) := L^2(\Omega) \times L^2(\Omega; \mathbb{S}) $. The operator $\mathcal{J}$ was defined in equation \eqref{eq:PH_sys_Kir_Ten}. It can be decomposed into the sum of three operators
\begin{equation}
\mathcal{J} = \mathcal{J}_{\mathrm{divDiv}} + \mathcal{J}_{\mathrm{Hess}}, 
\end{equation}
where $\mathcal{J}_{\mathrm{divDiv}}, \, \mathcal{J}_{\mathrm{Hess}}$ contain only the double divergence ($\mathrm{divDiv}$) and Hessian operator respectively. \\
The integration by part has to be performed so that the final bilinear form on the right-hand side remains skew-symmetric. Obviously, since $\mathcal{J}$ is skew-symmetric $\mathcal{J}_{\mathrm{divDiv}} = - \mathcal{J}_{\mathrm{Hess}}^*$, where $A^*$ is the formal adjoint of operator $A$. Depending on which of the two differential operators is chosen for the integration by parts, two different boundary controls can arise \cite{BrugnoliKir} (other choices are possible but less meaningful under a physical point of view). 


\subsection{Boundary control through forces and momenta}
\label{subsec:controlForces}
Applying the integration by parts twice on $\mathcal{J}_{\mathrm{divDiv}}$ is integrated by parts (meaning that 
the right-hand side of equation \eqref{eq:wf1_Kir_ten} is integrated by parts twice) then 
\begin{equation}
\label{eq:WF_Neum}
(v, Je) = j_{\mathrm{Hess}}(v, e) + f_{N}(v),
\end{equation}
where now the bilinear form 
\[j_{\mathrm{Hess}}(v, e) = (\mathcal{J}_{\mathrm{divDiv}}^* v, e) + (v, \mathcal{J}_{\mathrm{Hess}}e) \] 
is skew symmetric and can be expressed as follows
\begin{equation} \label{eq:j_allgrad}
\begin{aligned}
j_{\mathrm{Hess}}(v, e): =  &-  \displaystyle\int_{\Omega} \mathrm{Grad}(\mathrm{grad}(v_w))  \cddot \bm{E}_{\kappa} \, \d\Omega \d{s}\\
&+ \displaystyle\int_{\Omega} \bm{V}_{\kappa} \cddot \mathrm{Grad}(\mathrm{grad}(e_w)) \, \d\Omega. 
\end{aligned}
\end{equation}
The linear functional $f_{N}(v)$ represents the boundary term associated with forces and momenta. The subscript $N$ denotes the fact that classical Neumann conditions appear as boundary input. It reads
\begin{equation} \label{eq:Neum_funct}
f_{N}(v) =  \displaystyle\int_{\partial \Omega} \left\{ v_w \widetilde{q}_n +  v_{\omega_n} m_{nn} \right\}  \d{s},
\end{equation}
where $ v_{\omega_n}=\diffp{v_w}{n}$. In this first case,  the boundary controls $\bm{u}_\partial$ and the corresponding output $\bm{y}_\partial$ are 
\[\bm{u}_\partial = 
\begin{pmatrix}
{q}_n \\
m_{nn} \\
\end{pmatrix}_{\partial \Omega}, \qquad
\bm{y}_\partial = 
\begin{pmatrix}
w_t \\
\omega_n \\
\end{pmatrix}_{\partial \Omega}.
\]

\subsection{Finite Dimensional System}
\label{subsec:finPHs}
In this subsection the formulation \eqref{eq:WF_Neum} is used is order to explain the discretization procedure and the associated finite elements. 
\paragraph{Discretization Procedure}
Test and co-energy variables are discretized using the same basis function (Galerkin Method)
\begin{equation}
\begin{aligned}
v_w &= \sum_{i = 1}^{N_w} \phi_w^i(x,y) \, v_w^i, \\
\bm{V}_\kappa &= \sum_{i = 1}^{N_\kappa} \bm\Phi_\kappa^i(x,y) \, v_\kappa^i,\\
\end{aligned} \qquad 
\begin{aligned}
e_w &= \sum_{i = 1}^{N_w} \phi_w^i(x,y) \, e_w^i(t), \\
\bm{E}_\kappa &= \sum_{i = 1}^{N_\kappa} \bm\Phi_\kappa^i(x,y) \, e_\kappa^i(t),\\
\end{aligned}
\end{equation}
The basis function $\phi_w^i, \, \bm\Phi_\kappa^i, $ have to be chosen in a suitable function space $\mathcal{V}^h$ in the domain of operator $\mathcal{J}$, i.e. $\mathcal{V}^h \subset \mathcal{V} \in \mathcal{D}(\mathcal{J})$. The discretized skew-symmetric bilinear form given in \eqref{eq:j_allgrad} then reads
\begin{equation}
\bm{J}_d = 
\begin{bmatrix}
0 & -\bm{D}_{\mathrm{H}}^T \\
\bm{D}_{\mathrm{H}} & 0\\
\end{bmatrix},
\end{equation}
where $\bm{A}^T$ is the transpose of the $\bm{A}$ matrix. The matrix $\bm{D}_{\mathrm{H}}$ is computed in the following way
\begin{equation}
\bm{D}_{\mathrm{H}}(i,j) = \int_{\Omega} \bm{\Phi}_{\kappa}^i : \mathrm{Grad}(\mathrm{grad}(\phi_w^j)) \d\Omega, \quad \in \mathbb{R}^{N_\kappa \times N_w},
\end{equation}
where $A(i,j)$ indicates the entry in the matrix corresponding to the $i \, {\text{th}}$ row and $j \,{\text{th}}$ column. The energy variables are deduced from the co-energy variables 
\begin{equation}
\alpha_w = \rho h e_w, \qquad \bm{A}_{\kappa} = \bm{D}^{-1} \bm{E}_{\kappa}.
\end{equation}
The symmetric bilinear form on the left side of \eqref{eq:WF_Neum} is discretized as $\bm{M} = \text{diag}[\bm{M}_w,\, \bm{M}_\kappa]$ with
\begin{equation}
\begin{aligned}
&\bm{M}_w(i,j) = \int_{\Omega} \rho h \, \bm{\phi}_w^i \, \bm{\phi}_w^j \d\Omega \; \in \mathbb{R}^{N_w \times N_w}, \\
&\bm{M}_\kappa(i,j) = \int_{\Omega}  \left( \bm{D}^{-1} \bm{\Phi}_\kappa^i \right) \cddot \bm{\Phi}_\kappa^j \d\Omega \; \in \mathbb{R}^{N_\kappa \times N_\kappa},\\
\end{aligned}
\end{equation}
To deal with generic boundary conditions the Lagrange multipliers have to be introduced in \eqref{eq:Neum_funct}
\begin{equation}
\lambda_{\widetilde{q}_n} = \sum_{i = 1}^{N_{\widetilde{q}_n}} \bm{\phi}^i_{\widetilde{q}_n}(s) {\lambda}^i_{\widetilde{q}_n}, \quad
\lambda_{m_{nn}} = \sum_{i = 1}^{N_{m_{nn}}} \bm{\phi}^i_{m_{nn}}(s) {\lambda}^i_{m_{nn}},
\end{equation}
If inhomogeneous Neumann boundary conditions are then discretized as
\begin{equation}
\begin{aligned}
\widetilde{q}_n = \sum_{i = 1}^{N_{\widetilde{q}_n}} \phi_{\widetilde{q}_n}^i(s) \; \widetilde{q}_n^i, \qquad
m_{nn} = \sum_{i = 1}^{N_{m_{nn}}} \phi_{m_{nn}}^i(s) \; m_{nn}^i.
\end{aligned}
\end{equation}
It is now possible to construct the following matrices
\begin{equation}
\begin{aligned}
\bm{G}_{w}(i,j) &= \int_{\Gamma_{C} \cup \Gamma_{S} } {\phi}_{w}^i \, {\phi}_{\widetilde{q}_n}^j \d{s}, \quad  \in \mathbb{R}^{N_w \times N_{q_n}},\\
\bm{B}_{\widetilde{q}_n}(i,j) &= \int_{\Gamma_{\widetilde{q}_n}} {\phi}_{w}^i \, {\phi}_{\widetilde{q}_n}^j \d{s}, \quad  \in \mathbb{R}^{N_w \times N_{q_n}}, \\
\bm{G}_{\omega_n}(i,j) &= \int_{\Gamma_{C}} \diffp{\phi_w^i}{n} \, \phi_{m_{nn}}^j \d{s}, \quad \in \mathbb{R}^{N_w \times N_{m_{nn}}}, \\
\bm{B}_{m_{nn}}(i,j) &= \int_{\Gamma_{m_{nn}}} \diffp{\phi_w^i}{n} \, \phi_{m_{nn}}^j \d{s}, \quad \in \mathbb{R}^{N_w \times N_{m_{nn}}},\\
\end{aligned} 
\end{equation}
where $\Gamma_{C},\Gamma_{S}$ are subsets of the boundary where clamped and simply supported boundary conditions apply and $\Gamma_{\widetilde{q}_n},\Gamma_{m_{nn}}$ are subsets where inhomogeneous Neumann conditions hold. Consequently, the input matrix reads
The final port-Hamiltonian descriptor system (pHDAEs), as defined in \cite{beattie2018linear}, is written as
\begin{equation}
\label{eq:discr_pl}
\begin{bmatrix}
\bm{M} & \bm{0} \\
\bm{0} & \bm{0} \\
\end{bmatrix} \frac{\d}{\d t}
\begin{pmatrix}
\bm{e}\\
\bm{\lambda}_D \\
\end{pmatrix}
= \begin{bmatrix}
\bm{J}_d & \bm{G}_D\\
-\bm{G}_D^T & \bm{0} \\
\end{bmatrix}
\begin{pmatrix}
\bm{e}\\
\bm{\lambda}_D \\
\end{pmatrix} + \begin{bmatrix}
\bm{B}_N\\
\bm{0} \\
\end{bmatrix}
\bm{f}_N,
\end{equation}
where $\bm{e} = (\bm{e}_w; \, \bm{e}_\kappa), \;  \bm{\lambda}_D = (\bm{\lambda}_{\widetilde{q}_n}; \, \bm{\lambda}_{m_{nn}}), \; \bm{f}_N = (\widetilde{\bm{q}}_n; \, \bm{m}_{nn})$ are the concatenation of the co-energy variables, Lagrange multipliers and boundary forces and momenta at the borders and 
\[\bm{G}_D = \begin{bmatrix}
\bm{G}_{w} & \bm{G}_{\omega_n} \\
\bm{0} & \bm{0}  \\
\end{bmatrix}, \qquad 
\bm{B}_N = \begin{bmatrix}
\bm{B}_{\widetilde{q}_n} & \bm{B}_{m_{nn}} \\
\bm{0} & \bm{0}  \\
\end{bmatrix}.
\]. 
The subscript $N, D$ refer to Neumann and Dirichlet boundary conditions.

\paragraph{Finite Element Choice}
\label{par:FE}
The domain of the operator $\mathcal{J}$ in \eqref{eq:PH_sys_Kir_Ten} is $\mathcal{D}(J) = H^{2}(\Omega) \times  H^{\text{div Div}}(\Omega, \mathbb{R}^{2 \times 2}_{\text{sym}}) \,+ $ boundary conditions. For this reason a suitable choice for the functional space is
\begin{equation}
(v_w, \,\bm{V}_\kappa) \in H^{2}(\Omega) \times H^{2}(\Omega, \mathbb{R}^{2 \times 2}_{\text{sym}}) \equiv \mathscr{H},
\end{equation}
since $\mathscr{H} \subset \mathcal{D}(J)$. 
\begin{remark}
	It has to be appointed that the space $H^{\text{div Div}}(\Omega, \mathbb{R}^{2 \times 2}_{\text{sym}})$ was never addressed in the mathematical literature. For this reason the only way to deal with this problem numerically is to use $H^{2}(\Omega)$ conforming finite elements. 
\end{remark}
The Firedrake library \cite{firedrake} was used to implement the numerical analysis as it provides functionalities to automate the generalized mappings for $H^2$ conforming finite elements (like the Hermite, Bell or Argyris finite elements). All the variables, i.e. the velocity $e_w$ and the momenta tensor $\bm{E}_\kappa$ as well as the corresponding test functions, are discretized by the same finite element space, the Bell finite element \cite{Bell}, denoted by $H_r^2(\mathbb{P}_5, \Omega)$. The multipliers are therefore discretized by using second degree Lagrange polynomials defined over the boundary and denoted with $H_r^1(\mathbb{P}_2, \partial\Omega)$. 


\section{Interconnection with a finite dimensional conservative pH system}
In this section the interconnection of an infinite and finite port system is explained in both the infinite and finite dimensional setting.

\subsection{Infinite dimensional setting}
Consider an infinite dimensional pH system ( or distributed pH system, dpH) and a finite dimensional pH system denoted by equations 

\noindent
\begin{minipage}{.5\linewidth}
	\begin{equation}
	\text{pH} \left\{ 
	\begin{aligned}
	\dot{\bm{x}}_2 = \bm{J} \diffp{H_2}{\bm{x}_2} + \bm{B} \bm{u}_2 \\
	\bm{y}_{2} = \bm{B}^T \diffp{H_2}{\bm{x}_2} + \bm{D} \bm{u}_2 \\
	\end{aligned},
	\right. 
	\end{equation}
\end{minipage} %
\begin{minipage}{.45\linewidth}
	\begin{equation}
	\text{dpH} \left\{ 
	\begin{aligned}
	\diffp{x_1}{t} = \mathcal{J} \diffd{H_1}{x_1} \\
	u_{\partial, 1} = \mathcal{B} \diffd{H_1}{x_1} \\
	y_{\partial, 1} = \mathcal{C} \diffd{H_1}{x_1} \\
	\end{aligned},
	\right.
	\end{equation} 
\end{minipage}

where $\bm{x} \in \mathbb{R}^n, \bm{u}, \bm{y} \in \mathbb{R}^m$ $x_1 \in \mathscr{X}$ and $u_{\partial, 1}  \in \mathscr{U}, \, y_{\partial, 1} \in  \mathscr{Y} = \mathscr{U}^\prime$ belong to some Hilbert spaces (the prime denotes the dual of a space)  and  $\mathcal{B}: \mathscr{X} \rightarrow \mathscr{U}, \; \mathcal{C}: \mathscr{X} \rightarrow \mathscr{Y}$ are boundary operators. The duality pairings for the boundary ports are then denoted by
\[
\left\langle u_{\partial, 1}, \; y_{\partial, 1} \right\rangle_{\mathscr{U} \times \mathscr{Y}},  \qquad
\left\langle \bm{u}_{2}, \; \bm{y}_{2} \right\rangle_{\mathbb{R}^m}.
\]

Given a compact interconnection operator $\mathcal{I}: \mathscr{Y} \rightarrow \mathbb{R}^m$ consider the following power preserving interconnection
\begin{equation}
\label{eq:int_inf}
\bm{u}_2 = \mathcal{I} \, y_{\partial, 1},  \qquad u_{\partial, 1} = - \mathcal{I}^* \, \bm{y}_2,
\end{equation}
where $\mathcal{I}^*$ denotes the adjoint operator of $\mathcal{I}$. As an illustration, a rigid rod welded to the plate is considered. A rigid rod can be written undergoing small displacements about the $z$ axis and small rotation about the $x$ axis can be written as port-Hamiltonian in co-energy variables system with structure
\begin{equation}
\begin{aligned}
\begin{bmatrix}
M_G & 0 \\
0   & J_G \\
\end{bmatrix} 
\displaystyle \diff{}{t}
\begin{pmatrix}
v_G \\ \theta_G \\
\end{pmatrix} & = \begin{pmatrix}
F_z \\ T_x \\
\end{pmatrix} = \bm{u}_{\text{rod}}\vspace{1mm}, \\
\bm{y}_{\text{rod}} &= \begin{pmatrix}
v_G \\ \theta_{G} \\
\end{pmatrix}
\end{aligned}
\end{equation}
with $v_G, \, \theta_{G}$ the linear and angular velocity about $G$, the center of mass, $M, J_G$ the mass and rotary inertia about the $x$ axis and $F_z, T_x$ the force along $z$ and the torque along $x$. The Hamiltonian reads $H_{\text{rod}}  = \frac{1}{2} \left(M_G v_G^2 + J_G \theta_{G}^2 \right)$. The rod is welded to a rectangular thin plate of sides $L_x, L_y$ on side $x = L_x$. The boundary variables for the plate involved in the interconnection are 
\[u_{\partial, \text{pl}} = w_t(x = L_x, y),  \qquad  y_{\partial, \text{pl}} = \widetilde{q}_n(x = L_x, y).
\]
The space $\mathscr{Y}$ is the space of square-integrable functions on with support $\Gamma_{\text{int}} = \left\{ (x,y) \vert \; x=L_x, 0 \le y \le L_y  \right\}$. The compact interconnection operator then reads
\begin{equation}
\mathcal{I} y_{\partial, \text{pl}} = - \int_{\Gamma_{\text{int}}} \begin{pmatrix}
y_{\partial, \text{pl}} \\
y_{\partial, \text{pl}} \left( y - L_y/2 \right) \\
\end{pmatrix} \d{s}
\end{equation}
The adjoint operator is then obtained considering that $\bm{u}_{\text{rod}} = \mathcal{I} y_{\partial, \text{pl}}$ and that the inner product of $\mathbb{R}^m$ is easily converted to an inner product on the space $L^2(\Gamma_{\text{int}})$ (square-integrable functions on $\Gamma_{\text{int}}$)
\[
\left\langle \mathcal{I} y_{\partial, \text{pl}}, \; \bm{y}_{\text{rod}} \right\rangle_{\mathbb{R}^m} = \left\langle \mathcal{I}^* \bm{y}_{\text{rod}} , \; y_{\partial, \text{pl}} \right\rangle_{L^2(\partial\Omega_{\text{int}})},
\]
\[
\mathcal{I}^* \bm{y}_{\text{rod}} = - \left[ v_G + \theta_{G} \left( y - L_y/2 \right) \right].
\]
The interconnection \eqref{eq:int_inf} will assure that the rigid rod is welded in a power preserving manner.
\subsection{Finite dimensional setting}
Consider a rectangular plate of side $L_x, L_y$, clamped at $x=0$ and welded to a rigid rod on $x=L_x$. The discretized system \ref{eq:discr_pl} is now modified to take into account the presence of an inhomogeneous Dirichlet boundary condition, i.e the input needed for the interconnection. It reads
\begin{equation}
\begin{aligned}
\begin{bmatrix}
\bm{M}_{\text{pl}} & \bm{0} \\
\bm{0} & \bm{0} \\
\end{bmatrix} \frac{\d}{\d t}
\begin{pmatrix}
\bm{e}_{\text{pl}}\\
\bm{\lambda}_D \\
\end{pmatrix}
&= \begin{bmatrix}
\bm{J}_d & \bm{G}_D \\
-\bm{G}_D^T & \bm{0} \\
\end{bmatrix}
\begin{pmatrix}
\bm{e}_{\text{pl}}\\
\bm{\lambda}_D \\
\end{pmatrix} + \begin{bmatrix}
\bm{0} \\
\bm{B} \\
\end{bmatrix} \bm{u}_{\text{pl}}, \\
\bm{y}_{\text{pl}} &= \begin{bmatrix}
\bm{0} & \bm{B} \\
\end{bmatrix} \begin{pmatrix}
\bm{e}_{\text{pl}}\\
\bm{\lambda}_D \\
\end{pmatrix}.
\end{aligned} 
\end{equation}
Here the Lagrange multipliers and associated matrices $\bm{\lambda}_D =  (\bm\lambda_{\Gamma_{C}}; \; \bm\lambda_{\Gamma_{\text{int}}} ), \; \bm{G}_D = [\bm{G}_{\Gamma_{C}}, \; \bm{G}_{\Gamma_{\text{int}}}]$ are split between the homogeneous and non-homogeneous Dirichlet boundary conditions. The rigid rod is written


where $\bm{e}_{\text{rod}} = (v_G\; \theta_{G}), \; \bm{M}_{\text{rod}} = \mathrm{diag}(M_G, \, J_G)$


% An example of a floating figure using the graphicx package.
% Note that \label must occur AFTER (or within) \caption.
% For figures, \caption should occur after the \includegraphics.
% Note that IEEEtran v1.7 and later has special internal code that
% is designed to preserve the operation of \label within \caption
% even when the captionsoff option is in effect. However, because
% of issues like this, it may be the safest practice to put all your
% \label just after \caption rather than within \caption{}.
%
% Reminder: the "draftcls" or "draftclsnofoot", not "draft", class
% option should be used if it is desired that the figures are to be
% displayed while in draft mode.
%
%\begin{figure}[!t]
%\centering
%\includegraphics[width=2.5in]{myfigure}
% where an .eps filename suffix will be assumed under latex, 
% and a .pdf suffix will be assumed for pdflatex; or what has been declared
% via \DeclareGraphicsExtensions.
%\caption{Simulation results for the network.}
%\label{fig_sim}
%\end{figure}

% Note that the IEEE typically puts floats only at the top, even when this
% results in a large percentage of a column being occupied by floats.


% An example of a double column floating figure using two subfigures.
% (The subfig.sty package must be loaded for this to work.)
% The subfigure \label commands are set within each subfloat command,
% and the \label for the overall figure must come after \caption.
% \hfil is used as a separator to get equal spacing.
% Watch out that the combined width of all the subfigures on a 
% line do not exceed the text width or a line break will occur.
%
%\begin{figure*}[!t]
%\centering
%\subfloat[Case I]{\includegraphics[width=2.5in]{box}%
%\label{fig_first_case}}
%\hfil
%\subfloat[Case II]{\includegraphics[width=2.5in]{box}%
%\label{fig_second_case}}
%\caption{Simulation results for the network.}
%\label{fig_sim}
%\end{figure*}
%
% Note that often IEEE papers with subfigures do not employ subfigure
% captions (using the optional argument to \subfloat[]), but instead will
% reference/describe all of them (a), (b), etc., within the main caption.
% Be aware that for subfig.sty to generate the (a), (b), etc., subfigure
% labels, the optional argument to \subfloat must be present. If a
% subcaption is not desired, just leave its contents blank,
% e.g., \subfloat[].


% An example of a floating table. Note that, for IEEE style tables, the
% \caption command should come BEFORE the table and, given that table
% captions serve much like titles, are usually capitalized except for words
% such as a, an, and, as, at, but, by, for, in, nor, of, on, or, the, to
% and up, which are usually not capitalized unless they are the first or
% last word of the caption. Table text will default to \footnotesize as
% the IEEE normally uses this smaller font for tables.
% The \label must come after \caption as always.
%
%\begin{table}[!t]
%% increase table row spacing, adjust to taste
%\renewcommand{\arraystretch}{1.3}
% if using array.sty, it might be a good idea to tweak the value of
% \extrarowheight as needed to properly center the text within the cells
%\caption{An Example of a Table}
%\label{table_example}
%\centering
%% Some packages, such as MDW tools, offer better commands for making tables
%% than the plain LaTeX2e tabular which is used here.
%\begin{tabular}{|c||c|}
%\hline
%One & Two\\
%\hline
%Three & Four\\
%\hline
%\end{tabular}
%\end{table}


% Note that the IEEE does not put floats in the very first column
% - or typically anywhere on the first page for that matter. Also,
% in-text middle ("here") positioning is typically not used, but it
% is allowed and encouraged for Computer Society conferences (but
% not Computer Society journals). Most IEEE journals/conferences use
% top floats exclusively. 
% Note that, LaTeX2e, unlike IEEE journals/conferences, places
% footnotes above bottom floats. This can be corrected via the
% \fnbelowfloat command of the stfloats package.




\section{Conclusion}
The conclusion goes here.

\addtolength{\textheight}{-12cm}   % This command serves to balance the column lengths
% on the last page of the document manually. It shortens
% the textheight of the last page by a suitable amount.
% This command does not take effect until the next page
% so it should come on the page before the last. Make
% sure that you do not shorten the textheight too much.





% conference papers do not normally have an appendix


% use section* for acknowledgment
\section*{Acknowledgment}


The authors would like to thank Michel Sala\"n, Xavier Vaisseur and Ghislain Haine from ISAE-SUPAERO for their insightful observations and comments.



\bibliographystyle{IEEEtran}
\bibliography{biblio_CDCKirchhoff}


	
	% that's all folks
\end{document}


