\documentclass[11pt]{letter}
 \usepackage[utf8]{inputenc} % un package
 \usepackage[T1]{fontenc} % un second package
 \usepackage[francais]{babel} % un troisième package
 \usepackage{textcomp}
 \usepackage{comment}
 \usepackage{ifpdf}
\ifpdf
 \usepackage[pdftex]{graphicx}
 \else
 \usepackage[dvips]{graphicx}\fi
\pagestyle{empty}
\usepackage[top=0.5cm, bottom=0.5cm, left=1.5cm, right=1.5cm]{geometry}
 \setlength{\parindent}{0pt}
 \addtolength{\parskip}{6pt}
\renewcommand{\ttdefault}{pcr}

 \begin{document}
 \sffamily
 \hfill
 %\hfill
 %
 \begin{flushleft}
 {\bfseries Andrea \textsc{BRUGNOLI}}\\[.35ex]
% \small\itshape
% 1 Avenue de Rangueil\\
% 31400, Toulouse\\[.35ex]
% 0033 7 50 39 47 27 \\
 Andrea.BRUGNOLI@isae.supaero.fr or andrea.brugnoli92@gmail.com
 \end{flushleft}
 %
 \begin{flushright}
 {\bfseries Airbus Defence \& Space}\\[.35ex]
 \small\itshape
 31 Rue des Cosmonautes\\
 31400, Toulouse
 \end{flushright}
 %
 %\hfill
 %
 \begin{flushright}
 Toulouse, \today 
 \end{flushright}
 %
 \textbf{Subject}: Application for the following internships:
\begin{itemize}
\setlength\itemsep{0.4pt}
\item Hybridization of Heuristic and Deterministic Optimization in Spacecraft Trajectory Design 
\item Advanced control for agile flexible spacecraft with hardware demonstration 
\item Innovative Vision-Based Navigation for Orbital Rendezvous 
\end{itemize}

 Dear Sir, Madame

Airbus Defence \& Space reputation is daily confirmed by the growing number of successful projects. Its technical expertise is
worldwide recognised and testified by several collaborations with other international organisations, such as ESA or NASA. Your enterprise has always distinguished itself thanks to projects of huge scientific impact. The launch of the Biomass satellite and the Zephyr for example are among those and demonstrate its commitment in the present and the past. Being able to contribute to equally important projects and being surrounded by a stimulating, international and intensely engaged team are reasons of great enthusiasm and trepidation for me. This is why I am applying for an internship at Airbus Defence \& Space.


%After a reorganisation within Airbus group company Airbus Defence \& Space was born in 2014. This division is well-known for its engagement towards the environmental and humanitarian cause. Recently Airbus D\&S signed a contract with ESA . Its will measure forest biomass to assess carbon stocks and fluxes for five years. This mission can be extremely useful to provide a better understanding of world's climate and to discover in new water sources in arid regions. Among Airbus D\&S products the Zephyr is one of the most innovative and fascinating. Thanks to its solar cells it could hover above the clouds endlessly and soon it will replace conventional UAVs, saving tons of fuel. These and many others projects that are being developed at Airbus Defence \& Space pushed me to apply at Airbus Defence \& Space for this internship.

 I specialised in mechanics and space engineering during my four years of study at Politecnico di Milano. I always engaged in my studies and thanks the achieved results I had the possibility to be involved in insightful experiences. During my second year of bachelor I could join an intensive formation in Labview, in partnership with National Instrument. The following year I worked on the kinematic and dynamic analysis of a forging manipulator. This project was carried out in partnership with Danieli Spa, an Italian multinational company. Our group was selected among twenty to give a final presentation in front the company engineers. 
 %I especially enjoyed working on an interplanetary transfer project. In  challenging space missions many different disciplines have to be integrated and this is why I find it an intriguing subject.  Later on I had the occasion to work on multi-body dynamics in Simulink environment to study appendices flexibilities during satellites manoeuvers.  All those projects provided me a pragmatic formation combined with a theoretical background. 
 After my forth year I decided to engage in a new experience and to enrol myself at Supaero. Here I was challenged by a different pedagogic approach and destabilised by a different environment.  I learnt how to adapt myself and then found my path in the educational offer. I have always been amazed by how math can unveil the mysteries of nature. At the same time engineering is the medium by which we can obverse math potentialities at work. In particular automatics keeps changing and improving our world  and way of living. These are reasons why I am now attending courses in applied mathematics  and advanced automatics. The former focus on deterministic optimization, the latter on the dynamics of complex mechanical systems and model identification. My strong
interest for the second one pushed me to enrol in a research master in automatics and image processing at university Paris Saclay Sud. 

I would love to apply my competences in the space field. Space applications contribute to worldwide cooperation.  They had, have and will always have a key role in developing crucial technologies. Space research is the highest frontier of human progress and will allow mankind to discover before inconceivable mission scenarios. Airbus Defence \& Space could give me the possibility to express myself in a stimulating environment. I am confident that I possess all is needed to work in a multicultural enterprise.  I have no doubt that this company can enrich me in terms of technical capabilities and human values. In return I want to invest all my energies to benefit from this opportunity.

I stay at your disposal in case you might need contact me, at any time of your convenience.
Sincerely
 \begin{center}
 \large\textit{Andrea Brugnoli}
 \end{center}
 \end{document}
