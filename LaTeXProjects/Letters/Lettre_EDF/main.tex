\documentclass[11pt]{letter}
 \usepackage[utf8]{inputenc} % un package
 \usepackage[T1]{fontenc} % un second package
 \usepackage[english]{babel} % un troisième package
 \usepackage{textcomp}
 \usepackage{comment}
 \usepackage{ifpdf}
\ifpdf
 \usepackage[pdftex]{graphicx}
 \else
 \usepackage[dvips]{graphicx}\fi
\pagestyle{empty}
\usepackage[top=0.5cm, bottom=0.5cm, left=1.5cm, right=1.5cm]{geometry}
 \setlength{\parindent}{0pt}
 \addtolength{\parskip}{6pt}
\renewcommand{\ttdefault}{pcr}
 \begin{document}
 \sffamily
 %\hfill
 %
 \begin{flushleft}
 {\bfseries Andrea \textsc{BRUGNOLI}}\\[.35ex]
% \small\itshape
% 1 Avenue de Rangueil\\
% 31400, Toulouse\\[.35ex]
% 0033 7 50 39 47 27 \\
 Andrea.BRUGNOLI@isae.supaero.fr or andrea.brugnoli92@gmail.com\\
 +33750394727
 \end{flushleft}
 %
 \begin{flushright}
 {\bfseries EDF}\\[.35ex]
 \small\itshape
 Avenue de Wagram\\
75008 Paris  
 \end{flushright}
 %
 %\hfill
 %
 \begin{flushright}
 Toulouse, \today 
 \end{flushright}
 %
 \textbf{Sujet}: Candidature pour l'offre de stage "Détection de fuites dans les conduites forcées des aménagements hydroélectriques" 

Cher Madame, Monsieur\\  
EDF représente l'énergie en Europe. Parmi les nombreux sources énergétiques utilisées par EDF le nucléaire a eu toujours un rôle fondamental. Grâce à cette source la France à obtenu l'indépendante énergétique et peut vanter une expertise remarquable dans ce secteur industriel, le troisième, après l'automobile et l'aéronautique, du pays. Votre maîtrise technique est aussi accompagnée par une grande attention au problème de la durabilité: en 2016 DJSI World a élu EDF parmi les 10\% meilleurs entreprise dans cette thématique. Les performances économiques, environnementales, sociales montrent des résultats stupéfiants en continue augmentation. Pour ces raisons je me sens très motivé à postuler chez vous.

Je me suis spécialisée en ingénierie mécanique et spatial au cours de mes quatre années d'études au Politecnico di Milano. En suit à l'obtention de résultats notables j'ai eu la possibilité de participer à plusieurs expériences. En particulier, pendant mon deuxième ans de baccalauréat, j'ai pu rejoindre une formation intensive dans Labview, en partenariat avec National Instrument. L'année suivante, j'ai travaillé sur la cinématique et l'analyse dynamique d'un manipulateur de forge. Ce projet a été réalisé en partenariat avec Danieli Spa, une société multinationale italienne. Notre groupe a été choisi parmi vingt pour donner une présentation finale devant les ingénieurs de l'entreprise. 
\begin{comment}
J'ai spécialement apprécié de travailler sur un projet de transfert interplanétaire. Dans des missions spatiales complexes différentes disciplines doivent être intégrées et c'est la raison pour la quelle j'ai travaillé passionnément au projet. Plus tard, j'ai eu l'occasion de travailler sur la dynamique multi-corps dans l'environnement Simulink pour étudier les flexibilités des appendices au cours de manoeuvres de satellites.
\end{comment}
 Ces et autres projets m'ont fourni une formation pragmatique combiné avec une formation théorique. Après ma quatrième année j'ai décidé de m'engager dans une nouvelle expérience et de m'inscrire à Supaero. Ici, j'ai été déstabilisé par une approche pédagogique très diffèrent. J'ai appris à m'adapter et ensuite trouvé mon parcours dans l'offre de formation. J'ai toujours été étonné par le fait que les mathématiques peuvent dévoiler les mystères de la nature. Au même temps, l'ingénierie est le moyen par lequel les potentialités de les mathématiques peuvent être observées à l'oeuvre. En particulier l'automatique ne cesse de changer et d'améliorer notre monde et notre mode de vie. Ces sont les raisons pour lesquelles je suis maintenant en train de me spécialiser en mathématiques appliquées et de l'automatique avancée. Le première est focalisé sur l'optimisation déterministes et multidisciplinaire, le deuxième sur la dynamique de systèmes mécaniques complexes et d'identification du modèle. Ma forte passion pour le dernière m'a encouragé à m'inscrire à un Master Recherche en automatique et traitement d'image organisé par l'université Paris Saclay Sud. %Grâce à cette formation complémentaire je pourrai approfondir et consolider mes connaissance théoriques dans ce domaine.

Ce stage réponds parfaitement à mes atteintes. Il conjugue les mathématiques appliquées et concepts avancés de contrôle. EDF peut me donner la possibilité de m'exprimer dans un environnement stimulant et je suis persuadé de posséder tout le nécessaire pour travailler chez vous. Je n'ai aucun doute que cette entreprise peut m'enrichir en termes de capacités techniques et de valeurs humaines. En retour, je veux investir toutes mes énergies pour bénéficier de cette opportunité.

Je reste à votre disposition au cas où vous pourriez avoir besoin de me contacter, à tout moment de votre convenance. Dans l'atteinte d'une réponse de votre part vouliez accepter l'expression de mes salutations distinguées. Cordialement


 \begin{center}
 \large\textit{Andrea Brugnoli}
 \end{center}
 \end{document}
 