\documentclass[11pt]{letter}
\usepackage[utf8]{inputenc} % un package
\usepackage[T1]{fontenc} % un second package
\usepackage[english]{babel} % un troisième package
\usepackage{textcomp}
\usepackage{comment}
\usepackage{ifpdf}
\ifpdf
\usepackage[pdftex]{graphicx}
\else
\usepackage[dvips]{graphicx}\fi
\pagestyle{empty}
\usepackage[top=0.5cm, bottom=0.5cm, left=1.5cm, right=1.5cm]{geometry}
\setlength{\parindent}{0pt}
\addtolength{\parskip}{6pt}
\renewcommand{\ttdefault}{pcr}
\begin{document}
	\sffamily
	%\hfill
	%
	\begin{flushleft}
		{\bfseries Andrea \textsc{BRUGNOLI}}\\[.35ex]
		% \small\itshape
		% 1 Avenue de Rangueil\\
		% 31400, Toulouse\\[.35ex]
		% 0033 7 50 39 47 27 \\
		Andrea.BRUGNOLI@utwente.nl ou andrea.brugnoli92@gmail.com
	\end{flushleft}
	%
	\begin{flushright}
		{\bfseries Technical University of Berlin}\\[.35ex]
		\small\itshape
		Straße des 17. Juni 135,\\
		 10623 Berlin, Germania
	\end{flushright}
	%
	%\hfill
	%
	\begin{flushright}
		Enschede (NL), \today 
	\end{flushright}
	%
	\textbf{Subject}: Cover letter for Post-Doc position on numerics for hierarchical port-Hamiltonian models for smart grids
	
	
	To the attention of the recruitment committee.\\
	
	The expertise of the research staff and the numerous collaborations with industrial and academic partners are undeniable elements of the Technical University of Berlin's reputation. The university, whose image and prestige are recognized worldwide, has always been distinguished by its strong participation in projects with a significant scientific impact. The Department of Numerical Mathematics seeks to improve the understanding required to tackle the many technological, societal and economical challenges of the present and future. Being able to contribute to such projects and being surrounded by a stimulating, international and intensely committed environment are a reason of great enthusiasm for me.
	
	I studied mechanical and space engineering during my four years of study at Politecnico di Milano. After my fourth year, I enrolled in a double degree at ISAE-SUPAERO, where I specialized in applied mathematics and advanced automatic control. My strong passion for this latter subject encouraged me to enroll in a Research Master organized by the University Paris-Saclay. In November 2020, I obtained my thesis, titled \textit{A port-Hamiltonian formulation of flexible structures. Modelling and structure-preserving finite element discretization}, under the supervision of Daniel Alazard, Valérie Pommier-Budinger and Denis Matignon. The thesis was focused on the development of discretization schemes preserving the physical structure for the dynamics of thin flexible structures. It is an interdisciplinary subject, at the intersection between system theory, mechanics, scientific computing and numerical analysis. During my thesis, I did an exchange with the Instituto Tecnológico de Aeronáutica (ITA) in Brazil, to work with prof. Cardoso-Ribeiro. After my PhD thesis I joined as a post-Doc the Portwings project, funded by an ERC advanced grant and leaded by prof. Stefano Stramigioli. My mission is to set up geometrical structure preserving discretizations for distributed port-Hamiltonian systems.
	
	This post-Doc gives me the occasion to deepen my knowledge in my research field, in particular the aspects related to the actual implementation of simulation tools for network systems and complex applications more generally. I believe that my competences perfectly match the ones required for this position.
	
 
	I remain at your disposal for any further information. 
	
	
	
	
	\begin{center}
		\large\textit{Andrea Brugnoli}
	\end{center}
\end{document}