\documentclass[11pt]{letter}
 \usepackage[utf8]{inputenc} % un package
 \usepackage[T1]{fontenc} % un second package
 \usepackage[francais]{babel} % un troisième package
 %\usepackage{comment}
 \usepackage{textcomp}
 \usepackage{ifpdf}
\ifpdf
 \usepackage[pdftex]{graphicx}
 \else
 \usepackage[dvips]{graphicx}\fi
\pagestyle{empty}
\usepackage[top=0.5cm, bottom=0.5cm, left=1.5cm, right=1.5cm]{geometry}
 \setlength{\parindent}{0pt}
 \addtolength{\parskip}{6pt}
\renewcommand{\ttdefault}{pcr}
 \begin{document}
 %\sffamily
 %\hfill
 %
 \begin{flushleft}
 {\bfseries Mr. Andrea \textsc{BRUGNOLI}}\\[.35ex]
% \small\itshape
% 1 Avenue de Rangueil\\
% 31400, Toulouse\\[.35ex]
% 0033 7 50 39 47 27 \\
 Andrea.BRUGNOLI@isae.fr or andrea.brugnoli92@gmail.com
 \end{flushleft}
 %
 \begin{flushright}
 {\bfseries The PortWings Project}\\[.35ex]
 \small\itshape
 CVJM-Bildungstätte \\
Bundeshöhe 7 \\
Wuppertal 42285
 \end{flushright}
 %
 %\hfill
 %
 \begin{flushright}
 Toulouse, \today 
 \end{flushright}
 %
 \textbf{Subject}: Motivation letter for Post-Doc position in structure preserving discretization

 Dear Sir, Madame

The port-Hamiltonian formalism reputation has risen in the last years, as testified by the growing number of successful projects. INFIDHEM is a perfect example of this tendency. Several universities and researcher now join their efforts to make significant advancements in this new and promising mathematical framework. Many different topics are under investigation and six different universities now participate to the project. Undoubtedly in the following months insightful publications and results will be produced as a result of this fruitful academic collaboration.

 I specialised in mechanics and space engineering during my four years of study at Politecnico di Milano. I always engaged in my studies and thanks the achieved results I had the possibility to be involved in insightful experiences. During my second year of bachelor I could join an intensive formation in Labview, in partnership with National Instrument. The following year I worked on the kinematic and dynamic analysis of a forging manipulator. This project was carried out in partnership with Danieli Spa, an Italian multinational company.  After my forth year I decided to engage in a new experience and to enrol myself at ISAE-Supaero. There I was challenged by a different pedagogic approach and destabilised by a different environment.  I have always been amazed by how math can unveil the mysteries of nature. At the same time engineering is the medium by which we can obverse math potentialities at work. In particular automatics keeps changing and improving our world  and way of living. My strong interest for this discipline pushed me to enrol in a research master in automatics and image processing at university Paris Saclay Sud. For these reasons I am now pursuing a PHD at ISAE-SUPAERO, under the supervision of Daniel and Valérie Budinger. The object of my research is the "Modelling and control by the Port-Hamiltonian formalism of 2D flexible structures with varying boundary conditions."

I am confident that this summer school will provide me the notions and tools of Functional analysis that are necessary to understand the mathematical foundations of the port-Hamiltonian formalism.  I have no doubt that this experience can enrich me in terms of technical capabilities and human values. In return I want to invest all my energies to benefit from this opportunity.

I stay at your disposal in case you might need contact me, at any time of your convenience.
Sincerely
 \begin{center}
 \large\textit{Andrea Brugnoli}
 \end{center}
 \end{document}

