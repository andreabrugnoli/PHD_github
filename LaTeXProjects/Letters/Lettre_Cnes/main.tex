\documentclass[11pt]{letter}
 \usepackage[utf8]{inputenc} % un package
 \usepackage[T1]{fontenc} % un second package
 \usepackage[english]{babel} % un troisième package
 \usepackage{textcomp}
 \usepackage{comment}
 \usepackage{ifpdf}
\ifpdf
 \usepackage[pdftex]{graphicx}
 \else
 \usepackage[dvips]{graphicx}\fi
\pagestyle{empty}
\usepackage[top=0.5cm, bottom=0.5cm, left=1.5cm, right=1.5cm]{geometry}
 \setlength{\parindent}{0pt}
 \addtolength{\parskip}{6pt}
\renewcommand{\ttdefault}{pcr}
 \begin{document}
 \sffamily
 %\hfill
 %
 \begin{flushleft}
 {\bfseries Andrea \textsc{BRUGNOLI}}\\[.35ex]
% \small\itshape
% 1 Avenue de Rangueil\\
% 31400, Toulouse\\[.35ex]
% 0033 7 50 39 47 27 \\
 Andrea.BRUGNOLI@isae.supaero.fr or andrea.brugnoli92@gmail.com
 \end{flushleft}
 %
 \begin{flushright}
 {\bfseries Cnes}\\[.35ex]
 \small\itshape
 18 Avenue Edouard Belin\\
 31400, Toulouse, France
 \end{flushright}
 %
 %\hfill
 %
 \begin{flushright}
 Toulouse, \today 
 \end{flushright}
 %
 \textbf{Subject}: Réponse à les offres des stages suivantes:

Madame, Monsieur \\
Le nombre, toujours croissant, de projets ayant une bonne réussite, l'expertise du personnel technique et les nombreuses collaborations avec des organismes de recherche français et internationaux constituent  des éléments indéniables de la réputation du Cnes, dont l'image et le prestige sont reconnus au niveau mondial. Votre enterprise s'est toujours distinguée par sa forte participation dans des projets d'impact scientifique important. Rosetta et Bepicolombo par exemple figurent dans cet ensemble et représentent un témoignage de l'excellence du Cnes dans le passé et le présent. Le fait de pouvoir contribuer à des projets d'une telle portée et de pouvoir être entouré par une équipe stimulante, internationale et intensément engagée sont une raison de grand enthousiasme pour moi. 

Je me suis spécialisée en ingénierie mécanique et spatiale au cours de mes quatre années d'études au Politecnico di Milano. Grâce à l'obtention de résultats remarquables, j'ai eu la possibilité de participer à plusieurs expériences. En particulier, pendant ma deuxième année de baccalauréat, j'ai pu rejoindre une formation intensive dans Labview, en partenariat avec National Instrument. L'année suivante, j'ai travaillé sur la cinématique et l'analyse dynamique d'un manipulateur de forge. Ce projet a été réalisé en partenariat avec Danieli Spa, une société multinationale italienne. Notre groupe a été choisi parmi vingt pour donner une présentation orale devant les ingénieurs de l'entreprise.
\begin{comment}
J'ai spécialement apprécié de travailler sur un projet de transfert interplanétaire. Dans des missions spatiales complexes différentes disciplines doivent être intégrées et c'est la raison pour la quelle j'ai travaillé passionnément au projet. Plus tard, j'ai eu l'occasion de travailler sur la dynamique multi-corps dans l'environnement Simulink pour étudier les flexibilités des appendices au cours de manoeuvres de satellites.
\end{comment}
 Ce projet et d'autres m'ont fourni une formation à la fois pragmatique et théorique. Après ma quatrième année, j'ai décidé de m'engager dans une nouvelle expérience et de m'inscrire à Supaero. Ici, j'ai été confronté à une approche pédagogique très différente et j'ai pu trouver ma voie dans la riche offre de formation. J'ai toujours été étonné par la façon dont les mathématiques peuvent dévoiler les mystères de la nature. De plus, l'ingénierie est le moyen par lequel les potentialités des mathématiques peuvent être observées à l'oeuvre. En particulier l'automatique ne cesse de changer et d'améliorer notre monde et notre mode de vie. Ce sont les raisons pour lesquelles je suis maintenant en train de me spécialiser en mathématiques appliquées, focalisés sur l'optimisation déterministes,  et en automatique avancée, centrée sur la dynamique de systèmes mécaniques complexes et l'identification du modèle. Ma forte passion pour cette dernière matière m'a encouragé à m'inscrire dans un Master Recherche en automatique et traitement d'image organisé par l'université Paris Saclay Sud. 

J'aimerais appliquer mes compétences dans le domaine spatial. Les applications spatiales ont eu, ont et auront toujours un rôle clé dans le développement de technologies cruciales. La recherche spatiale est la plus haute expression du progrès humain et permettra à l'humanité de découvrir d'inconcevables scénarios. Le Cnes peut me donner la possibilité de m'exprimer dans un environnement stimulant et vos offres correspondent parfaitement à mes attentes. Je n'ai aucun doute sur le fait que cette entreprise puisse m'enrichir en termes de capacités techniques et de valeurs humaines. En retour, je veux investir toute mon énergie pour bénéficier de cette opportunité.

Je reste à votre disposition pour me contacter à votre convenance. Dans l'attente d'une réponse de votre part, veuillez accepter l'expression de mes salutations distinguées. Cordialement


 \begin{center}
 \large\textit{Andrea Brugnoli}
 \end{center}
 \end{document}