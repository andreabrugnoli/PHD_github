\documentclass{article}

\usepackage{filecontents}

\begin{filecontents}{\jobname.bib}
	@article{MATIGNONdamp,
		title = "A class of damping models preserving eigenspaces for linear conservative port-{H}amiltonian systems",
		journal = "European Journal of Control",
		volume = "19",
		number = "6",
		pages = "486 - 494",
		year = "2013",
		note = "Special issue of LHMNLC",
		issn = "0947-3580",
		doi = "https://doi.org/10.1016/j.ejcon.2013.10.003",
		url = "http://www.sciencedirect.com/science/article/pii/S0947358013001672",
		author = "Denis Matignon and Thomas Hélie",
		keywords = "Energy storage, Port-Hamiltonian systems, Eigenfunctions, Damping, Caughey series, Partial differential equations, Fractional Laplacian",
	}
@INPROCEEDINGS{mehrmann2019structurepreserving,
	title={Structure-preserving discretization for port-{H}amiltonian descriptor systems},
	author={Volker Mehrmann and Riccardo Morandin},
	booktitle={Proceedings of the 59th IEEE Conference on Decision and Control}, 
	year={2019},
	pages = {6663 - 6868},
}
\end{filecontents}

\bibliographystyle{plain}


\title{Review to paper 0205}

\begin{document}
	\maketitle
	
	The paper is well written and interesting. Publication is recommended.
	However, this reviewer have some remarks. 
	\begin{enumerate}
		\item Concerning Eq. 5, the equations can be written in pH form using non-conservative variables but that requires special care. The problem clearly arises for the fact that the Hamiltonian should be defined w.r.t. a \textit{weighted} inner product, where the cross section represent the weight. 
		\begin{itemize}
			\item To obtain a suitable pH formulation you can follow example 5 of \cite{MATIGNONdamp} and obtain a system as Eq. 28 therein. The skew-symmetric operator $\mathcal{J}_{MH}$ in Eq. 28 of \cite{MATIGNONdamp} is formally skew-adjoint w.r.t. the weighted $L^2$ inner product. 
			\item Or, equivalently you can  rely on the formulation detailed in \cite{mehrmann2019structurepreserving}. In this case the presence of a mass matrix $\mathcal{E} = \mathrm{Diag}(A, A)$, imposes the use of effort function $z$ such that $\delta_\alpha \mathcal{H}_s = \mathcal{E}^* z$, where $\alpha$ is the state variable. The skew-symmetric operator is simply $\mathcal{E} \mathcal{J}_{MH}$. 
		\end{itemize}
		In either case, the formulation corresponds to a port-Hamiltonian system with a linear interconnection operator. This reviewer recommends that you take all this into account and cite the appropriate references. 
		\item Secondly, the boundary variables appearing in the Stokes-Dirac structure of Proposition 3 and 4 should be related to the physical variables and the boundary conditions of the classical PDE problem. This reviewer recommends to clearly state the physical meaning  of the boundary variables in each Stokes-Dirac structure and the connection with the classical boundary conditions of problem 2 and 3.
	\end{enumerate}
	
	
	
	
	
	\bibliography{\jobname}

	
\end{document}