\documentclass[aspectratio=169, french]{ISAE-Beamer}
\usepackage{fontspec}
%\usepackage[french]{babel}
\usefonttheme[onlymath]{serif}
\usepackage{amsmath,amssymb,amsthm}
\usepackage{arydshln,mathtools}
\usepackage{bm}
\usepackage{color}
\definecolor{theme}{RGB}{0,73,114}
\usepackage{multicol}
%\usepackage[caption=false]{subfig}
\usepackage{subcaption}

\usepackage{comment}

\usepackage{graphicx}
\usepackage{diffcoeff}
\usepackage{dsfont}
\usepackage{mathrsfs}
\usepackage[most]{tcolorbox}

\usepackage{xspace}
\usepackage{appendixnumberbeamer}


\usepackage{media9}
\usepackage[backend=bibtex,style=verbose,doi=false,isbn=false,url=false,eprint=false,autocite=footnote]{biblatex}

\addtobeamertemplate{footnote}{\vspace{-6pt}\advance\hsize-0.5cm}{\vspace{6pt}}
\makeatletter
% Alternative A: footnote rule
\renewcommand*{\footnoterule}{\kern -3pt \hrule \@width 2in \kern 8.6pt}
% Alternative B: no footnote rule
% \renewcommand*{\footnoterule}{\kern 6pt}
\makeatother

\graphicspath{{./images/}}

\bibliography{biblioTrEquation}

% Math macros
\DeclareMathOperator*{\grad}{grad}
\DeclareMathOperator*{\Grad}{Grad}
\DeclareMathOperator*{\Div}{Div}
\renewcommand{\div}{\operatorname{div}}
\DeclareMathOperator*{\Hess}{Hess}
\DeclareMathOperator*{\curl}{curl}
\DeclareMathOperator{\Tr}{Tr}
\DeclareMathOperator{\Dom}{Dom}
\DeclareMathOperator*{\esssup}{ess\,sup}

\newcommand{\bbR}{\mathbb{R}}
\newcommand{\bbF}{\mathbb{F}}
\newcommand{\bbA}{\mathbb{A}}
\newcommand{\bbB}{\mathbb{B}}
\newcommand{\bbS}{\mathbb{S}}

\newcommand*{\norm}[1]{\ensuremath{\left\|#1\right\|}}
\newcommand{\where}{\qquad \text{where} \qquad}
\newcommand{\inner}[3][]{\ensuremath{\left\langle #2, \, #3 \right\rangle_{#1}}}
\newcommand{\bilprod}[2]{\left\langle \left\langle \, #1, #2 \, \right\rangle \right\rangle}
\newcommand{\pder}[2]{\ensuremath{\partial_{#2} #1}}
\newcommand{\dder}[2]{\ensuremath{\delta_{#2} #1}}
\newcommand{\secref}[1]{\S\ref{#1}}
\newcommand{\energy}[1]{\frac{1}{2} \int_{\Omega} \left\{ #1 \right\} \d\Omega}
\newcommand{\crmat}[1]{\ensuremath{\left[#1\right]_\times}}
\newcommand{\fenics}{\textsc{FEniCS}\xspace}
\newcommand{\firedrake}{\textsc{Firedrake}\xspace}

\DeclareMathOperator*{\argmax}{arg\,max}
\DeclareMathOperator*{\argmin}{arg\,min}

\newtheorem{proposition}{Proposition}
\newtheorem{remark}{Remark}
\newtheorem{hypothesis}{Hypothesis}
\newtheorem{assumption}{Assumption}
\newtheorem{conjecture}{Conjecture}


\def\onedot{$\mathsurround0pt\ldotp$}
\def\cddot{% two dots stacked vertically
	\mathbin{\vcenter{\baselineskip.67ex
			\hbox{\onedot}\hbox{\onedot}}%
}}

\renewcommand\bibfont{\scriptsize}


\makeatletter \renewcommand\d[1]{\ensuremath{%
		\;\mathrm{d}#1\@ifnextchar\d{\!}{}}}
\makeatother


\graphicspath{{./imagesEqTr/}}

\title[DF Transport 1D]{Méthode des différences finies pour l'EDP de transport 1D}


\author[Andrea Brugnoli]{Andrea Brugnoli}

\date[11/4/22]{11 Avril 2022}

%\thanks{}



\begin{document}
	
	
	\maketitle
	
	
	\begin{frame}{Outline}
		
		\tableofcontents
		
	\end{frame}

\section{Équation du transport 1D : le cas continu}

\begin{frame}{Équation du transport 1D}
	\begin{tcolorbox}[title = L'EDP la plus simple, coltitle=white]
		L'évolution d'un champ scalaire $u(x, t)$ transporté par un fluide satisfait 
		\begin{equation*}
			\diffp{u}{t} + \diffp{q}{x} = 0, \qquad x \in [0, l], \quad t \in  (0, T]. 
		\end{equation*}
	\end{tcolorbox}

\begin{tcolorbox}
	Pour le flux $q(u, x, t)$ on considère une vitesse constante pour le fluide
	\begin{equation*}
		q(u, x, t) = c \ u(x, t), \qquad c>0.
	\end{equation*}
	Le problème est bien posé lorsque on spécifie 
	\begin{align*}
		u(x, 0) &= g(x), \qquad \text{Donnée initiale}, \\
		u(0, t) &= f(t), \qquad \text{Condition au bord (compatible, i.e. $f(0) = g(0)$).}
	\end{align*}
\end{tcolorbox}

\end{frame}

\begin{frame}{Solution Analytique}
En utilisant un argument géométrique ou algébrique, on obtient la solution analytique :
\begin{equation*}
	u(x, t) = g(x - ct), \qquad\text{i.e. $u$ constant sur $\gamma$ telle que $\dot{\gamma} = (c ,1)$.}
\end{equation*}

\begin{columns}
	\begin{column}{.45\textwidth}
	Exemple : 
	\begin{align*}
		c=2, \quad l=20, \quad T = 10. \\
		g(x) = \exp(-x^2/4), \\
		f(t) = \exp(-t^2/4),
	\end{align*}
	donne $u(x, t)= \exp(-(x-2t)^2/4)$.
	\end{column}
	\begin{column}{.5\textwidth}
		\begin{figure}
			\centering
			\includegraphics[height=.74\textheight]{u_sol.eps}
		\end{figure}
	\end{column}
\end{columns}	

\end{frame}

\section{Discrétisation par différence finies}

\begin{frame}{}

\end{frame}
	
	
\end{document}