\documentclass[11pt]{letter}
\usepackage[utf8]{inputenc} % un package
\usepackage[T1]{fontenc} % un second package
\usepackage[english]{babel} % un troisième package
\usepackage{textcomp}
\usepackage{comment}
\usepackage{ifpdf}
\ifpdf
\usepackage[pdftex]{graphicx}
\else
\usepackage[dvips]{graphicx}\fi
\pagestyle{empty}
\usepackage[top=0.5cm, bottom=0.5cm, left=1.5cm, right=1.5cm]{geometry}
\setlength{\parindent}{0pt}
\addtolength{\parskip}{6pt}
\renewcommand{\ttdefault}{pcr}
\begin{document}
	\sffamily
	%\hfill
	%
	\begin{flushleft}
		{\bfseries Andrea \textsc{BRUGNOLI}}\\[.35ex]
		% \small\itshape
		% 1 Avenue de Rangueil\\
		% 31400, Toulouse\\[.35ex]
		% 0033 7 50 39 47 27 \\
		Andrea.BRUGNOLI@utwente.nl ou andrea.brugnoli92@gmail.com
	\end{flushleft}
	%
	\begin{flushright}
		{\bfseries ISAE-SUPAERO}\\[.35ex]
		\small\itshape
		10 Av. Edouard Belin,  \\
		31400 Toulouse, France.
	\end{flushright}
	%
	%\hfill
	%
	\begin{flushright}
		Enschede (NL), \today 
	\end{flushright}
	%
	\textbf{Sujet}: Candidature \`a la position suivante:\\
	\textit{Enseignant-Chercheur en Méthodes Mathématiques pour le Calcul Scientifique}
	
	
	A l'attention du comité de recrutement, \\
	Le nombre, toujours croissant, de publications ayant un fort impact, l'expertise du personnel chercheur et les nombreuses collaborations avec les principaux acteurs industriels et académiques constituent des éléments indéniables de la réputation de l'ISAE-SUPAERO. L'université, dont l'image et le prestige sont reconnus au niveau mondial, s'est toujours distinguée par sa forte participation dans des projets d'impact scientifique important. Le fait de pouvoir contribuer à des projets d'une telle portée et d'être entouré par un environnement stimulante, internationale et intensément engagée sont une raison de grand enthousiasme pour moi. 
	
	Je me suis spécialisée en ingénierie mécanique et spatiale au cours de mes quatre années d'études au Politecnico di Milano. Après ma quatrième année, j'ai effectue un double diplôme \`a l'ISAE-SUPAERO, o\`u je me suis spécialisé en mathématiques appliquées et en automatique avancée. Ma forte passion pour cette dernière matière m'a encouragé à m'inscrire dans un Master Recherche organisé par l'université Paris-Saclay. J'ai effectué mon stage de fin d'etudes au CNES, en travaillant sur la simulation du comportement couplé attitude-orbite pour le débris spatiaux. En Novembre 2020, j'ai obtenu ma thèse, intitulée \textit{Une formulation port-Hamiltonienne des structures flexibles. Modélisation et discrétisation symplectique par éléments finis}. Ce travail, financé par l'ISAE-SUPAERO, a été dirigé par Daniel Alazard, Valérie Pommier-Budinger et Denis Matignon. La thèse a été centrée sur le développement des chemins de discrétisation préservant la structure physique pour la dynamique des structures flexible minces. Il s'agit d'un sujet interdisciplinaire, \`a l'intersection entre la théorie des systèmes, la mécanique, le calcul scientifique et l'analyse numérique. Pendant ma thèse, j'ai pu effectué un échange avec l'Instituto Tecnológico de Aeronáutica (ITA) en Brésil, pour travailler avec prof. Cardoso-Ribeiro. Notre collaboration continue \`a ce jour, avec la supervision d'un étudiant de l'ITA en double diplôme \`a l'Universit\'e de Twente.
	
	Ma recherche se situe entre les calcul scientifique et le systèmes dynamiques. En premier lieu, je souhaite continuer mon travail sur la discrétisation structurée des modèles physiques port-Hamiltoniens, pour développer une librairie de calcul pour le problèmes multiphysiques. En suite, j'envisage approfondir mes connaissances en méthodes de réduction des modèles. Ce sujet de recherche vise \`a obtenir des modèles réduits de petite dimension, capable de retenir les dynamiques dominantes du problème. Cette étape permet de s'affranchir des simulations extrêmement couteuse pour des exigences d'optimisation, contrôle temps réel ou quantification d'incertitude. Dans ma recherche future je souhaite également explorer la préservation des invariants physiques dans le processus de réduction. Mon plan sur le long temps est de développer un projet computationelle pour fournir un cadre unifiant pour la modélisation, discrétisation et réduction des modèles physique utilisées en ingénierie. L'idee est de pouvoir améliorer lo status quo industriel pour ce qui concerne l'optimisation des structures mécaniques en phase de design. Je vous invite a consulter le fichier annexe pour plus des détails sur mon projet de recherche. 
	

	Je reste à votre disposition pour me contacter à votre convenance. Dans l'attente d'une réponse de votre part, veuillez accepter l'expression de mes salutations distinguées. Cordialement
	
	
	\begin{center}
		\large\textit{Andrea Brugnoli}
	\end{center}
\end{document}