\documentclass[aspectratio=169, french]{beamer}
\usepackage{fontspec}
\usepackage[french]{babel}
\usefonttheme[onlymath]{serif}
\usepackage{amsmath,amssymb,amsthm}
\usepackage{arydshln,mathtools}
\usepackage{bm}
\usepackage{color}
\definecolor{theme}{RGB}{0,73,114}
\usepackage{multicol}
%\usepackage[caption=false]{subfig}
\usepackage{subcaption}
\usepackage{multirow}

\usepackage{comment}

\usepackage{graphicx}
\usepackage{diffcoeff}
\usepackage{dsfont}
\usepackage{mathrsfs}
\usepackage[most]{tcolorbox}

\usepackage{xspace}
\usepackage{appendixnumberbeamer}


\usepackage{media9}
\usepackage[backend=bibtex,style=verbose,doi=false,isbn=false,url=false,eprint=false,autocite=footnote]{biblatex}

\addtobeamertemplate{footnote}{\vspace{-6pt}\advance\hsize-0.5cm}{\vspace{6pt}}
\makeatletter
% Alternative A: footnote rule
\renewcommand*{\footnoterule}{\kern -3pt \hrule \@width 2in \kern 8.6pt}
% Alternative B: no footnote rule
% \renewcommand*{\footnoterule}{\kern 6pt}
\makeatother

\graphicspath{{./images/}}

\bibliography{biblio_pres}

\renewcommand\bibfont{\scriptsize}

% Remove navigation bar
\setbeamertemplate{navigation symbols}{}

\setbeamertemplate{blocks}[rounded][shadow]

\setbeamercolor{block body alerted}{bg=alerted text.fg!10}
\setbeamercolor{block title alerted}{bg=alerted text.fg!20}
\setbeamercolor{block body}{bg=structure!10}
\setbeamercolor{block title}{bg=structure!20}
\setbeamercolor{block body example}{bg=green!10}
\setbeamercolor{block title example}{bg=green!20}

\graphicspath{{./images/}}

\newif\iftocsub
\tocsubtrue
\AtBeginSection[] {
	\begin{frame}[noframenumbering]{Aperçu}
		\tableofcontents[sectionstyle=show/shaded, subsectionstyle=show/show/hide]
	\end{frame}
	\tocsubfalse
}
\AtBeginSubsection[] {
	\iftocsub
	\begin{frame}[noframenumbering]{Aperçu}
		\tableofcontents[currentsubsection, sectionstyle=show/shaded, subsectionstyle=show/shaded/hide]
	\end{frame}
	\fi
	\tocsubtrue
}

\newcommand{\beginbackup}{
	\newcounter{framenumbervorappendix}
	\setcounter{framenumbervorappendix}{\value{framenumber}}
}
\newcommand{\backupend}{
	\addtocounter{framenumbervorappendix}{-\value{framenumber}}
	\addtocounter{framenumber}{\value{framenumbervorappendix}} 
}


\begin{document}
	
	
\begin{frame}[plain]
	
	\input{TitlePres}
	
\end{frame}

	
	
\begin{frame}{Aperçu}
		
		\tableofcontents
		
\end{frame}

	
\section{Nature du poste}

\begin{frame}{La simulation numérique au service de l'ingénierie}
	
	\begin{columns}
		\begin{column}{.45\textwidth}
		\begin{figure}
			\includegraphics[width=.9\textwidth]{cse.jpg}
			\caption{\href{https://www.uio.no/english/studies/programmes/computational-science-master/why-choose/}{https://www.uio.no}}
		\end{figure}	
		\end{column}
		\begin{column}{.5\textwidth}
		Il s'agit d'une discipline multidisciplinaire, à l'intersection de\footnotemark :
		\begin{itemize}
			\item mathématiques appliquées (analyse numérique, discrétisation des EDP, optimisation);
			\item informatique et développement logiciel;
			\item modélisation physique.
		\end{itemize}
		
		\end{column}
	\end{columns}	
	\footcitetext{ulrich2018cse}
\end{frame}

\section{Enjeux du poste}


\begin{frame}{Poste et motivation}
	\centering
	\begin{tcolorbox}[title = Missions, coltitle=white, halign=center, hbox]
		\begin{tabular}{c|c}
			\textbf{Enseignant}	& \textbf{Chercheur}  \\
			\hline
			Formations	& Agenda de recherche \\
			Projets	& Transversalité (ISAE)  \\
			Évolution programmes & Relations Internationales \\
			Stages & Partenariats Industrielles \\
			&  Thèses \\
			&  Publications \\
		\end{tabular}
	\end{tcolorbox}
\end{frame}


\section{Projet d'intégration}


\subsection{Missions enseignement}

\begin{frame}{Activités d'enseignement prévues pour le poste}
	L'offre de formation ISAE-Supaero prévoit plusieurs cours pour cette discipline.\\
	\vspace{.5cm}
	\textbf{Formation ingénieur (FISE)}:
	\begin{itemize}
		\item 1A : Méthodes numériques EDO et EDP 1D (différences finies, éléments finis).
		\item 2A : Équations aux dérivées partielles - Théorie et simulations numériques.
		\item 3A : Domaine SXS 
		\begin{itemize}
			\item[--] Méthodes numériques pour la mécanique et la dynamique des fluides (éléments finis classiques et mixtes, volumes finis);
			\item[--] Calcul Haute performance (HPC));
		\end{itemize}
	\end{itemize}
	\vspace{.5cm}
	\textbf{Formation Mastère international MAE} :  EDP et calcul scientifique (en anglais).\\
	\vspace{.5cm}
	\textbf{Formation par Apprentissage (FISA)} : donner aux professionnels et apprentis les instruments nécessaires pour comprendre les logiciels des simulations.
	
\end{frame}


\begin{frame}{Expérience formations }
	\textbf{Compétences} : 
	\begin{itemize}
		\item Méthodes numériques pour l'ingénieur (Formation ISAE, Stage CNES, Thèse, Post-Doc)
		\item Contrôle automatique et théorie des systèmes (Formation ISAE, Master recherche, Thèse, Conférences Internationales)
		\item Mécanique des solides et fluides (Licence, Thèse, Post-Doc) 
	\end{itemize}
\vspace{.5cm}
\textbf{Expériences Enseignement} : 
\begin{itemize}
	\item 12h Mathématiques appliques
	\item 48h vacataire TP et BE en automatique (formation SUPAERO) 
	\item 40h vacataire TP et laboratoire expérimental en automatique (MAE)
\end{itemize}
\end{frame}

\begin{frame}{Encadrement projets et thèses}
	\textbf{Expériences encadrement} : 
	\begin{itemize}
		\item Organisation et encadrement du PIE : "\textit{Simulation et contrôle des structures thermoélastiques pour applications spatiales}".
		\item Supervision du PFE de Vitor Borges Santos, en double diplôme ITA/UT Twente : "\textit{Model reduction for highly flexible thin structures}" (collaboration avec prof. Flavio Cardoso Ribeiro)
		\item Supervision de la thèse "\textit{On the modeling and mechanical design of flexures (compliant mechanisms)}" (collaboration avec prof. Marijn Nijenhuis).
	\end{itemize}

\end{frame}


\begin{frame}{Modules électifs à proposer à moyen terme}
	J'aimerais proposer des cours entre le calcul scientifique et les systèmes dynamiques :
	\begin{itemize}
		\item Modélisation et discrétisation symplectique des structures flexibles complexes (intégration et continuation du module PFEM4PHS).
		\item Éléments de Géométrie différentielle pour la modélisation physique : du continu au discret.
	\end{itemize}
\end{frame}


\subsection{Missions recherche}

\begin{frame}{Analyse numérique et discrétisation des EDP}
	\begin{block}{Discrétisation structurée de modèles physiques pour l'ingénierie}
		Développement des activités de recherche liées à la modélisation et discrétisation structurées des EDP (port-)Hamiltoniennes.
	\end{block}

\begin{columns}		
	\begin{column}{.3\textwidth}
		\includemedia[
		label=vidPlateRod,
		addresource=/home/andrea/Videos/CandidatureISAE/KirchhRod.mp4,
		activate=pageopen, 
		deactivate=onclick,
		width=5cm, height=5cm,
		flashvars={
			source=/home/andrea/Videos/CandidatureISAE/KirchhRod.mp4
			&%
			autoPlay=true&%
			loop=true%
		}
		]{}{VPlayer.swf}
		%\mediabutton[
		%mediacommand=vidPlateRod:playPause
		%]{\fbox{Play/Pause}}
		2D plaque interconnectée
	\end{column}
	\begin{column}{.3\textwidth}
		\includemedia[
		label=vidTG2D,
		addresource=/home/andrea/Videos/CandidatureISAE/vorticityTG2D.mp4,
		activate=pageopen, 
		deactivate=onclick,
		width=5cm, height=5cm,
		flashvars={
			source=/home/andrea/Videos/CandidatureISAE/vorticityTG2D.mp4&%
			autoPlay=true&%
			loop=true%
		}
		]{}{VPlayer.swf}
		%\mediabutton[
		%mediacommand=vidTG2D:playPause
		%]{\fbox{Play/Pause}}
		2D Hydrodynamique
	\end{column}
	\begin{column}{.3\textwidth}
		\includemedia[
		label=vidMaxwell3D,
		addresource=/home/andrea/Videos/CandidatureISAE/MaxwellE13D.mp4,
		activate=pageopen, 
		deactivate=onclick,
		width=5cm, height=5cm,
		flashvars={
			source=/home/andrea/Videos/CandidatureISAE/MaxwellE13D.mp4
			&%
			autoPlay=true&%
			loop=true%
		}
		]{}{VPlayer.swf}
		%\mediabutton[
		%mediacommand=vidMaxwell3D:playPause,
		%]{\fbox{Play/Pause}}
		3D Maxwell
	\end{column}
\end{columns}	
%\mediabutton[
%mediacommand=vidPlateRod:playPause,
%overface=\color{blue}{\fbox{\strut Play/Pause}},
%downface=\color{red}{\fbox{\strut Play/Pause}}
%]{\fbox{\strut Play/Pause}}

%\mediabutton[
%mediacommand=vidPlateRod:setSource [(KirchhRod.mp4)]
%]{\fbox{\strut KirchhRod.mp4}}

\end{frame}




\begin{frame}{Agenda de recherche}
	
	\begin{block}{Thématiques de recherche}
		\begin{itemize}
			\item Analyse numérique des schémas.
			\item Lien entre géométrie et discrétisation : \textit{les éléments finis en calcul extérieur}.
			\item Stratégies pour le gain en performance : maillage adaptatif, solveurs et preconditionneurs, parallélisation du code et calcul haute performance. 
			\item Lien vers les applications : mécanique, dynamique des fluides, électromagnétisme, \textit{intelligence artificielle} et compression de données.
		\end{itemize}

		
	\end{block}
	
	
	\begin{block}{Développement logiciel}
		\begin{itemize}
			\item Développement d'un code de calcul pour la multiphysique (SCRIMP).
			\item Création d'un gitlab commun de travail, pour les doctorants, les stagiaires mais également pour le BE et TP des cours liés à la modélisation.
		\end{itemize}	
		
	\end{block}
	
\end{frame}


\begin{frame}{Transversalité au sein du DISC}
	L'\textbf{intelligence artificielle} offre des outils essentiels pour la compression de données issus des schémas de discrétisation.
	
	\begin{figure}[t]
		\begin{subfigure}[t]{0.48\textwidth}
			\includegraphics[width=\columnwidth]{DGROM_T_param1.pdf} 
			\caption*{Modèle réduit avec réseaux des neurones.}
		\end{subfigure}\hfill
		\begin{subfigure}[t]{0.48\textwidth}
			\includegraphics[width=\columnwidth]{GROM_T_param1.pdf}%
			\caption*{Modèle réduit avec la méthode linéaire POD.}
			\label{fig:POD_ROM}
		\end{subfigure}
		\caption*{Erreur des modèles réduits sur le champ de température pour un problème de convection-diffusion-réaction (Reproduit avec permission de \cite{lee2020}).}%
		\label{fig:deepROM}%
	\end{figure}
\end{frame}


\begin{frame}{Transversalité au sein de l'ISAE}
\begin{itemize}
	\item DMSM : mécanique structurelle et optimisation topologique. 
	\item DCAS : pour la réduction des modèles et le contrôle automatique.
	\item DAEP : dynamique des fluides.
\end{itemize}
\begin{figure}[t]
	\begin{subfigure}{0.45\textwidth}
		\includegraphics[height=.45\textheight]{MDO_wing.pdf}%
		\caption*{\textbf{Optimisation multidisciplinaire d'une aile} (\cite{masColomer2021mdo}). }
	\end{subfigure}\hfill
	\begin{subfigure}{0.5\textwidth}
		\includegraphics[height=.45\textheight]{Codesign_satellite.pdf} 
		\caption*{\textbf{Co-design contrôle/structure pour un satellite} (\cite{finozzi2022sub}).}
	\end{subfigure}
\end{figure}

\end{frame}

\begin{frame}{Relations Internationales}
	\begin{figure}[t]
		\begin{subfigure}{0.5\textwidth}
			\includegraphics[width=.95\textwidth]{mappe_reseau_europe.pdf}%
		\end{subfigure}\hfill
		\begin{subfigure}{0.4\textwidth}
			\includegraphics[width=.9\textwidth]{mappe_reseau_world.pdf} 
		\end{subfigure}
	\end{figure}
\end{frame}


\begin{frame}{Création de partenariats}	

	\begin{columns}
		\begin{column}{.5\textwidth}
			Partenariats industriels envisageables : 
			\begin{itemize}
				\item CEA : simulation multiphysique.
				\item CERFACS : CFD et assimilation des données. 
				\item AIRBUS : aéroélasticité et mécanique.
				\item THALES : électromagnétisme, thermoélasticité.
			\end{itemize}
		\end{column}
		\begin{column}{.43\textwidth}
			\begin{figure}
				\includegraphics[width=1\textwidth]{image_CERFACS.png}
				\caption{\href{https://cerfacs.fr/logiciels-de-simulation-pour-la-mecanique-des-fluides/}{https://cerfacs.fr}}
			\end{figure}	
		\end{column}
	\end{columns}
	
\end{frame}




\begin{frame}{}

\end{frame}

\appendix

\section{Formation et Expériences}

\begin{frame}{Parcours académique}
\begin{itemize}
	\item (2011-2014) \textbf{Licence en génie mécanique}, Politecnico DI MILANO
	\item (2014-2017) \textbf{Master en génie spatial}, Politecnico DI MILANO
	\item (2014-2017) \textbf{Diplôme d’Ingénieur Supaéro}, Programme de
	Double Diplôme, ISAE SUPAERO/Politecnico DI MILANO
	\item (2016-2017) \textbf{Master recherche en automatique et traitement d’images
	} ISAE-SUPAERO/SUPELEC (Université Paris Saclay)
	\item (2017-2020) \textbf{Thèse} : "\textit{Une formulation port-Hamiltonienne des structures flexibles. Modélisation et discrétisation symplectique par éléments finis}", ISAE-SUPAERO
\end{itemize}
\end{frame}

\begin{frame}{Expériences}
\begin{itemize}
\item (2014 : 4 mois) \textbf{Projet Fin Études (Licence)} : \textit{Dynamique d’un manipulateur pour machines de forgeage}
\item (2017 : 6 mois) \textbf{Projet Fin Études (Master)} : \textit{Analyse de la dynamique des débris spatiaux soumis à la pression de radiation solaire}, CNES, Toulouse (France)
\item (2019 : 4 mois) \textbf{Chercheur invité} : \textit{Collaboration avec prof. Flavio Cardoso-Riberio}, Instituto Tecnológico de Aeronáutica (Brésil).
\item (2020-2022) : \textbf{Chercheur post-doctoral} : \textit{Méthodes numériques pour problèmes couplés fluide-structure}, ERC avancé, PI Stefano Stramigioli, Enschede (Pays-Bas)
\item (2021 : 1 semaine) \textbf{École d'été} : \textit{Intelligence artificielle et apprentissage par renforcement}, Institut Cifar, Toronto (Canada)
\item (2022 : 1 semaine) \textbf{Invitation au CIRM} (centre international de rencontre mathématiques) : \textit{Modélisation structurée, intégration géométrique et commande de systèmes multiphysiques contraints}
\end{itemize}
  
\end{frame}

\section{Production scientifiques}

\begin{frame}{Articles de revue (mathématiques appliquées)}
\begin{enumerate}
	\item \cite{brugnoli2022df}
	\item \cite{brugnoli2021num}
	\item \cite{brugnoli2019ammmin}
	\item \cite{brugnoli2019ammkir}
\end{enumerate}


\end{frame}


\begin{frame}{Congres internationales (mathématiques appliquées)}
\footnotesize	
\begin{enumerate}
	\item \cite{brugnoli2021vk}
	\item \cite{rashad2021ext}
	\item \cite{cherifi2021data}
	\item \cite{brugnoli2021siamcse}
	\item \cite{brugnoli2020mtns}
	\item \cite{brugnoli2019cpde}
\end{enumerate}

\end{frame}

\begin{frame}{Articles de revue et Congres internationales (mécanique et automatique)}
	\textbf{Articles de revue} : 
	\begin{enumerate}
		\item \cite{brugnoli2021ther}
		\item \cite{brugnoli2020msd}
	\end{enumerate}
	\textbf{Congres internationales} :
	\begin{enumerate}
		\item \cite{brugnoli2019cdc}
		\item \cite{cardoso2019cdc}
	\end{enumerate}
\end{frame}

\section{Résume activités enseignement et scientifiques}

\begin{frame}{Résumé activité d'enseignement}
	\begin{table}
		\centering
		\begin{tabular}{p{\dimexpr.15\linewidth-2\tabcolsep}p{\dimexpr.15\linewidth-2\tabcolsep}p{\dimexpr.15\linewidth-2\tabcolsep}p{\dimexpr.4\linewidth-2\tabcolsep}p{\dimexpr.15\linewidth-2\tabcolsep}}
			\hline
			Année & Niveau & Nature  & Discipline & Durée  \\
			\hline
			\multirow{2}{*}{2019-2020} & L1 & TD &  Résolution numérique des EDP & 6h \\
			& L1 & TD &  Optimisation & 6h \\
			\hline
			\multirow{3}{*}{2018-2019} & L2  & TP-TD  & Automatique & 20h \\
			& L2  & TD     & Contrôle des structures flexibles & 8h \\
			& L2  & TP-TD  & Automatic control & 20h \\
			\hline
			\multirow{2}{*}{2017-2018} & L2  & TP-TD  & Automatique & 20h \\
			& L2  & TP-TD  & Automatic control & 20h \\				   
			\hline
		\end{tabular}


	\end{table}
\end{frame}

\begin{frame}{Résumé des activité scientifiques.}
	\small
	\begin{table}
		\centering
		\begin{tabular}{p{\dimexpr.15\linewidth-2\tabcolsep}p{\dimexpr.2\linewidth-2\tabcolsep}p{\dimexpr.65\linewidth-2\tabcolsep}}
			\hline
			Année & Lieu & Description  \\
			\hline
			2022 (en cours) & University of Twente (Enschede) & Supervision du projet fin étude de Vitor Borges Santos dans le cadre du double diplôme ITA/UT Twente (collaboration avec prof. Flavio Cardoso Ribeiro). \\
			2022 (en cours) & University of Twente (Enschede) & Supervision de la thèse "On the modeling and mechanical design of flexures (compliant mechanisms)" (collaboration avec prof. Marijn Nijenhuis). \\
			\hline
			2021  & Technical University of Berlin (Berlin) & Organisation de la session invitée: "Theoretical and numerical advancements in Hamiltonian formulations of continuum mechanics" pour la conférence "Lagrangian and Hamiltonian method in non linear control 2021". \\
			\hline
			2020 & --- & Critiques (Peer reviews) des journaux \textit{Journal of Elasticity} et \textit{Mathematical and Computer Modelling of Dynamical Systems} et conférences LHMNC, MATHMOD, MTNS. \\
			\hline
			2019-2020 & ISAE-SUPAERO (Toulouse) & Organisation et encadrement du Projet Ingénierie et Entreprise intitulé "Simulation et contrôle des structures thermoélastiques pour
			applications spatiales". \\
			\hline
		\end{tabular}

		\label{tab:activites}
	\end{table}
\end{frame}

\section{Détails Activités recherches récentes}

\begin{frame}{Activités recherche post Doc}
Discrétisation géométrique des systèmes port-Hamiltoniens. \\
\textbf{Équation d'onde }
	
\only<1>{	
\begin{figure}
\centering
\subfloat[][Erreur $L^2$ pour $p^3_h$]{%
	\label{fig:err_p3}%
	\includegraphics[width=0.48\columnwidth]{p_3_3D_DN.pdf}}%
\hspace{8pt}%
\subfloat[][Erreur $L^2$ pour $p^0_h$]{%
	\label{fig:err_p0}%
	\includegraphics[width=0.48\columnwidth]{p_0_3D_DN.pdf}}%
\caption*{Pente de convergence pour la rapresentation duale de la pression}%
\end{figure}
}
\only<2>{	
\begin{figure}
	\centering
	\subfloat[][Erreur $L^2$ pour $u^1_h$]{%
		\label{fig:err_u1}%
		\includegraphics[width=0.48\columnwidth]{q_1_3D_DN.pdf}}
	\hspace{8pt}%
	\subfloat[][Erreur $L^2$ pour $u^2_h$]{%
		\label{fig:err_u2}%
		\includegraphics[width=0.48\columnwidth]{q_2_3D_DN.pdf}}
	\caption*{Pente de convergence pour la rapresentation duale de la vitesse}%
\end{figure}
}
\only<3>{
\begin{figure}
	\centering
	\subfloat[][Norme $L^2$ de la difference $p^3_h - p^0_h$]{%
		\label{fig:diff_p30}%
		\includegraphics[width=0.48\columnwidth]{p_30_3D_DN.pdf}}%
	\hspace{8pt}%
	\subfloat[][Norme $L^2$ de la difference $q^1_h - q^2_h$]{%
		\label{fig:diff_q12}%
		\includegraphics[width=0.48\columnwidth]{q_12_3D_DN.pdf}}%
	\caption*{Différence $L^2$ pour la représentation duale des variables.}%
\end{figure}
}

\only<4>{

\begin{columns}		
	\begin{column}{.45\textwidth}
		\includemedia[
			addresource=/home/andrea/Videos/CandidatureISAE/Waveu23D.mp4,
		activate=pageopen, 
		deactivate=onclick,
		width=7cm, height=6cm,
		flashvars={
			source=/home/andrea/Videos/CandidatureISAE/Waveu23D.mp4
			&%
			autoPlay=true&%
			loop=true%
		}
		]{}{VPlayer.swf}
		Vitesse 2 forme
	\end{column}
	\begin{column}{.45\textwidth}
		\includemedia[
		addresource=/home/andrea/Videos/CandidatureISAE/Waveu13D.mp4,
		activate=pageopen, 
		deactivate=onclick,
		width=7cm, height=6cm,
		flashvars={
			source=/home/andrea/Videos/CandidatureISAE/Waveu13D.mp4&%
			autoPlay=true&%
			loop=true%
		}
		]{}{VPlayer.swf}
		Vitesse 1 forme
	\end{column}
	
\end{columns}	
}

\end{frame}

\begin{frame}{Activités recherche post Doc}
	Discrétisation géométrique des systèmes port-Hamiltoniens. \\
	\textbf{Équation de Maxwell}
	
	\only<1>{	
		\begin{figure}
			\centering
			\subfloat[][Erreur $L^2$ pour $E^2_h$]{%
				\label{fig:err_E2}%
				\includegraphics[width=0.48\columnwidth]{E_2_3D_EH.pdf}}%
			\hspace{8pt}%
			\subfloat[][Erreur $L^2$ pour $E^1_h$]{%
				\label{fig:err_E1}%
				\includegraphics[width=0.48\columnwidth]{E_1_3D_EH.pdf}}%
			\caption{Pente de convergence pour la représentation duale du champ Électrique}%
		\end{figure}
	}
	\only<2>{	
		\begin{figure}
			\subfloat[][Erreur $L^2$ pour $H^2_h$]{%
				\label{fig:err_H2}%
				\includegraphics[width=0.48\columnwidth]{H_2_3D_EH.pdf}}
			\hspace{8pt}%
			\subfloat[][Erreur $L^2$ error for $H^1_h$]{%
				\label{fig:err_H1}%
				\includegraphics[width=0.48\columnwidth]{H_1_3D_EH.pdf}}
			\caption{Pente de convergence pour la représentation duale du champ Magnétisant}%
		\end{figure}
	}
	\only<3>{
		\begin{figure}
			\centering
			\subfloat[][$L^2$ norm of the difference $E^1_h - E^2_h$]{%
				\label{fig:diff_E21}%
				\includegraphics[width=0.48\columnwidth]{E_21_3D_EH.pdf}}%
			\hspace{8pt}%
			\subfloat[][$L^2$ norm of the difference $H^1_h - H^2_h$]{%
				\label{fig:diff_H21}%
				\includegraphics[width=0.48\columnwidth]{H_21_3D_EH.pdf}}%
			\caption*{Différence $L^2$ pour la représentation duale des variables.}%
		\end{figure}
	}

\only<4>{
	
	\begin{columns}		
		\begin{column}{.45\textwidth}
			\includemedia[
			addresource=/home/andrea/Videos/CandidatureISAE/MaxwellE23D.mp4,
			activate=pageopen, 
			deactivate=onclick,
			width=7cm, height=6cm,
			flashvars={
				source=/home/andrea/Videos/CandidatureISAE/MaxwellE23D.mp4
				&%
				autoPlay=true&%
				loop=true%
			}
			]{}{VPlayer.swf}
			Champ électrique 2 forme
		\end{column}
		\begin{column}{.45\textwidth}
			\includemedia[
			addresource=/home/andrea/Videos/CandidatureISAE/MaxwellE13D.mp4,
			activate=pageopen, 
			deactivate=onclick,
			width=7cm, height=6cm,
			flashvars={
				source=/home/andrea/Videos/CandidatureISAE/MaxwellE13D.mp4&%
				autoPlay=true&%
				loop=true%
			}
			]{}{VPlayer.swf}
			Champ électrique 1 forme
		\end{column}
		
	\end{columns}	
}
	
\end{frame}

\section{Détails collaborations industrielles}

\begin{frame}{Collaborations avec les industries}

\end{frame}



\end{document}