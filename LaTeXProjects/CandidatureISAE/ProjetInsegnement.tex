\documentclass[12pt, french]{article}
\usepackage{fontspec}
\setmainfont{Arial}
\usepackage{wrapfig}
\usepackage[french]{babel}
\usepackage{multirow}

\usepackage{setspace}
\singlespacing

% normal box
\newcommand{\sqboxs}{1.2ex}% the square size
\newcommand{\sqboxf}{0.6pt}% the border in \sqboxEmpty
\newcommand{\sqbox}[1]{\textcolor{#1}{\rule{\sqboxs}{\sqboxs}}}

%\usepackage[none]{hyphenat}
%\usepackage{hyphenat}

\usepackage{graphicx}
\usepackage{caption}
%\usepackage[T1]{fontenc}
%\usepackage[utf8]{inputenc}
\usepackage{lmodern}
\usepackage{geometry}
\geometry{
	a4paper,
	left=20mm,
	right=20mm,
	top=25mm,
	bottom=25mm,
}

\usepackage{pgfgantt}
\usepackage{eurosym}

\usepackage{pdfpages}


\usepackage{subcaption}

\usepackage[unicode=true,pdfusetitle,bookmarks=true,bookmarksnumbered=false,bookmarksopen=false,
breaklinks=false,pdfborder={0 0 0},backref=false,colorlinks=true,urlcolor=magenta]{hyperref}

\usepackage{multibib}
\newcites{article,conf,confnoproc}{{Articles dans des revues internationales à comité de lecture
	},{Communications dans des congrès internationaux à comité de lecture et actes
		publiés},{Communications dans des congrès internationaux sans comité de lecture}}


\newcommand{\review}[1]{\textcolor{blue}{#1}}

\graphicspath{{images/}}

\author{Andrea Brugnoli \\ 
	\hspace{2.8pt} Docteur ISAE-SUPAERO 2020\\
	Ingénieur ISAE-SUPAERO 2017}
\title{Projet d'enseignement}

\date{}

\begin{document}
	
	\maketitle
	
	
	\thispagestyle{empty}
	
	\tableofcontents


	\section{Activités d'enseignement}
	
	Pendant ma thèse, j'ai effectué mes activités d'enseignement à l'Institut Supérieur de l'Aéronautique et de l'Espace, soit pour la formation ingénieur, soit pour les masters internationaux. J'ai encadré des cours en Automatique et en mathématiques appliquées. Pour ce qui concerne l'Automatique, j'ai donné des cours pour le tronc commun en 2A  (représentation d'état, contrôle en fréquence, lieu des racines, estimation), pour le module électif \textit{Systèmes dynamiques: comment ça marche} en 2A (identification, retour de sortie colocalisé, supervision du projet étudiant), et dans le master international Aerospace systems and control (systèmes discrets, transformation en Z, projet étudiant identification et contrôle des structures flexibles). Pour la partie mathématiques appliquées, j'ai participé aux activités d'enseignement pour le tronc commun 1A (résolution numériques des EDP, différences finies, éléments finis, méthodes numériques pour l'optimisation convexe). Le Tableau \ref{tab:einsegnemnt} résume les différentes activités en termes de volume horaire. 
	
	\begin{table}[h]
		\centering
		\begin{tabular}{p{\dimexpr.15\linewidth-2\tabcolsep}p{\dimexpr.15\linewidth-2\tabcolsep}p{\dimexpr.15\linewidth-2\tabcolsep}p{\dimexpr.4\linewidth-2\tabcolsep}p{\dimexpr.15\linewidth-2\tabcolsep}}
			\hline
			Année & Niveau & Nature  & Discipline & Durée  \\
			\hline
			\multirow{2}{*}{2019-2020} & L1 & TD &  Résolution numérique des EDP & 6h \\
			& L1 & TD &  Optimisation & 6h \\
			\hline
			\multirow{3}{*}{2018-2019} & L2  & TP-TD  & Automatique & 20h \\
			& L2  & TD     & Contrôle des structures flexibles & 8h \\
			& L2  & TP-TD  & Automatic control & 20h \\
			\hline
			\multirow{2}{*}{2017-2018} & L2  & TP-TD  & Automatique & 20h \\
			& L2  & TP-TD  & Automatic control & 20h \\				   
			\hline
		\end{tabular}
	\caption{Résume des activité des enseignement.}
	\label{tab:einsegnemnt}
	\end{table}
	



\section{Activités scientifiques}

Pendant mon parcours professionnel, j'ai été impliqué dans différentes activité scientifiques. Pendant ma thèse, j'ai participé à la supervision du Projet Ingénierie et Entreprise pour le domaine SXS (cf. \ref{tab:activites}). Ce projet à été centré sur le développement d'algorithmes numériques pour la thermoélasticité linéaire. J'ai contribué en tant que examinateur pour le processus \textit{peer review} du \textit{Journal of Elasticity}. J'ai co-organisé une session invité pour la conférence \textit{Lagrangian and Hamiltonian method in non linear control 2021}. Actuellement, je co-encadre avec prof. Marijn Nijenhuis une doctorante. La thèse est centrée sur le développement des modèles Hamiltoniens interconnectés pour des structures très flexible (en anglais \textit{flexures}). En particulier, l'idée est de concevoir un cadre computationnelle structuré pour la digitalisation des composantes mécaniques flexibles pour des finalités de design et optimisation topologiques. Je collabore également avec l'Instituto Tecnológico de Aeronáutica (Bresil), pour l'encadrement du stage du fin étude d'un étudiant brésilien en double diplome avec l'Université de Twente. Le sujet de ce stage est la réduction des modèles pour le systèmes multi-corps flexibles, à l'aide des outils de l'intelligence artificielles (décomposition en modes dynamiques, identification parcimonieuse de la dynamique non linéaire etc.).


\begin{table}[h]
	\centering
	\begin{tabular}{p{\dimexpr.15\linewidth-2\tabcolsep}p{\dimexpr.2\linewidth-2\tabcolsep}p{\dimexpr.65\linewidth-2\tabcolsep}}
		\hline
		Année & Lieu & Description  \\
		\hline
		2022 (en cours) & University of Twente (Enschede) & Supervision du projet fin étude de Vitor Borges Santos dans le cadre du double diplôme Instituto Tecnológico de Aeronáutica/University of Twente (collaboration avec prof. Flavio Cardoso Ribeiro). \\
		2022 (en cours) & University of Twente (Enschede) & Supervision de la thèse "On the modeling and mechanical design of flexures (compliant mechanisms)" entre le département de Robotique et le département d'ingénierie de précision à l'Université de Twente (collaboration avec prof. Marijn Nijenhuis). \\
		\hline
		2021  & Technical University of Berlin (Berlin) & Organisation de la session invitée: "Theoretical and numerical advancements in Hamiltonian formulations of continuum mechanics" pour la conférence "Lagrangian and Hamiltonian method in non linear control 2021". \\
		\hline
		2020 & --- & Critiques (Peer reviews) du \textit{Journal of Elasticity}. \\
		\hline
		2019-2020 & ISAE-SUPAERO (Toulouse) & Organisation et encadrement du Projet Ingénierie et Entreprise intitulé "Simulation et contrôle des structures thermoélastiques pour
		applications spatiales". \\
		\hline
	\end{tabular}
	\caption{Résumé des activité scientifiques.}
	\label{tab:activites}
\end{table}

\section{Lien entre mes compétences et la formation ISAE-SUPAERO}
	
Mon expertise se situe entre le calcul scientifique, l'automatique et la mécanique. Plus précisément, je développe des algorithmes numériques (éléments finis mixtes ou éléments finis en calcul extérieur) pour la dynamique Hamiltonien à ports des structures flexibles et des fluides. Ces méthodes numériques ont une importance cruciale en Automatique, car les systèmes port-Hamiltoniens représentent un paradigme de modalisation inspiré de la théorie des systèmes. Pendant mon expérience de thèse et post-Doc, j'ai pu participer à plusieurs conférence d'haut niveau pour l'Automatique et ses applications. De ce fait, je connais les développement récents dans ce domaine. \\

Du fait des compétences obtenues pendant ma thèse et mon port-Doc, je pourrai donc participer aux formations suivantes pour la formation ingénieur ISAE. \\

\textbf{Mathématiques Appliquées} \\
Première année: 
\begin{itemize}
	\item Tronc commun : Optimisation, Analyse Numérique.
	\item Module électif : Mathématiques et espace.
	\item  Module électif : Optimisation numérique avancée.
\end{itemize}
Deuxième année
\begin{itemize}
	\item Tronc commun : Équations aux dérivées partielles - Théorie et simulations numériques.
	\item Module électif : Optimisation numérique avancée
	\item Module électif : PFEM4PHS
\end{itemize}
Troisième année
\begin{itemize}
	\item Domaine SXS: Méthodes numériques de l'ingénieur - EDP avancées
\end{itemize}

\textbf{Automatique} \\ 
Deuxième année
\begin{itemize}
	\item Tronc commun : Signaux et Systèmes
	\item Module électif : Automatique avancée
\end{itemize}
Troisième année
\begin{itemize}
	\item Filière Signaux et Systèmes :  Représentation et d'analyse des systèmes dynamiques
\end{itemize}

\textbf{Mécanique} \\
Troisième année
\begin{itemize}
	\item Filière Structure et Matériaux: Calcul de structures par la méthode à éléments finis.
	\item Filière Structure et Matériaux: Dynamique des structures.
\end{itemize}


\section{Projet d'enseignement}
J'aimerais introduire un cours sur le systèmes dynamiques non-linéaires. En particulier, le aspect concernant l'interaction entre géométrie et dynamique. A ce propos il est fondamentale d'introduire le langage de la géométrie différentielle \cite{wiggins2003introduction,haddad2011nonlinear}. Des applications avancées, en lien avec la simulation numérique des EDP, pourront également être proposées.


	\bibliographystyle{unsrt}
	\bibliography{biblio_einsegnement}
	
	
\end{document}
