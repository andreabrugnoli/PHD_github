\documentclass[aspectratio=169]{ISAE-Beamer}
\usefonttheme[onlymath]{serif}
\usepackage{amsmath,amssymb,amsthm}
\usepackage{bm}

\usepackage{graphicx}
\usepackage{diffcoeff}
\usepackage{dsfont}
\usepackage{mathrsfs}
\usepackage{tcolorbox}

%\usepackage{multimedia}
\usepackage{media9}
\usepackage[backend=bibtex]{biblatex}

\graphicspath{{../images/}}

%\bibliography{../mybibliography}

\DeclareMathOperator{\Tr}{Tr}
\DeclareMathOperator*{\grad}{grad}
\DeclareMathOperator*{\Grad}{Grad}
\DeclareMathOperator*{\Div}{Div}
\renewcommand{\div}{\operatorname{div}}
\DeclareMathOperator*{\Hess}{Hess}

\DeclareMathOperator*{\argmax}{arg\,max}
\DeclareMathOperator*{\argmin}{arg\,min}

\def\onedot{$\mathsurround0pt\ldotp$}
\def\cddot{% two dots stacked vertically
	\mathbin{\vcenter{\baselineskip.67ex
			\hbox{\onedot}\hbox{\onedot}}%
}}

\renewcommand\bibfont{\scriptsize}


\makeatletter \renewcommand\d[1]{\ensuremath{%
		\;\mathrm{d}#1\@ifnextchar\d{\!}{}}}
\makeatother

\title[58th CDC conference]{A port-Hamiltonian formulation of flexible structures \\
Modelling and structure preserving finite element discretization}

%\institute[ISAE]
%{\inst{1}ISAE-SUPAERO, Toulouse}

\author[Andrea Brugnoli]{Andrea Brugnoli\\
	{\and} \\
	{\textit{Supervisors}} \\
	{Daniel Alazard} \\ {Valérie Pommier-Budinger}}

\date[Toulouse, 9/11/20]{November, the 9th, 2020}

%\thanks{}

\begin{document}

\maketitle

\begin{frame}{Outline}

\tableofcontents

\end{frame}

\section{Introduction: discretization of pH Systems}

\begin{frame}{Infinite dimensional pH systems}
\begin{block}{\only<1>{General}\only<2>{Linear} infinite dimensional pH system}
	\begin{equation*}
	\begin{cases}
	\displaystyle \diffp{x}{t}(z, t) &= \mathcal{J} \only<1>{\displaystyle \diffd{H}{x}} \only<2>{\mathcal{Q} x} + B \textcolor{red}{u(z, t)}, \vspace{3pt} \\
	\displaystyle \textcolor{red}{y(z, t)} &= B^* \only<1>{\displaystyle \diffd{H}{x}} \only<2>{\mathcal{Q} x}.
	\end{cases}
	\end{equation*}
	With boundary conditions
	\[\textcolor{blue}{u_\partial} = \mathcal{B} \only<1>{\displaystyle \diffd{H}{x}} \only<2>{\mathcal{Q} x}, \quad \textcolor{blue}{y_\partial} = \mathcal{C} \only<1>{\displaystyle \diffd{H}{x}} \only<2>{\mathcal{Q} x} \]
	Energy rate: $\dot{H} = \displaystyle \int_{\partial \Omega} \textcolor{blue}{u_\partial y_\partial} \d{s} +  \int_{\Omega} \textcolor{red}{u(z, t) y(z, t)} \d{\Omega}$
	\begin{itemize}
		\item $x$ energy variables, $e = \delta_x H = \only<2>{\mathcal{Q} x}$: co-energy variables \only<2>{($\mathcal{Q}$ symmetric positive operator)};
		\item $\mathcal{J}$: skew-symmetric differential operator;
		\item $\mathcal{B}, \mathcal{C}$: boundary operator;
		\item $u, y, B$: distributed input, output and control operator.
	\end{itemize}
\end{block}

\end{frame}


\begin{frame}{Structure preserving discretization}
The main properties of the dpH (conservative, passive system) are preserved at a discrete level.
\begin{columns}[T]
	\setlength{\abovedisplayskip}{1pt}
	\setlength{\belowdisplayskip}{1pt}
	\begin{column}{.4\textwidth}
		\begin{block}{Infinite dimensional pHs \only<2>{(linear case)}}
			PDE:
			\begin{align*}
			\partial_t{x}(z, t) &= \mathcal{J} \only<1>{\displaystyle \diffd{H}{x}} \only<2>{\mathcal{Q} x} + B \textcolor{red}{u(z, t)}, \\
			\textcolor{red}{y(z, t)} &= B^* \only<1>{\displaystyle \diffd{H}{x}} \only<2>{\mathcal{Q} x}.
			\end{align*}
			Boundary conditions: 
			\[\textcolor{blue}{u_\partial} = \mathcal{B} \only<1>{\displaystyle \diffd{H}{x}} \only<2>{\mathcal{Q} x}, \quad \textcolor{blue}{y_\partial} = \mathcal{C} \only<1>{\displaystyle \diffd{H}{x}} \only<2>{\mathcal{Q} x} \]
			Power balance (Stokes Theorem): 
			\[ \dot{H} = \displaystyle \int_{\partial \Omega} \textcolor{blue}{u_\partial y_\partial} \d{s} +  \int_{\Omega} \textcolor{red}{u(z, t) y(z, t)} \d{\Omega}
			\]
		\end{block}
	\end{column}
	\begin{column}{.4\textwidth}
		\begin{block}{Finite dimensional pHs \only<2>{(linear case)}}
			ODE:
			\begin{align*}
			\dot{x} &= J \only<1>{\displaystyle \partial_x {H}} \only<2>{Q x} + B_d \textcolor{red}{u_d} + B_\partial \textcolor{blue}{u_\partial}, \\
			\textcolor{red}{y_d} &= B_d^T \only<1>{\displaystyle \partial_x {H}} \only<2>{Q x}, \\
			\textcolor{blue}{y_\partial} &= B_\partial^T \only<1>{\displaystyle \partial_x {H}} \only<2>{Q x}
			\end{align*}
			Power balance: 
			\[ \dot{H} = \textcolor{blue}{u_\partial^T y_\partial} +  \textcolor{red}{u_d^T y_d}
			\]
		\end{block}
	\end{column}
\end{columns}

\end{frame}

\begin{frame}{}
\begin{exampleblock}{Available methods}
	\begin{itemize}
		\item Spectral methods (Moulla 2012):
		\begin{itemize}
			\item[\textcolor{green}{\checkmark}] Rapid spectral convergence;
			\item[\textcolor{red}{$\times$}] Only for 1D problem;
		\end{itemize}
		\item Finite differences (Trenchant 2018);
		\begin{itemize}
			\item[\textcolor{green}{\checkmark}] Valid up to 2D geometries;
			\item[\textcolor{red}{$\times$}] Requires \textit{ad hoc} implementation (staggered grids);
		\end{itemize}
		\item Finite elements based
		\begin{itemize}
			\item Golo 2004, Kotyczka 2018: the implementation requires exterior calculus knowledge and depends on the some parameters that ensure the preservation of power flow;		
			\item \textcolor{blue}{Cardoso-Ribeiro 2018}:
			\begin{itemize}
				\item[\textcolor{green}{\checkmark}] Natural extension of the mixed finite element method to pH systems;
				\item[\textcolor{green}{\checkmark}] Implementable using well-established libraries (Fenics, Firedrake);
			\end{itemize}
		\end{itemize}
	\end{itemize}
\end{exampleblock}
\end{frame}


\section{Structure preserving discretization}

\subsection{The partitioned finite element method}



\begin{frame}{The partitioned finite element method (PFEM)}
General form of a linear pH system in co-energy variables
\begin{equation*}
\mathcal{M} \diffp{e}{t} = \mathcal{J} e, \qquad \mathcal{M} = \mathcal{Q}^{-1}
\end{equation*}

\begin{block}{General procedure for PFEM}
	\setlength{\abovedisplayskip}{1pt}
	\setlength{\belowdisplayskip}{1pt}
	\begin{enumerate}
		\item Put the system into weak form:
		\begin{equation*}
		\left(v, \mathcal{M} \diffp{e}{t} \right)_{\Omega} = \left(v, \mathcal{J} e \right)_{\Omega}.
		\end{equation*}
		\item Apply integration by part on a partition of $\mathcal{J}$:
		\begin{equation*}
		\left(v, \mathcal{J} e \right)_{\Omega} \overbrace{=}^{i.b.p.} j(v, e)_{\Omega} + b(v, u_\partial)_{\partial \Omega},
		\end{equation*}
		so that $j(v, e)_{\Omega}$ is a skew-symmetric bilinear form.
		\item Discretization by Galerkin method (same basis function for test and co-energy variables)
	\end{enumerate}
\end{block}
\end{frame}



\begin{frame}{Results}
\begin{center}
	\onslide*<1>{
		\setlength{\abovedisplayskip}{0pt}
		\setlength{\belowdisplayskip}{0pt}
		Distributed load ($t_{\text{end}} = 10 \, [\mathrm{ms}]$)
		\begin{equation*}
		p = \begin{cases}
		10^5 \left[ y + 10 \left( y - L_y/2 \right)^2 \right] [Pa], \quad &\forall \, t < 2 \, [\mathrm{ms}], \\
		0, \quad &\forall \, t \ge 2 \, [\mathrm{ms}].
		\end{cases}
		\end{equation*}
		\begin{columns}
			\begin{column}{.45\textwidth}
				\includemedia[
				label=vidNoRod,
				addresource=/home/a.brugnoli/Videos_defense/Kirchh_NoRod.mp4,
				activate=pageopen,
				width=6cm, height=5cm,
				flashvars={
					source=/home/a.brugnoli/Videos_defense/Kirchh_NoRod.mp4
					&loop=true
				}
				]{}{VPlayer.swf}
			\end{column}
			\begin{column}{.45\textwidth}
				\includemedia[
				label=vidRod,
				addresource=/home/a.brugnoli/Videos_defense/Kirchh_Rod.mp4,
				activate=pageopen,
				width=6cm, height=5cm,
				flashvars={
					source=/home/a.brugnoli/Videos_defense/Kirchh_Rod.mp4
					&loop=true
				}
				]{}{VPlayer.swf}
			\end{column}
		\end{columns}
	
	\mediabutton[
	mediacommand=vidNoRod:playPause,
	mediacommand=vidRod:playPause
	]{\fbox{Play/Pause}}
				
		%\movie[width=0.8\textwidth, height=0.6\textheight]{Plate and rod}{../Videos/Comparison_RodNoRod.mp4}	
	}
	
	\end{center}
\end{frame}

\section{Stabilization by boundary injection}

\begin{frame}{Boundary stabilization of the Kirchhoff plate}
\only<1>{Consider the problem
	\begin{equation*}\small
	\begin{bmatrix}
	\rho h & 0 \\ 0 & \mathbb{D}^{-1} \\
	\end{bmatrix}
	\diffp{}{t}
	\begin{bmatrix}
	\partial_t w \\ \bm{M} \\
	\end{bmatrix} = 
	\begin{bmatrix}
	0 & -\div\Div \\ \nabla^2 & 0 \\
	\end{bmatrix}
	\begin{bmatrix}
	\partial_t w \\ \bm{M} \\
	\end{bmatrix} \quad (x, y) \in \Omega = [0, 1]\times[0,1]
	\end{equation*}
	subjected to the following boundary conditions
	\begin{align*}
	\begin{aligned}
	\partial_t w|\textcolor{blue}{\Gamma_D} &= 0, \\
	\partial_x \partial_t w|\textcolor{blue}{\Gamma_D} &= 0, \\
	\end{aligned} \qquad \textcolor{blue}{\Gamma_D} &= \left\{x = 0 \right\}\\
	\begin{aligned}
	{M}_{nn}|\textcolor{red}{\Gamma_N} &= u_M, \; \\
	\widetilde{q}|\textcolor{red}{\Gamma_N} &= u_F,\\
	\end{aligned} \qquad \textcolor{red}{\Gamma_N} &= \left\{y = 0 \cup x=1 \cup y=1 \right\}
	\end{align*}
	
	with initial conditions (compatible with the constraints):
	\[
	\partial_t w(x,y,0) = x^2; \qquad \bm{M}(x,y,0) ={0}.
	\]
}
\only<2>{ Obtain a finite-dimensional uncontrolled system
	\begin{equation*}
	\begin{aligned}
	\begin{bmatrix}
	{M} & {0} \\
	{0} & {0} \\
	\end{bmatrix}\frac{d}{d t}
	\begin{pmatrix}
	\bm{e}\\
	\bm{\lambda} \\
	\end{pmatrix}
	&= \begin{bmatrix}
	{J} & {G} \\
	-{G}^T & {0} \\
	\end{bmatrix}
	\begin{pmatrix}
	\bm{e} \\
	\bm{\lambda} \\
	\end{pmatrix} + \begin{bmatrix}
	{B} \\
	0 \\
	\end{bmatrix} \bm{u}, \\
	{y} &= \begin{bmatrix}
	{B}^T & {0} \\
	\end{bmatrix} \begin{pmatrix}
	\bm{e}\\
	\bm{\lambda} \\
	\end{pmatrix},
	\end{aligned} 
	\end{equation*}
	Apply the control law $\bm{u} = -K\bm{y}, \ K>0$
	\begin{equation*}
	\begin{bmatrix}
	M & {0} \\
	{0} & {0} \\
	\end{bmatrix}
	\frac{d}{d t}
	\begin{pmatrix}
	\bm{e}\\
	\bm{\lambda} \\
	\end{pmatrix}
	= \begin{bmatrix}
	{J} - {R} & {G} \\
	-{G}^T & {0} \\
	\end{bmatrix}
	\begin{pmatrix}
	\bm{e}\\
	\bm{\lambda} \\
	\end{pmatrix},
	\end{equation*}
	with ${R} = {B} {K} {B}^T \succeq 0$. \\
	The Hamiltonian $\dot{H} = - \bm{e}^T R \bm{e} \le 0$ is a non increasing function and by La~Salle principle the equilibrium point $\bm{e} = 0$ is asymptotically stable.	
}

\end{frame}
\begin{frame}
\begin{center}
	\only<1>{Control parameter ($t_{\text{end}} = 5 [\mathrm{s}]$)
		\begin{equation*}
		K = 
		\begin{cases}
		0, \quad &\forall t < 1 \, [\mathrm{s}], \\
		100, \quad &\forall t \ge 1 \, [\mathrm{s}].
		\end{cases}
		\end{equation*}
		\vspace{.1cm} \\
		\includemedia[
		label=vidDam,
		addresource=/home/a.brugnoli/Videos_defense/Kirchh_Damped.mp4,
		activate=pageopen,
		width=10cm, height=5cm,
		flashvars={
			source=/home/a.brugnoli/Videos_defense/Kirchh_Damped.mp4
			&loop=true
		}
		]{}{VPlayer.swf}
		
		\mediabutton[
		mediacommand=vidDam:playPause,
		]{\fbox{Play/Pause}}
		
		%\movie[width=0.42\textwidth, height = 0.7 \textheight]{Damped Kirchhoff Plate}{../Videos/Kirchh_Damped_4faster.mp4}			
	}

\end{center}
\end{frame}

\begin{frame}{Conclusion}
The following has been presented:
	\begin{itemize}
	\onslide<2->{\item the Kirchhoff plate model as a port Hamiltonian system;}
	\onslide<3->{\item a structure preserving discretization method capable of dealing with generic bcs;}
	\onslide<4->{\item interconnection with rigid elements (multibody framework);}
		\onslide<5->{\item a simple control application by damping injection;}
	\end{itemize}
	\onslide<6->{Still no rigorous proof of convergence for the finite elements. Existing solutions (only for static problems):}
	\begin{itemize}
		\onslide<7->{\item The Hellan-Herrmann-Johnson  method, but difficulties when dealing with inhomogeneous bcs;}
		\onslide<8->{\item New discretization method capable that handles inhomogeneous bcs.}
	\end{itemize}
\end{frame}


\begin{frame}{}
\centering
\Huge Thanks for your attention \\
\Huge Questions?
\end{frame}


\end{document}
