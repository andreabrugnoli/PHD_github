\chapter*{Conclusions and future directions}
\addcontentsline{toc}{chapter}{Conclusions and future directions}

\epigraph{Je n’ai cherché de rien prouver, mais de bien peindre et d’éclairer bien ma peinture.}{\textit{André Gide \\ Préface de L'Immoraliste}}

\lettrine{\color{theme}{T}}his works has investigated the benefits of the pH formalism as a physics based modelling paradigm. Particular attention has been devoted to continuum mechanics models. These models are of hyperbolic nature and exhibit a partitioned structure when rephrased in  pH form. This partitioned structure is intimately connected with an abstract integration by parts formula. These two concepts are crucial to demonstrate the well-posedness of linear pH system on multi-dimensional domains \cite{skrepek2019wellposedness}. Furthermore, they are at the core of the proposed finite element based discretization strategy. Because it is based on the partitioned structure of the problem, this methodology goes under the name of Partitioned Finite Element Method. As far as the system under consideration possesses this partitioned structure, the method remains applicable. Hence, it is not restricted to  hyperbolic system but can be extended to parabolic systems \cite{serhani2019discretization}. Non-linearities inherent in the Hamiltonian are easily handled as well, since the constitutive law are discretized separately from the dynamics. The proposed discretization has been implemented using finite elements. There is a clear connection between PFEM, mixed finite element discretization and standard finite element discretization. For this reason a number of known finite elements can be used to achieve a structure-preserving discretization. Many conjectures for the error estimates are proposed and validated through numerical experiments, assessing the validity and the performance of the numerical schemes. An application field that strongly benefits from the modularity of pHs is the dynamics of multibody system. It was shown that classical Lagrangian models based on the floating frame of reference paradigm can be recast as a coupled system of ODEs and PDEs in pH form. The floating frame of reference is employ since it allows incorporating all linear mechanical models discussed in this thesis. The discretization used is then similar to a standard discretization with reduced integration of the stresses. It is then possible to connect together subcomponents to model mechanisms modularly. \\

There are many points that require further investigations.


\paragraph{Modelling}
Only linear plate models have been formulated as pH systems. It is of interest how thin linear structures on manifolds, i.e. shells, can be formulated in terms of Hamiltonian systems. Boundary values problems have been formulated for the membrane problem and the Koiter shell \cite{ciarlet2000shells}. It should be possible to formulate these problems as Hamiltonian systems by making use of differential geometry tools. To avoid the use of differential geometry, tangential differential calculus \cite{delfour2011shapes} may be employed. This framework has been successfully adopted to provide an intrinsic formulation of Kirchhoff \cite{schollhammer2019kirchhoff} and Mindlin \cite{schollhammer2019reissner} shell problems. \\
For what concerns non linear models, it would be of great interest to see if the F\"oppl–von K\`arm\'an equations describing the large deformations of large thin plates \cite{bilbao2015conservative} admit a pH reformulation. Apart from structural mechanics problems, many other topics may reveal the benefits of the pH framework: fluid dynamics, electromagnetism, and the coupling of those (i.e. plasma physics).


\paragraph{Discretization}
Some issues deserve a deeper analysis concerning the discretization of plate models. \\
The Kirchhoff plate discretization is achieved using a Dual-Mixed discretization and the non-conforming HHJ method. The first method is computationally expensive and for this reason has not been analyzed in the literature. For second strategy, inhomogeneous boundary conditions in the presence of free boundary conditions are not easy to handle. To cope with the limitations of these two methodologies, the discretization proposed in \cite{rafetseder2018siam} may be used.  This formulation allows using $C^0$ functions to approximate this problem with generic inhomogeneous boundary conditions. It would be of great interest to extend this method for the dynamic case. \\ For the Midlin plate a key point to address is the shear locking phenomenon. Mixed finite element allows constructing approximations of the static problem that are not affected by this phenomenon \cite{veiga2013}. For the dynamic case things get more complicated as the thickness play a role also in the inertial terms. \\
To implement PFEM finite elements approximation have been used. However spectral method can be used as well

\paragraph{Model reduction}


\paragraph{Flexible multibody dynamics}


\paragraph{Control}



