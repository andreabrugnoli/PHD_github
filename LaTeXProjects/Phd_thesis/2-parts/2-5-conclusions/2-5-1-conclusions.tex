\chapter*{Conclusions and future directions}
\addcontentsline{toc}{chapter}{Conclusions and future directions}

\epigraph{Je n’ai cherché de rien prouver, mais de bien peindre et d’éclairer bien ma peinture.}{\textit{André Gide \\ Préface de L'Immoraliste}}

\lettrine{\color{theme}{T}}his work has investigated the benefits of the pH formalism as a physics-based modelling paradigm. Particular attention has been devoted to continuum mechanics models. These models are of hyperbolic nature and exhibit a partitioned structure when rephrased in  pH form. This partitioned structure is intimately connected with an abstract integration by parts formula. These two concepts are crucial to demonstrate the well-posedness of linear pHs on multi-dimensional domains \cite{skrepek2019wellposedness}. Furthermore, they are at the core of the proposed finite element based discretization strategy. Because it is based on the partitioned structure of the problem, this methodology goes under the name of Partitioned Finite Element Method. As far as the system under consideration possesses this partitioned structure, the method remains applicable. Hence, it is not restricted to  hyperbolic systems but can be extended to parabolic systems \cite{serhani2019discretization}. Non-linearities associated to the Hamiltonian are easily handled as well, since the constitutive law are discretized separately from the dynamics. The proposed discretization has been implemented using finite elements. There is a clear connection between the Partitioned Finite Element Method, mixed finite elements and standard finite elements discretization. For this reason a number of known finite elements can be used to achieve a structure-preserving discretization. Many conjectures for the error estimates are proposed and validated through numerical experiments, assessing the validity and the performance of the numerical schemes. The developed algorithms can be used to develop model-based control strategies. This is illustrated by means of the simple damping injection method. An application field that strongly benefits from the modularity of pHs is the dynamics of multibody system. It was shown that classical Lagrangian models based on the floating frame of reference paradigm can be recast as a coupled system of ODEs and PDEs in pH form. The floating frame of reference formulation is employed since it allows incorporating all linear mechanical models discussed in this thesis. The discretization used is then similar to a standard discretization with reduced integration of the stresses. It is then possible to connect together subcomponents to model mechanisms modularly. \\ There are many points that require further investigations. Possible future directions concern the following topics.


\paragraph{Modelling}
Only linear plate models have been formulated as pH systems. It is of interest to see how thin linear structures on manifolds, i.e. shells, can be formulated in terms of Hamiltonian systems. Boundary values problems have been formulated for the membrane problem and the Koiter shell \cite{ciarlet2000shells}. It should be possible to formulate these problems as Hamiltonian systems by making use of differential geometry tools. To avoid the use of differential geometry, tangential differential calculus \cite{delfour2011shapes} may also be employed. This framework has been successfully adopted to provide an intrinsic formulation of Kirchhoff \cite{schollhammer2019kirchhoff} and Mindlin \cite{schollhammer2019reissner} shell problems. \\
For what concerns non linear models, it would be of great interest to see if the F\"oppl–von K\'arm\'an equations, describing the large deformations of thin plates \cite{bilbao2015conservative}, admit a pH reformulation. \\


\paragraph{Discretization}
Some issues deserve a deeper analysis concerning the discretization of plate models. The Kirchhoff plate discretization is achieved using a Dual-Mixed discretization and the non-conforming HHJ method. The first method is computationally expensive and for this reason has not been analyzed in the literature. For the second strategy, inhomogeneous boundary conditions in the presence of free boundary conditions are not easy to handle. To cope with the limitations of these two methodologies, the discretization proposed in \cite{rafetseder2018siam} may be used.  This formulation allows using $C^0$-functions to approximate this problem with generic inhomogeneous boundary conditions. It would be of great interest to extend this method for the dynamic case. \\ For the Mindlin plate, a key point to address is the shear locking phenomenon. Mixed finite element allows constructing approximations of the static problem that are not affected by this phenomenon \cite{veiga2013}. For the dynamical case things get more complicated as the thickness plays a role also in the inertial terms. \\ A complete convergence study for the boundary-controlled wave equation is carried out in \cite{haine2020numerical}. Rigorous convergence studies for the models proposed in this work should be performed as well. \\ For the numerical implementation finite elements approximation has been used. However, spectral methods could be used as well. This would provide discretized systems of small dimension with dense matrices, drastically reducing the computational burdens. In particular modal analysis techniques may be employed to construct approximation for given frequency bandwidth.

\paragraph{Model reduction}
Model reduction strategies have not been addressed in this manuscript. Given the size of the matrices arising from the discretization, this remains a fundamental issue. Promising methodologies, relying on Proper Orthogonal Decomposition and $H_2$-optimal subspaces for non-linear pHODE \cite{chaturantabut2016} and on Krylov methods for linear pHDAE \cite{egger2018}, are already available.  In practice it is of interest to reduce the system accurately in a limited frequency bandwidth, representative of the system and instrumentation dynamics \cite{vuillemin2014frequency}. Structure-preserving frequency limited model reduction techniques have been recently extended to pH systems \cite{xu2020sp}. It would be of great interest to apply these techniques to the models proposed in this work.

\paragraph{Flexible multibody dynamics}
The proposed formulation gives rise to a stiff differential-algebraic system of index two. The resolution of these kind of problems is notoriously hard \cite{brenan1995dae}. There is a need for reliable and efficient computational tools for the time-integration. Methods preserving the passive nature of the system are of course preferable. \\ Another interesting topic would be the inclusion of large deformations. In principle, this should be possible using a co-rotational formulation or substructuring techniques \cite{wu1988substructuring}.

\paragraph{Control}
The pH framework has proven to be rather effective for the design of control laws for non-linear systems \cite{ortega2004survey} and 1-dimensional boundary control PDEs \cite{macchelli2020exponential}. A still open subject is the introduction of performance specifications in the pH formalism. Performance specifications are usually expressed in the frequency domain. For infinite-dimensional systems, recent works address the implementation of controllers based on $H_\infty$ techniques \cite{apkarian2018structured,apkarian2020bd}. These strategies are applicable to parabolic and hyperbolic PDEs. For pH systems, it would be of interest to see if robust and passivity-based control can be combined together.  Furthermore it would be important to consider the case of non-collocated controls and observations \cite{cardoso2016}. By introducing appropriate state estimators \cite{yaghmaei2019}, it should be possible to reconstruct the conjugated input to guarantee a passive output feedback.


