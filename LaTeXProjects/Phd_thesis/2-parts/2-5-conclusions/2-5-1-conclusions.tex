\chapter*{Conclusions and future directions}
\addcontentsline{toc}{chapter}{Conclusions and future directions}

\epigraph{Je n’ai cherché de rien prouver, mais de bien peindre et d’éclairer bien ma peinture.}{\textit{André Gide \\ Préface de L'Immoraliste}}

\lettrine{\color{theme}{T}}his works has investigated the benefits of the pH formalism as a physics based modelling paradigm. Particular attention has been devoted to continuum mechanics models. These models are of hyperbolic nature and exhibit a partitioned structure when rephrased in  pH form. This partitioned structure is intimately connected with an abstract integration by parts formula. These two concepts are crucial to demonstrate the well-posedness of linear pH system on multi-dimensional domains \cite{skrepek2019wellposedness}. Furthermore, they are at the core of the proposed finite element based discretization strategy. Because it is based on the partitioned structure of the problem, this methodology goes under the name of Partitioned Finite Element Method. As far as the system under consideration possesses this partitioned structure, the method remains applicable. Hence, it is not restricted to  hyperbolic system but can be extended to parabolic systems \cite{serhani2019discretization}. Non-linearities inherent in the Hamiltonian are easily handled as well, since the constitutive law are discretized separately from the dynamics. The proposed discretization has been implemented using finite elements. There is a clear connection between PFEM, mixed finite element discretization and standard finite element discretization. For this reason a number of known finite elements can be used to achieve a structure-preserving discretization. Many conjectures for the error estimates are proposed and validated through numerical experiments, assessing the validity and the performance of the numerical schemes. An application field that strongly benefits from the modularity of pHs is the dynamics of multibody system. It was shown that classical Lagrangian models based on the floating frame of reference paradigm can be recast as a coupled system of ODEs and PDEs in pH form. The floating frame of reference is employ since it allows incorporating all linear mechanical models discussed in this thesis. The discretization used is then similar to a standard discretization with reduced integration of the stresses. It is then possible to connect together subcomponents to model mechanical mechanism.






