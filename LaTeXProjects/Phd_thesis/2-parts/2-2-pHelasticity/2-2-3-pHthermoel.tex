\chapter{Thermoelasticity in port-Hamiltonian form}

\epigraph{Eh bien, mon ami, la terre sera un jour ce cadavre refroidi. Elle deviendra inhabitable et sera inhabitée comme la lune, qui depuis longtemps a perdu sa chaleur vitale.}{\textit{Vingt mille lieues sous les mers\\
Jules Verne}}
\minitoc
 
\lettrine{\color{theme}{T}}hermoelasticity is the study of deformable bodies undergoing thermal excitations. It is a clear example of a multiphysics phenomenon since the heat transfer and elastic vibrations within the body mutually interact. In this chapter, a linear model of thermoelasticity is obtained under the pH formalism. Each physics is described separately and the final system is obtained considering a power-preserving interconnection of two pHs.

\begin{comment}
The first work on this discipline dates back to \cite{duhamel1837}, but it was only more than a century later, thanks to the paper of Biot \cite{biot1956thermoelasticity}, that research on this topic received a new impulse.
\end{comment}

\section{Port-Hamiltonian linear coupled thermoelasticity}\label{sec:phTher}
In this section, a pH formulation of heat transfer is first introduced. The classical model of thermoelasticity is then recalled. The same model is found by interconnecting the heat equation and the linear elastodynamics problem seen as pHs. It is shown that the interconnection preserves a quadratic functional that plays the role of a fictitious energy. The resulting system is dissipative  with respect to this functional. The proposed construction makes use of the intrinsic modularity of pHs \cite{kurula2010} to derive a multiphysics composing together each physical realm.

\subsection{The heat equation as a pH descriptor system}
Consider the heat equation in a bounded connected set $\Omega \subset \mathbb{R}^d, \, d=\{1,2,3\}$, describing the evolution of the temperature field $T(\bm{x}, t)$
\begin{equation}\label{eq:heatEq}
\rho c_V \diffp{T}{t} = \lambda \Delta T + r_Q, \qquad \bm{x} \in \Omega,
\end{equation}
where $\rho, c_V, k, r_Q$ are the mass density, the specific heat density at constant strain, the thermal diffusivity and an heat source. The Dirichlet and Neumann condition of this problem are 
\begin{equation*}
	\begin{aligned}
	T \text{ known on } \Gamma_D^T \times (0, +\infty), \qquad \text{Dirichlet condition}, \\
	-k \, \grad T \cdot \bm{n} \text{ known on } \Gamma_N^T \times (0, +\infty) \qquad \text{Neumann condition},
	\end{aligned}
\end{equation*}
where a partition of the boundary $\partial \Omega = \Gamma_D^T \cup \Gamma_N^T$ has been considered. This model can be put in pH form by means of a canonical interconnection structure ( cf. \cite[Chapter 2]{kotyczka2019numerical}). An algebraic relatioship that describes the Fourier law has to be incorporated in the model. Here, a differential-algebraic formulation is exploited to obtain the same system. \\

Let $T_0$ be a constant reference temperature (the introduction of this variables is instrumental for coupled thermoelasticity). The functional 
\begin{equation*}
	H_T = \frac{1}{2} \int_\Omega \rho c_V T_0 \left(\frac{T-T_0}{T_0} \right)^2 \d\Omega
\end{equation*} 
has the physical dimension of an energy and represents a Lyapunov functional of this system. Even though it does not represent the internal energy, it has some important properties. Select as energy variable 
\begin{equation*}
\alpha_T := \rho c_V (T-T_0),
\end{equation*}
whose corresponding co-energy is 
\begin{equation*}
	e_T := \diffd{H_T}{\alpha_T} = \frac{\alpha_T}{\rho c_V T_0} = \frac{T-T_0}{T_0} =: \theta.
\end{equation*}
Introducing the heat flux $\bm{j}_Q := -k \grad T$ as additional variable, the heat equation \eqref{eq:heatEq} is equivalently reformulated as
\begin{equation}\label{eq:phsysHeat}
\begin{aligned}
\begin{bmatrix}
1 & 0 \\
\bm{0} & \bm{0} \\
\end{bmatrix}
\diffp{}{t}
\begin{pmatrix}
\alpha_T \\
\bm{j}_Q \\
\end{pmatrix} &= 
\begin{bmatrix}
0 & -\div \\
-\grad & -(T_0 k)^{-1} \\
\end{bmatrix}
\begin{pmatrix}
e_T \\
\bm{j}_Q \\
\end{pmatrix} + 
\begin{bmatrix}
1 \\
\bm{0} \\
\end{bmatrix} u_T, \\
y_T &= \begin{bmatrix}
1 & \bm{0} \\
\end{bmatrix} \begin{pmatrix}
e_T \\
\bm{j}_Q \\
\end{pmatrix}.
\end{aligned}
\end{equation}
with $u_T:=r_Q$ and $y_T$ represents the corresponding power-conjugated variable. In matrix notation, it is obtained
\begin{equation}
\begin{aligned}
\mathcal{E}_T \partial_{t} \bm{\alpha}_T  &= \left(\mathcal{J}_T - \mathcal{R}_T \right)\bm{e}_T + \mathcal{B}_{T}\, u_T, \\
y_d &= \mathcal{B}_{T}^* \, \bm{e}_T
\end{aligned}
\end{equation}
where $\bm{\alpha}_T = (\alpha_T,\ \bm{j}_Q), \; \bm{e}_T = (e_T,\ \bm{j}_Q)$ and
\begin{equation*}
\mathcal{E}_T = \begin{bmatrix}
1 & 0 \\
\bm{0} & \bm{0} \\
\end{bmatrix}, \quad
\mathcal{J}_T = \begin{bmatrix}
0 & -\div \\
-\grad & \bm{0} \\
\end{bmatrix}, \quad 
\mathcal{R}_T = \begin{bmatrix}
0 & 0 \\
\bm{0} & (T_0 k)^{-1} \\
\end{bmatrix}, \quad
\mathcal{B}_T = \begin{bmatrix}
1 \\
\bm{0} \\
\end{bmatrix}.
\end{equation*}
The system is an example of pH descriptor system (cf. \cite{beattie2018linear} for the finite dimensional case). The Hamiltonian reads 
\begin{equation}\label{eq:hamHeat}
	H_T = \frac{1}{2}\int_\Omega \bm{e}_T \cdot \mathcal{E}_T\bm{\alpha}_T \d\Omega.
\end{equation}
The power rate is then deduced
\begin{equation}\label{eq:enrateHeat}
\begin{aligned}
\dot{H}_T &= \int_\Omega \bm{e}_T \cdot \mathcal{E}_T\, \partial_t \bm{\alpha}_T \d\Omega, \\
		&= \int_\Omega \bm{e}_T \cdot \left\{(\mathcal{J}_T-\mathcal{R}_T) \bm{e} + \mathcal{B}_T u_T \right\} \d\Omega, \\
		&= \int_\Omega u_T \, y_T \d\Omega -\int_\Omega \left(e_T \div \bm{j}_Q + \bm{j}_Q \grad e_T + \frac{\norm{\bm{j}_Q}^2}{k T_0} \right) \d\Omega, \\
		&\le \int_\Omega u_T \, y_T \d\Omega -\int_{\partial\Omega} e_T \ \bm{j}_Q \cdot \bm{n} \d{S}.
\end{aligned}
\end{equation}
This choice of Hamiltonian allows retrieving the classical boundary conditions and leads to a dissipative system. Other formulations, based on an entropy or internal energy functionals are possible for the heat equation \cite{serhani2019modeling}. Unfortunately these formulations are non linear and their discretization is a difficult task \cite{serhani2019discretization}. 


\subsection{Classical thermoelasticity}
The derivation of the classical theory of thermoelasticity is not carried out here. The reader may consult in \cite[Chapter 1]{hetnarski2009thermal} or \cite[Chapter 8]{abeyaratne2012notes} for a detailed discussion on this topic. 

Consider a bounded connected set $\Omega \subset \mathbb{R}^d, \, d=\{1,2,3\}$. The classical equations for linear fully-coupled thermoelasticity for an isotropic thermoelastic material are \cite{carlson1973}
\begin{equation}\label{eq:sysThElas}
\begin{aligned}
\displaystyle \rho \diffp[2]{\bm{u}}{t} &= \Div(\bm{\Sigma}_{ET}) , \\
\displaystyle \rho c_V \diffp{T}{t} &= -\div(\bm{j}_Q) - \mathcal{I}_\beta \cddot \displaystyle\diffp{\bm{\varepsilon}}{t}, \\
\bm{\Sigma}_{ET} &= \bm{\Sigma}_E + \bm{\Sigma}_{T}, \\
\bm{\Sigma}_E &= 2\mu \bm{\varepsilon} + \lambda \Tr(\bm{\varepsilon}) \bm{I}_{d\times d}, \\
\bm{\Sigma}_T &= - \mathcal{I}_\beta \theta,  \\
\bm{\varepsilon} &= \Grad(\bm{u}), \\
\bm{j}_Q &= -k \, \grad T.
\end{aligned}
\end{equation}
For simplicity  the coupling term
\[\mathcal{I}_\beta:=T_0 \beta(2\mu + d \lambda)\bm{I}_{d\times d} \]
has been introduced. Field $\bm{u}$ is the displacement, $\bm{\varepsilon}$ is the infinitesimal strain tensor, $\bm{\Sigma}_E, \bm{\Sigma}_T$ are the stress tensor contribution due to mechanical deformation and a thermal field. Coefficients $\lambda, \mu$ are the Lam\'e parameters,  and $\beta$ the thermal expansion coefficient. Given a partition of the boundary $\partial \Omega = \Gamma_D^E \cup \Gamma_N^E = \Gamma_D^T \cup \Gamma_N^T$ for the elastic and thermal domain. The general homogeneous boundary conditions read 
\begin{equation}
\begin{aligned}
\bm{u} \text{ known on } \Gamma_D^E \times (0, +\infty), \\
\bm{\Sigma}_{ET} \cdot \bm{n} \text{ known on } \Gamma_N^E \times (0, +\infty), 
\end{aligned} \qquad
\begin{aligned}
T \text{ known on } \Gamma_D^T \times (0, +\infty), \\
\bm{j}_Q \cdot \bm{n} \text{ known on } \Gamma_N^T \times (0, +\infty).
\end{aligned}
\end{equation}

\section{Thermoelasticity as two coupled pHs}

Consider again the equation of elasticity on $\Omega \subset \mathbb{R}^d, \, d =\{2, 3\}$ (cf. Eq. \eqref{eq:phsysElas}), together with a distributed input $\bm{u}_E$ that plays the role of a distributed force
\begin{equation}
\begin{aligned}
\diffp{}{t}
\begin{pmatrix}
\bm{\alpha}_v \\
\bm{A}_{\varepsilon} \\
\end{pmatrix} &= 
\begin{bmatrix}
\bm{0} & \Div \\
\Grad & \bm{0} \\
\end{bmatrix}
\begin{pmatrix}
\bm{e}_v \\
\bm{E}_{\varepsilon} \\
\end{pmatrix} + 
\begin{bmatrix}
\bm{I}_{d\times d} \\
\bm{0} \\
\end{bmatrix}\bm{u}_E, \\
\bm{y}_E &= \begin{bmatrix}
\bm{I}_{d\times d} & \bm{0} \\
\end{bmatrix}
\begin{pmatrix}
\bm{e}_v \\
\bm{E}_{\varepsilon}
\end{pmatrix},
\end{aligned}
\end{equation}
with Hamiltonian  
\[
H_E = \energy{\bm{\alpha}_v \cdot \bm{e}_v + \bm{A}_{\varepsilon} \cddot \bm{E}_{\varepsilon}}.
\]
Recall the pH formulation of the heat equation \eqref{eq:phsysHeat}
\begin{equation}
\begin{aligned}
\begin{bmatrix}
1 & 0 \\
\bm{0} & \bm{0} \\
\end{bmatrix}
\diffp{}{t}
\begin{pmatrix}
\alpha_T \\
\bm{j}_Q \\
\end{pmatrix} &= 
\begin{bmatrix}
0 & -\div \\
-\grad & -(T_0 k)^{-1} \\
\end{bmatrix}
\begin{pmatrix}
e_T \\
\bm{j}_Q \\
\end{pmatrix} + 
\begin{bmatrix}
1 \\
\bm{0} \\
\end{bmatrix} u_T, \\
y_T &= \begin{bmatrix}
1 & \bm{0} \\
\end{bmatrix} \begin{pmatrix}
e_T \\
\bm{j}_Q \\
\end{pmatrix},
\end{aligned}
\end{equation} 
with Hamiltonian $H_T$ defined in \eqref{eq:hamHeat}. The linear thermoelastic problem can be expressed as a coupled port-Hamiltonian system.  Consider the following interconnection
\begin{align}
\bm{u}_E &= - \Div(\mathcal{I}_\beta \, y_T), \\
u_T &= - \mathcal{I}_\beta \cddot\Grad(\bm{y}_E). 
\end{align}
The interconnection is power preserving as it can be compactly written as 
\begin{equation*}
\bm{u}_E = \mathcal{C}_\beta(y_T), \qquad u_T = - \mathcal{C}_\beta^*(\bm{y}_E).
\end{equation*}
where the coupling operator $\mathcal{C}_\beta := - \Div(\mathcal{I}_\beta \, \cdot) : L^2(\Omega) \rightarrow L^2(\Omega, \mathbb{R}^d)$ has formal adjoint $ \mathcal{C}_\beta^* =  \mathcal{I}_\beta^* \cddot  \Grad(\cdot) =  \mathcal{I}_\beta \cddot  \Grad(\cdot): L^2(\Omega, \mathbb{R}^3) \rightarrow L^2(\Omega)$ ($\mathcal{I}_\beta$ is self adjoint given its diagonal structure). As a consequence, under the assumption that $\bm{y}_E, y_T$ have compact support, it holds
\[
\left\langle u_T, y_T \right\rangle_{L^2(\Omega)} + \left\langle \bm{u}_E, \bm{y}_E \right\rangle_{L^2(\Omega, \mathbb{R}^3)} = 0.
\]
If the compact support assumption is removed, it is obtained
\begin{equation}\label{eq:balIntThElas}
	\begin{aligned}
	\left\langle u_T, y_T \right\rangle_{L^2(\Omega)} + \left\langle \bm{u}_E, \bm{y}_E \right\rangle_{L^2(\Omega, \mathbb{R}^3)} &= -\int_\Omega \left\{(\mathcal{I}_\beta \cddot \Grad \bm{e}_v) e_T + \Div(\mathcal{I}_\beta e_T) \cdot \bm{e}_v \right\} \d\Omega, \\
	&= -\int_{\Omega} \div(e_T \mathcal{I}_\beta \cdot \bm{e}_v ) \d\Omega, \\
	&= -\int_{\partial \Omega} (e_T \mathcal{I}_\beta \cdot \bm{n}) \cdot \bm{e}_v  \d{S}
	\end{aligned}
\end{equation}
Using the expression of $y_T, \bm{y}_E$, considering that $T_0$ is constant and applying Schwarz theorem for smooth function, the inputs are equal to
\begin{equation*}
\bm{u}_E =  \Div(\bm{\Sigma}_T), \qquad u_T = - \mathcal{I}_\beta \cddot  \Grad(\bm{v}) = - \mathcal{I}_\beta \cddot  \diffp{\bm{\varepsilon}}{t} .
\end{equation*}

The coupled thermoelastic problem can now be written as
\begin{equation}
\begin{bmatrix}
\bm{1} & \bm{0} & \bm{0} & \bm{0}\\
\bm{0} & \bm{1} & \bm{0} & \bm{0}\\
0 & 0 & 1 & 0\\
\bm{0} & \bm{0} & \bm{0} & \bm{0}\\
\end{bmatrix}
\diffp{}{t}
\begin{pmatrix}
\bm{\alpha}_v \\
\bm{A}_\varepsilon \\
{\alpha}_T \\
\bm{j}_Q \\
\end{pmatrix} = 
\begin{bmatrix}
\bm{0} & \Div & - \Div(\mathcal{I}_\beta \, \cdot) & \bm{0}\\
\Grad & \bm{0} & \bm{0} & \bm{0} \\
- \mathcal{I}_\beta \cddot  \Grad(\cdot) & 0 & 0 & -\div \\
\bm{0} & \bm{0} & -\grad & - (T_0 \mathcal{K})^{-1} \\
\end{bmatrix}
\begin{pmatrix}
\bm{e}_v \\
\bm{E}_\varepsilon \\
{e}_T \\
\bm{j}_Q \\
\end{pmatrix},
\end{equation}
with total energy given by $H=H_E + H_T$. The power balance for each subsystem is given by 
\begin{align}
\dot{H}_E &= \int_\Omega \bm{u}_E \cdot \bm{y}_E \d\Omega + \int_{\partial\Omega} \bm{e}_v \cdot (\bm{E}_\varepsilon \cdot \bm{n}) \d{S}, \label{eq:enrateElasU} \\
\dot{H}_T &\le \int_\Omega u_T \ y_T \d\Omega - \int_{\partial\Omega} \theta \ \bm{j}_Q \cdot \bm{n} \d{S}, \label{eq:enrateHeatU}
\end{align}
The overall power balance is easily computed considering Eqs. \eqref{eq:enrateElasU} \eqref{eq:enrateHeatU} and \eqref{eq:balIntThElas}
\begin{equation}\label{eq:enrateIntThElas}
\begin{aligned}
\dot{H} = \dot{H}_E + \dot{H}_T \le \int_{\partial \Omega} \left\{\left[\bm{E}_\varepsilon - e_T \mathcal{I}_\beta \right] \cdot \bm{n}\right\}  \cdot \bm{e}_v  \d{S} - \int_{\partial\Omega} \theta \ \bm{j}_Q \cdot \bm{n} \d{S}
\end{aligned}
\end{equation}
From the power balance the classical boundary conditions are retrieved. This allows defining appropriate boundary operators for the thermoelastic problem
\begin{equation}
	\bm{u}_\partial = 
	\underbrace{\begin{bmatrix}
	\bm{\gamma}_{0}^{\Gamma_D^E} & \bm{0} & \bm{0} & \bm{0} \\
	\bm{0} & \bm{\gamma}_n^{\Gamma_N^E} & - \bm{\gamma}_n^{\Gamma_N^E}(\mathcal{I}_\beta\, \cdot )  & \bm{0}  \\ 
	{0} & {0} & {\gamma}_{0}^{\Gamma_D^T} & {0} \\
	\bm{0} & \bm{0} & \bm{0} & \bm{\gamma}_{n}^{\Gamma_N^T} \\
	\end{bmatrix}}_{\mathcal{B}_\partial}
	\begin{pmatrix}
	\bm{e}_v \\
	\bm{E}_\varepsilon \\
	{e}_T \\
	\bm{j}_Q \\
	\end{pmatrix}, \quad 
	\bm{y}_\partial = 
	\underbrace{\begin{bmatrix}
		\bm{0} & \bm{\gamma}_n^{\Gamma_D^E} & - \bm{\gamma}_n^{\Gamma_D^E}(\mathcal{I}_\beta\, \cdot )  & \bm{0}  \\ 
		\bm{\gamma}_{0}^{\Gamma_N^E} & \bm{0} & \bm{0} & \bm{0} \\
		\bm{0} & \bm{0} & \bm{0} & \bm{\gamma}_{n}^{\Gamma_D^T} \\
		{0} & {0} & {\gamma}_{0}^{\Gamma_N^T} & {0} \\
		\end{bmatrix}}_{\mathcal{B}_\partial}
	\begin{pmatrix}
	\bm{e}_v \\
	\bm{E}_\varepsilon \\
	{e}_T \\
	\bm{j}_Q \\
	\end{pmatrix}.
\end{equation} 

\begin{equation}
\begin{aligned}
\bm{e}_v \text{ known on } \Gamma_D^E \times (0, +\infty), \\
\left(\bm{E}_\varepsilon - \mathcal{I}_\beta \bm{e}_T \right) \cdot \bm{n} \text{ known on } \Gamma_N^E \times (0, +\infty), \\
\end{aligned} \qquad
\begin{aligned}
\bm{e}_T \text{ known on } \Gamma_D^T \times (0, +\infty), \\
\bm{j}_Q \cdot \bm{n} \text{ known on } \Gamma_N^T \times (0, +\infty).
\end{aligned}
\end{equation}


\section{Thermoelastic Euler-Bernoulli beam}

\section{Thermoelastic Kirchhoff plate}
