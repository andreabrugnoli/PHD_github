\chapter{Elasticity in port-Hamiltonian form}

\epigraph{I try not to break the rules but merely to test their elasticity.}{\textit{Bill Veeck}}
\minitoc

Continuum mechanics is the mathematical description of how materials behave kinematically under external excitations. In this framework, the microscopic structure of a material body is neglected and a macroscopic viewpoint, that describes the body as a continuum, is adopted. An elastic material is able to resist distorting excitations and return to its original size and shape when these are removed. In this chapter, the general linear elastodynamics problem is recalled. A suitable port-Hamiltonian realization is then derived using a velocity-stress formulation.

\section{Deformation, strain and stress}
In this section, the main concepts behind a deformable continuum are briefly recalled following \cite{lee2012mixed}. For a detailed discussion on this topic, the reader may consult \cite{abeyaratne2012notes,landau2012elasticity}. \\

The bounded region of $\mathbb{R}^n \; (n=2, 3)$ occupied by a solid is called configuration. The reference configuration $\Omega$ is the domain that a bodies occupies at the initial state. To describe how the body deforms in time the deformation map $\bm\Phi: \Omega \times [0, T_f] \rightarrow \Omega' \subset \mathbb{R}^n$ is introduced. This map is differentiable and orientation preserving and the image of $\Omega$ under $\bm\Phi(\cdot, t) \; \forall t \in [0, T_f]$ is called the deformed configuration $\Omega_t$. The gradient of the deformation map is called the deformation gradient $\bm{F}:=\grad\bm{\Phi}$. A rigid deformation maps a point $\bm{x} \in \mathbb{R}^n \rightarrow \bm{A}(t) \bm{x} + \bm{b}(t)$, where $\bm{A}(t)$ is an orthogonal matrix and $\bm{b}(t)$ a $\mathbb{R}^n$ vector. A differentiable deformation map $\bm\Phi$ is a rigid deformation iff $\bm{F}^\top \bm{F} - \bm{I} = 0$,  where $\bm{I}$ is the identity in $\mathbb{R}^{n\times n}$ (for the proof see \cite{ciarlet1988mathematical}, page 44). For this reason, a suitable measure of the deformation is the Green-St.Venant strain tensor $\frac{1}{2} (\bm{F}^\top \bm{F} - \bm{I})$.  \\

A quantity of interest is the displacement $\bm{u}: \Omega \rightarrow \mathbb{R}^n$ with respect to the reference configuration. It is defined as $\bm{u} = \bm{\Phi}(\bm{x}, t) - \bm{x}$. The gradient of the displacement verifies $\grad\bm{u} = \bm{F} - \bm{I}$. The strain tensor can now be written in terms of the displacement
\begin{align*}
	\frac{1}{2} (\bm{F}^\top \bm{F} - \bm{I}) &= \frac{1}{2}\left[(\grad \bm{u} + \bm{I})^\top (\grad \bm{u} + \bm{I}) - \bm{I}\right] \\
	&= \frac{1}{2}\left[\grad \bm{u} + (\grad \bm{u})^\top + (\grad \bm{u})^\top (\grad \bm{u})\right]
\end{align*}

The linear and angular momentum in a subdomain $\omega \subset \Omega$ are then computed as 
\[
\int_{\omega} \rho \, \bm{v} \d{\omega}, \qquad \int_{\omega} \rho \, \bm{x} \times \bm{v} \d{\omega}
\]

\section{The linear elastodynamics problem}


\section{Port-Hamiltonian formulation}


