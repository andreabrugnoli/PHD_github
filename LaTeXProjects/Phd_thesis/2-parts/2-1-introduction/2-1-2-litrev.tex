\chapter[Literature review]{Literature review}

\epigraph{Books serve to show a man that those original thoughts of his aren't very new after all.}{\textit{Abraham Lincoln}}



\section{Port-Hamiltonian distributed systems}

 For 1D linear PH systems with a generalized skew-adjoint system operator, \cite{legorrec2005} gives conditions on the assignment of boundary inputs and outputs for the system operator to generate a contraction semigroup. The latter is instrumental to show well-posedness of a linear PH
system, see \cite{zwart2012}. Essentially, at most half the number of boundary port variables
can be imposed as control inputs for a well-posed PH system in 1D. The complete characterization of pH in arbitrary dimension is still an open research field. Two notable exceptions \cite{zwart2015wave,skrepek2019wellposedness} provide partial answers to this problem. The first demonstrate the well-posedness of the linear wave equation in arbitrary geometrical dimensions. The second generalizes this result to treat the case of generic first order linear pHs in arbitrary geometrical dimensions. \\


\section{Structure-preserving discretization}

\section{Mixed finite element for elasticity}

\section{Multibody dynamics}