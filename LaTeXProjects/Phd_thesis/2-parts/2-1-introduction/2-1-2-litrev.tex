\chapter[Literature review]{Literature review}

\epigraph{Books serve to show a man that those original thoughts of his aren't very new after all.}{\textit{Abraham Lincoln}}



\section{Port-Hamiltonian distributed systems}
\textbf{Differential geometry}
An interesting reference that can provide some ideas in this direction is \cite{yao2011modeling,nishida2004}. 

 For 1D linear PH systems with a generalized skew-adjoint system operator, \cite{legorrec2005} gives conditions on the assignment of boundary inputs and outputs for the system operator to generate a contraction semigroup. The latter is instrumental to show well-posedness of a linear PH system, see \cite{zwart2012}. Essentially, at most half the number of boundary port variables
can be imposed as control inputs for a well-posed PH system in one-dimensional domains. The complete characterization of pH in arbitrary dimension is still an open research field. Two notable exceptions \cite{zwart2015wave,skrepek2019wellposedness} provide partial answers to this problem. The first demonstrate the well-posedness of the linear wave equation in arbitrary geometrical dimensions. The second generalizes this result to treat the case of generic first order linear pHs in arbitrary geometrical dimensions. \\


\section{Structure-preserving discretization}

\section{Mixed finite element for elasticity}

Thanks to \cite{CardosoRibeiro2018}, it has become evident that there is a strict link between  discretization of port-Hamiltonian (pH) systems and mixed finite elements. Velocity-stress formulation for the wave dynamics and elastodynamics problems are indeed Hamiltonian and their mixed discretization preserves such a structure. For instance in \cite{kirby2015} the authors employed mixed finite elements to obtain a  symplectic semi-discretization for the wave equation. This allows using known finite element scheme to preserve the pH structure at the discrete level.

Mixed finite elements for the wave equation have been studied in \cite{geveci1988,becache2000wave}. For elastodynamics the construction of stable elements gets more complicated because of the presence of the symmetric stress tensor. Existing elements enforce symmetry either strongly \cite{becache2001elas} or weakly \cite{arnold2014elastodynamics}.

\section{Multibody dynamics}