% Math macros
\DeclareMathOperator*{\grad}{grad}
\DeclareMathOperator*{\Grad}{Grad}
\DeclareMathOperator*{\Div}{Div}
\renewcommand{\div}{\operatorname{div}}
\DeclareMathOperator*{\Hess}{Hess}
\DeclareMathOperator*{\curl}{curl}
\DeclareMathOperator{\Tr}{Tr}
\newcommand*{\norm}[1]{\ensuremath{\left\|#1\right\|}}


\newcommand{\where}{\qquad \text{where} \qquad}
\newcommand{\inner}[3][]{\ensuremath{\langle #2, #3 \rangle_{#1}}}
\newcommand{\bilprod}[2]{\langle \langle \, #1, #2 \, \rangle \rangle}
\newcommand{\pder}[2]{\ensuremath{\partial_{#2} #1}}
\newcommand{\dder}[2]{\ensuremath{\delta_{#2} #1}}
\newcommand{\secref}[1]{\S\ref{#1}}

\newcommand*{\SET}[1]  {\ensuremath{\mathrm{\mathbf{#1}}}}
\newcommand*{\VEC}[1]  {\ensuremath{\boldsymbol{#1}}}
\newcommand*{\MAT}[1]  {\ensuremath{\boldsymbol{#1}}}
\newcommand*{\OP}[1]  {\ensuremath{\boldsymbol{\mathcal{#1}}}}
\newcommand*{\ESP}[1]  {\ensuremath{ \mathbb{E} \left \{#1 \right \}}}
\newcommand*{\ESPENS}[2]  {\ensuremath{ \mathbb{E}_{#1} \left \{#2 \right \}}}
\newcommand*{\DPR}[2]  {\ensuremath{\left \langle #1,#2 \right \rangle}}
\newcommand*{\FOURIER}[1]  {\ensuremath{\widehat{#1}}}
\newcommand{\eqdef}{\stackrel{\mathrm{def}}{=}}
\newcommand{\argmax}{\operatornamewithlimits{argmax}}
\newcommand{\argmin}{\operatornamewithlimits{argmin}}
\newcommand{\diag}{\operatorname{diag}}
\newcommand{\ud}{\, \mathrm{d}}
\newcommand{\vect}{\mathrm{Vect}}
\newcommand{\sinc}{\mathrm{sinc}}
\newcommand{\esp}{\ensuremath{\mathrm{E}}} % Problème pour les short captions
\newcommand{\hilbert}{\ensuremath{\mathcal{H}}}
\newcommand{\supps}{\ensuremath{\tilde{\mathrm{supp}}}}
\newcommand{\supp}{\ensuremath{\mathrm{supp}}}
\newcommand{\sgn}{\mathrm{sgn}}
\newcommand{\intTT}{\int_{-T}^{T}}
\newcommand{\intT}{\int_{-\frac{T}{2}}^{\frac{T}{2}}}
\newcommand{\intinf}{\int_{-\infty}^{+\infty}}
\newcommand{\iintinf}{\iint_{-\infty}^{+\infty}}
\newcommand{\iintrr}{\iint\limits_{\SET{R}^2}}
\newcommand{\intr}{\int\limits_{\R}}
\newcommand{\Sh}{\ensuremath{\boldsymbol{U}}}
\newcommand{\C}{\ensuremath{\mathbf{C}}}
\newcommand{\R}{\ensuremath{\mathbf{R}}}
\newcommand{\Z}{\ensuremath{\mathbf{Z}}}
\newcommand{\N}{\ensuremath{\mathbf{N}}}
\newcommand{\K}{\ensuremath{\mathbf{K}}}
\newcommand{\reel}{\mathcal{R}}
\newcommand{\imag}{\mathcal{I}}
\newcommand{\cmnr}{c_{m,n}^\reel}
\newcommand{\cmni}{c_{m,n}^\imag}
\newcommand{\cnr}{c_{n}^\reel}
\newcommand{\cni}{c_{n}^\imag}
\newcommand{\tproto}{g}
\newcommand{\rproto}{\check{g}}
\newcommand{\Tproto}{G}
\newcommand{\Rproto}{\check{G}}
\newcommand{\Tpoly}{F}
\newcommand{\Rpoly}{\check{F}}
\newcommand{\estim}{\tilde{c}}
\newcommand{\egal}{\bar{c}}
\newcommand{\bb}{b}
\newcommand{\bbf}{z}
\newcommand{\bbr}{\zeta}
\newcommand{\LR}{\mathcal{L}_2(\R)}
\newcommand{\LRR}{\mathcal{L}_2(\R^2)}
\newcommand{\LZ}{\ell_2(\Z)}
\newcommand{\LZZ}{\ell_2(\Z^2)}
\newcommand{\peigne}{\ensuremath{\Psi}}
\newcommand{\avec}{\qquad \text{avec} \qquad}

% Theorems definition
\newtheoremstyle{break}
  {11pt}{11pt}%
  {\itshape}{}%
  {\bfseries}{}%
  {\newline}{}%
\theoremstyle{break}

\newtheorem{theorem}{Theorem}
\newtheorem{definition}{Definition}
\newtheorem{remark}{Remark}
\newtheorem{proposition}{Proposition}
\newtheorem{corollary}{Corollary}

\newtheorem{définition}{Définition}[chapter]
\newtheorem{theoreme}{Théorème}[chapter]
\newtheorem{remarque}{Remarque}[chapter]
\newtheorem{propriete}{Propriété}[chapter]
\newtheorem{exemple}{Exemple}[chapter]



\def\blankpage{%
      \clearpage%
      \thispagestyle{empty}%
      \addtocounter{page}{-1}%
      \null%
      \clearpage}

% Example of background tag
% \AddToShipoutPicture{%
% \begin{tikzpicture}[remember picture,overlay]
%   \node [rotate=60,scale=10,text opacity=0.1] at (current page.center) {Brouillon};
% \end{tikzpicture}}

% Example of header tag
% \AddToShipoutPicture{%
% \tikzstyle{block} = [draw, thick, color=blue, scale=1.5,rectangle, minimum height=3em, minimum width=6em]
% \begin{tikzpicture}[remember picture,overlay]
%   \node [coordinate] at (current page.north) (accroche) {};
%   \node [block, below of=accroche] {Diffusion restreinte};
% \end{tikzpicture}}

% Another example of header tag
% \AddToShipoutPicture{%
% \tikzstyle{block} = [draw, thick, color=red, scale=1.5,rectangle, minimum height=3em, minimum width=6em]
% \begin{tikzpicture}[remember picture,overlay]
%   \node [coordinate] at (current page.north) (accroche) {};
%   \node [block, below of=accroche] {Confidentiel Défense};
% \end{tikzpicture}}

%%% Local Variables: 
%%% mode: latex
%%% TeX-master: "../phdthesis"
%%% End: 