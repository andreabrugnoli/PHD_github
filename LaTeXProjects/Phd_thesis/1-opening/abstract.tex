\chapter*{Abstract}
\addstarredchapter{Abstract}
\chaptermark{Abstract}

This thesis aims at extending the port-Hamiltonian (pH) approach to continuum mechanics in higher geometrical dimensions (particularly in 2D). The pH formalism has a strong multiphysics character and represents a unified framework to model, analyze and control both finite- and infinite-dimensional systems. Despite the large literature on this topic, elasticity problems in higher geometrical dimensions have almost never been considered.  This work establishes the connection between port-Hamiltonian distributed systems and elasticity problems. The originality resides in three major contributions. First, the novel pH formulation of plate models and coupled thermoelastic phenomena is presented. The use of tensor calculus is mandatory for continuum mechanical models and the inclusion of tensor variables is necessary to obtain an intrinsic, i.e. coordinate free, and equivalent pH description. Second, a finite element based discretization technique, capable of preserving the structure of the infinite-dimensional problem at a discrete level, is developed and validated. The discretization of elasticity problems requires the use of non-standard finite elements because of the symmetric nature of the tensorial variable under consideration. Nevertheless, the numerical implementation is performed thanks to well-established open-source libraries, providing external users with an easy to use tool for simulating flexible systems in pH form. Third, flexible multibody systems are recast in pH form by making use of a floating frame description valid under small deformations assumptions. This reformulation include all kinds of linear elastic models and exploits the intrinsic modularity of pH systems.   \\\\