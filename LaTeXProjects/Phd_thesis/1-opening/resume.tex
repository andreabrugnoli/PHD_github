\chapter*{R\'esum\'e}
\addstarredchapter{R\'esum\'e}
\chaptermark{R\'esum\'e}

Cette thèse vise à étendre l'approche port-Hamiltonienne (pH) à la mécanique des milieux continus dans des dimensions géométriques plus élevées (en particulier on se focalise sur la dimension 2). Le formalisme pH, avec son fort caractère multi-physique, représente un cadre unifié pour modéliser, analyser et contrôler les systèmes de dimension finie et infinie. Malgré l'abondante littérature sur ce sujet, les problèmes d'élasticité en deux ou trois dimensions géométriques n'ont presque jamais été considérés. Dans ce travail de thèse la connexion entre problèmes d'élasticité et systèmes distribués port-Hamiltoniens est établie. L'originalité apportée réside dans trois contributions majeures. Tout d'abord, une nouvelle formulation pH des modèles de plaques et des phénomènes thermoélastiques couplés est présentée. L'utilisation du calcul tensoriel est obligatoire pour modéliser les milieux  continus et l'introduction de variables tensorielles est nécessaire pour obtenir une description pH équivalente qui soit intrinsèque, c'est-à-dire indépendante des coordonnées choisies. Deuxièmement, une technique de discrétisation basée sur les éléments finis et capable de préserver la structure du problème de la dimension infinie au niveau discret est développée et validée. Cette méthodologie repose sur une formule d'intégration par parties abstraite et peut être appliquée aux systèmes hyperboliques et paraboliques linéaires et non linéaires. Plusieurs éléments finis pour les structures minces (poutres et plaques) sont proposés et testés. La discrétisation des problèmes d'élasticité écrits en forme port-Hamiltonienne nécessite l'utilisation d'éléments finis non standards. Néanmoins, l'implémentation numérique est réalisée grâce à des bibliothèques open source bien établies, fournissant aux utilisateurs externes un outil facile à utiliser pour simuler des systèmes flexibles sous forme pH. Troisièmement, une nouvelle formulation pH de la dynamique multicorps flexible est dérivée. Cette reformulation, valable sous de petites hypothèses de déformations, inclut toutes sortes de modèles élastiques linéaires et exploite la modularité intrinsèque des systèmes pH.
