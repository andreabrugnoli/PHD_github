\chapter{Mathematical tools}
\label{app:math}

ARTICLE PAUL
Formal differential operator J is defined without boundary conditions (see e. g. [39], Sect. III.3). Formal skew-symmetry is verified by () ei under zero boundary conditions, where () is the inner product on the appropriate functional space.



\section{Differential operators}

The space of all, symmetric and skew-symmetric $d\times d$ matrices are denoted by $\mathbb{M}, \mathbb{S}, \mathbb{K}$ respectively. The space of $\mathbb{R}^d$ vectors is denoted by $\mathbb{V}$. $\Omega \subset \mathbb{R}^d$ is an open connected set. For a scalar field $u: \Omega \rightarrow \mathbb{R}$ the gradient is defined as 
\begin{equation*}
\mathrm{grad}(u) =  \nabla u := \begin{pmatrix}
\partial_{x_1} u \dots \partial_{x_d} u \\
\end{pmatrix}^\top.
\end{equation*}
For a vector field $\bm{u}: \Omega \rightarrow \mathbb{V}$, with components $u_j$, the gradient (Jacobian) is defined as
\begin{equation*}
\mathrm{grad}(\bm{u})_{i j}:= (\nabla \bm{u})_{ij} = \partial_{x_j} u_i.
\end{equation*}
The symmetric part of the gradient operator $\mathrm{Grad}$ (i. e. the deformation gradient in continuum mechanics) is thus given by
\begin{equation*}
\mathrm{Grad}(\bm{u}) := \frac{1}{2} \left(\nabla \bm{u} + \nabla^\top \bm{u} \right).
\end{equation*}
The Hessian operator of $u$ is then computed as follows
\begin{equation*}
\mathrm{Hess}(u) = \nabla^2 u = \mathrm{Grad}(\mathrm{grad}(u)),
\end{equation*}
For a tensor field $\bm{U}: \Omega \rightarrow \mathbb{M}$, with components $u_{ij}$, the divergence is a vector, defined column-wise as
\begin{equation*}
\mathrm{Div}(\bm U) = \nabla \cdot \bm{U} := \left( \sum_{i = 1}^d \partial_{x_i} u_{ij} \right)_{j = 1, \dots, d}.
\end{equation*}
The double divergence of a tensor field $\bm{U}$ is then a scalar field defined as
\begin{equation*}
\mathrm{div}(\mathrm{Div}(\bm U)):= \sum_{i, j = 1}^d \partial_{x_i} \partial_{x_j} u_{ij}.
\end{equation*}

