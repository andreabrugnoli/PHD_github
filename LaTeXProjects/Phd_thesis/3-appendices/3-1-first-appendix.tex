\chapter{Mathematical tools}
\label{app:math}



\section{Differential operators}

The space of all, symmetric and skew-symmetric $d\times d$ matrices are denoted by $\mathbb{M},\, \mathbb{S},\, \mathbb{K}$ respectively. The space of $\mathbb{R}^d$ vectors is denoted by $\mathbb{V}$. $\Omega \subset \mathbb{R}^d$ is an open connected set. For a scalar field $u: \Omega \rightarrow \mathbb{R}$ the gradient is defined as 
\begin{equation*}
\mathrm{grad}(u) =  \nabla u := \begin{pmatrix}
\partial_{x_1} u \dots \partial_{x_d} u \\
\end{pmatrix}^\top.
\end{equation*}
For a vector field $\bm{u}: \Omega \rightarrow \mathbb{V}$, with components $u_i$, the gradient (Jacobian) is defined as
\begin{equation*}
\mathrm{grad}(\bm{u})_{i j}:= (\nabla \bm{u})_{ij} = \partial_{x_j} u_i.
\end{equation*}
The symmetric part of the gradient operator $\mathrm{Grad}$ (i. e. the deformation gradient in continuum mechanics) is thus given by
\begin{equation*}
\mathrm{Grad}(\bm{u}) := \frac{1}{2} \left(\nabla \bm{u} + (\nabla\bm{u})^\top \right).
\end{equation*}
The Hessian operator of $u$ is then computed as follows
\begin{equation*}
\mathrm{Hess}(u) = \nabla^2 u = \mathrm{Grad}(\mathrm{grad}(u)),
\end{equation*}
For a tensor field $\bm{U}: \Omega \rightarrow \mathbb{M}$, with components $u_{ij}$, the divergence is a vector, defined column-wise as
\begin{equation*}
\mathrm{Div}(\bm U) = \nabla \cdot \bm{U} := \left( \sum_{i = 1}^d \partial_{x_i} u_{ij} \right)_{j = 1, \dots, d}.
\end{equation*}
The double divergence of a tensor field $\bm{U}$ is then a scalar field defined as
\begin{equation*}
\mathrm{div}(\mathrm{Div}(\bm U)):= \sum_{i, j = 1}^d \partial_{x_i} \partial_{x_j} u_{ij}.
\end{equation*}

\begin{definition}[Formal adjoint, Def. 5.80 \cite{rogers2004pde}]\label{def:foradj}
	Consider the differential operator defined on $\Omega$
	\begin{equation}
	\mathcal{L}(\bm{x}, \partial) =\sum_{|\alpha| \le k} a_\alpha(\bm{x})\partial^\alpha,
	\end{equation}
	where $\alpha := (\alpha_1, \dots , \alpha_d)$ is a multi-index of order $|\alpha| := \sum_{i=1}^d \alpha_i$, $a_\alpha$ are a set of  real scalars and $\partial^{\alpha} := \partial_{x_1}^{\alpha_1} \dots \partial_{x_d}^{\alpha_d}$ is a differential operator of order $|\alpha|$ resulting from a combination of spatial derivatives. The formal adjoint of $\mathcal{L}$ is the operator defined by
	\begin{equation}
	\mathcal{L}^*(\bm{x}, \partial)u = \sum_{|\alpha| \le k} (-1)^\alpha \partial^\alpha(a_\alpha(\bm{x}) u(\bm{x})).
	\end{equation}
\end{definition}
The importance of this definition lies in the fact that
\begin{equation}\label{eq:propadj}
\inner{\phi}{ \mathcal{L}(\bm{x}, \partial)\psi}{\Omega} = \inner{\mathcal{L}^*(\bm{x}, \partial)\phi}{\psi}{\Omega}
\end{equation}
for every $\phi, \psi \in C^\infty_0(\Omega)$. If the assumption of compact support is removed, then \eqref{eq:propadj} no longer holds; instead the integration by parts yields additional terms involving integrals over the boundary $\partial\Omega$. However, these boundary terms vanish if $\phi$ and $\psi$ satisfy certain restrictions on the boundary.

