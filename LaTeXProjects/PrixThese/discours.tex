\documentclass[16pt]{article}
\usepackage{geometry}
\geometry{
	a4paper,
	left=15mm,
	right=15mm,
	top=5mm,
	bottom=5mm,
}


\begin{document}

\Large
Bonsoir à tous. Je suis Andrea Brugnoli et je vais vous présenter ma thèse intitulé:
Une formulation port-Hamiltonienne des structure flexibles. Modélisation et discrétisation symplectique par éléments finis. Cette thèse, financé par l'ISAE-Supaero, à été dirigé par Daniel Alazard, Valérie Pommier Budinger (département DCAS) et Denis Matignon (DISC). 
(photo satellite) En aéronautique, l'utilisation des structures minces est essentiel pour limiter le poids: prenez par exemple un satellites de communication et ses panneaux solaires, \textbf{PAUSE} ou (photo projet Helios de la NASA) les ailes et le fuselage des avions. Un poids inférieur implique toujours une majeure flexibilité (vidéo glider) et dans certains conditions de charge cela peut amener à des instabilités. (dis. 1) Il est donc très important de disposer d'outils pour prédire le comportement des systèmes mécaniques complexes, \textbf{PAUSE} d'une manière à la fois intuitive et fidèle à la physique. Un outil de modélisation intuitif permet de analyser la complexité d'une façon modulaire, \textbf{PAUSE} en permettant de quantifier les performances selon différents configurations, choix de design et composants mécaniques. Un outil fidèle à la physique peut intégrer non seulement la mécanique, \textbf{PAUSE} mais aussi d'autre domaines d'applications comme par exemple la fluidodynamique ou l'electromagnetisme. (dis. 2) Dans le cadre de ma thèse j'ai étudié un formalisme mathématique qui permet d'analyser différents systèmes physiques d'une manière unifié et modulaire. En particulier, j'ai appliqué ce formalisme au cas des structure flexibles minces que on retrouve si souvent dans l'aeronautique. (dis. 3)  \textbf{PAUSE} L'aspect central à été le développement d'algorithmes numériques capables de conserver la physique du problème et basés sur un outil très utilisé en ingénierie, la méthode à éléments finis. Par méthodes à élément fins on indique les processus de mailler un domaine spatial intéressé par un phénomène physique à l'aide des petit éléments pour pouvoir résoudre les équations avec un calculateur. Pour que les techniques puissent être utilisé et vérifié par d'autres scientifiques, j'ai utilisé des librairies de calcul non propriétaire. \textbf{PAUSE} (vidéo Kirchh) Les algorithmes développés peuvent être utilisé pour la mécanique mais aussi pour décrire d'autres systèmes, comme (vidéo sw) les vagues dans les bassins d'eau, les onde électromagnétiques ou la thermoelasticité. (dis 4) Ce travail représente le point de départ pour plusieurs développements futures. Il devient possible de représenter d'une manière unifié des phénomènes multiphysiques comme l'interaction entre l'aerodynamique et la mécanique des ailes. Les modèles obtenus se prêtent à une interprétation physique directe et cela facilite la conception des lois de commande. Il est en suite possible de mettre en place des algorithme numériques pour réduire la cout computationel des modèles, \textbf{PAUSE} tout en conservant la structure physique sous-jacente, en obtenant des résultats plus précis et fiables que les méthodes qui ne cherche pas à a garantir la préservation de ces propriétés fondamentales. 
\\



disegno 1: aereo triste senza ala e contento con
disegno 2: persona pensa formule a caso, aero con ala a parte con mesh da interconnettere 
disegno 3: tizio davanti allo schermo con Daniel dietro che fa : je suis pas sure que l'ordi aie compris et apres (il font pas des miracles ces librairies open sources)
disegno 4: gufo senz'ali con le ali staccate
\end{document}