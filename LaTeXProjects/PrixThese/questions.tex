\documentclass[16pt]{article}
\usepackage{geometry}
\geometry{
	a4paper,
	left=15mm,
	right=15mm,
	top=5mm,
	bottom=5mm,
}


\begin{document}
	\Large
	\textit{Votre sujet de thèse est : "Modélisation et discrétisation
		symplectique des structures flexibles port-Hamiltoniennes". 
		Pourriez-vous en quelques mots nous expliquer votre problématique et dans quel cas d'application votre thèse pourrait être utilisée?
\\}
	
	
	Ma thèse a été centrée sur le développement d'algorithmes numériques pour la génération des modèles de structures flexibles minces. La nouveauté par rapport à l'existant réside dans le fait que ces algorithmes sont conçus pour garantir la préservation de propriétés physiques, comme la conservation d'énergie, et aussi pour implémenter des modèles modulaires. Grâce à ces outils il devient possible de concevoir des systèmes mécaniques à partir de leur composantes simples d'une manière à respecter les échanges d'énergie entre les différentes parties. 
	
	Les outils développés peuvent être utilisés dans l'industrie aérospatiale pour la conception ou le monitorage des satellites, en acoustique pour la simulation des instrument de musique ou en ingénierie civile pour l'analyse structurelle des bâtiments. Ces méthodes sont aussi utilisables pour la simulation des microsystèmes électromécaniques.
\\
	
	\textit{Quelle a été votre plus belle découverte lors de la rédaction de votre thèse (la découverte peut d’ailleurs être sur soi-même !) /OU/ Quel est votre meilleur souvenir ? \\}
	
	 Une thèse consiste à commencer un discours pour le terminer trois ans après. Il peut donc arriver que ce discours ne progresse pas d'une manière complètement logique et linéaire. Néanmoins en phase de rédaction, j'ai vu les différentes parties de mon travail prendre place dans un cadre cohérent, d'une manière claire et compréhensible. Cette réalisation m'a fait énormément plaisir car je comprends maintenant la valeur ajouté d'une thèse. Elle n'est pas dans la super spécialisation mais plutôt dans le développement de la pensée scientifique et de l'esprit critique dont notre société à extrêmement besoin. \\
	 
	\textit{Vous recevez aujourd'hui un prix vous félicitant pour la qualité de votre thèse, quels sont vos plans pour la suite? \\}
	          
	
	Je pense que travailler dans la recherche est un privilège immense, car les chercheurs sont payés pour étudier et approfondir leurs connaissances. Pour cette raison, après ma soutenance j'ai décidé de continuer comme chercheur post-doctoral dans le laboratoire de robotique à l'Université de Twente. Le projet dans lequel je travaille est très stimulant car il essaie de fondre des aspects théoriques fondamentaux avec l'ingénierie, l'objectif étant d'améliorer un drone bio-inspiré d'un faucon. 
	
	Grâce à cette expérience, j'ai réalisé à quel point j'aimerais monter mon projet de recherche. Pour cette raison, j'ai décidé de candidater au prix Loretta-Lopez, car il représente une possibilité unique de financement pour les jeunes chercheurs. Si mon projet n'est pas retenu, j'aimerais travailler dans un laboratoire académique ou industriel qui cherche à résoudre des problématiques qui touchent à la société et au développement humain.
\end{document}