\documentclass[pdftex,a4paper,12pt,origdate]{letter}

\usepackage[utf8]{inputenc}
%\usepackage[french]{babel}
\usepackage[T1]{fontenc}
\usepackage{ae,aecompl,aeguill}
\usepackage{tikz}
\usepackage{amsmath}
\usepackage[pdftex]{hyperref}

%======================================================================
% Début du document
%======================================================================
\begin{document}

%======================================================================
% Le nom pour la signature et le destinataire
%======================================================================
%\name{D. Matignon\\Resp. «~Mise à niveau hindi, ourdou et marwari~»}
\name{Professor Denis MATIGNON, \\ Head of Applied Mathematics  group, ISAE-SUPAERO.}

\begin{letter}{Editor of Applied Mathematical Modelling}

% University of California, Berkeley 
% Industrial Engineering \& Operations Research MS program

%======================================================================
% Adresse différente, pas de téléphone etc.
%======================================================================
%\address{}
%\telephone{}
%\nofax
%\email{}
%lieu{}
%\Nref{}
%\Vref{}
%\conc{}

\opening{Dear  Sir,}


Please find hereafter the responses to the reviews and revised versions of the submitted companion papers:\\


{\em Port-Hamiltonian formulation and symplectic discretization of plate models}.
\vspace{-5mm}
\begin{enumerate}
\item Part I: Mindlin model for thick plates.
\item Part II: Kirchhoff model for thin plates.
\end{enumerate}

We gratefully acknowledge the reviewers for their constructive comments and suggestions. \\

We adopted all the indications of reviewer 2. Indeed, his review helped us present the
results in a more coherent, rigorous and transparent manner. The redundant sections
of Part I have been removed, as suggested. The exposition now is clearer for the
reader. \\

For what concerns the synthetic comments of reviewer~1, we do understand the necessity of providing numerical evidence (comment n$^\circ$1), in support to the consistency
of new models. For this reason a totally new final section dedicated to the numerical
part has been added to both papers. The results obtained with our method prove
consistent with previously published material, assessing the validiy of the proposed
models. However, we were not able to satisfy request n$^\circ$2, asking for a complete
reshaping of the papers structure. This recommendation could not be followed, as we
believe that it would not significantly contribute to the improvement of the papers;
furthermore, it would strongly collide with the recommendations made by reviewer~1.
Reviewer~2 asks (question n$^\circ$3) about the computational efficiency of the proposed
discretization method: details upon the numerical implementation were given. Yet,
an accurate analysis of the finite element convergence property is more indicated for
publication in a numerical analysis journal and out of our main scope. Reviewer~1 is
also interested about possible insights about analytical solution given by the proposed
formulations (question n$^\circ$4). This is indeed the common topic of references suggested
by reviewer~1, but definitely not ours. The proposed models are quite complex and
require particular families of finite elements for the discretization. Even if the PH formulation for plate models provides no advantage in finding analytical solution, it is a
very powerful instrument in modelling complex multi-physics systems. This so-called
modularity feature is really appealing as industries now need to simulate complex
processes. \\

We do hope that the revised manuscripts will be considered for publication in Applied
Mathematical Modelling.


\closing{Yours Sincerely,}

%\encl{}

\end{letter}
\end{document}





\newpage

The authors are:
\begin{enumerate}
\item Denis MATIGNON, University of Toulouse, ISAE; 10, av. E. Belin; BP 54032. 31055 Toulouse Cedex~4, France. Phone:  00 33 5 61 33 81 12. \\ Email:~{\tt denis.matignon at isae.fr}
\item Thomas HELIE, IRCAM \& CNRS, UMR 9912,  1, pl. Igor Stravinsky, 75004 Paris, France. Phone:  00 33 1 44 78 48 24. \\ Email:~{\tt thomas.helie at ircam.fr}
\end{enumerate}








\iffalse

%\newpage

\vspace{15mm}

\hspace{45mm}
Professeur Denis MATIGNON, 

\hspace{45mm}
Responsable U.F. Math\'ematiques Appliqu\'ees, Supa\'ero / {\sc ISAE},

\hspace{45mm}
ancien Professeur Charg\'e de Cours au D\'ept. de  Math\'ematiques\\
\vspace{-1mm}
\hspace{45mm}
Appliqu\'ees de l'{\sc Ecole Polytechnique} (2004-2006).

\fi

\closing{Yours Sincerely,}

%\encl{}

\end{letter}
\end{document}
