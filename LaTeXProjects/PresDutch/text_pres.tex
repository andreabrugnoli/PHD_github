\documentclass[30pt]{article}

\usepackage{geometry}
\geometry{top=1cm,bottom=1cm,left=2cm,right=2cm}

\title{Presentatie}

\begin{document}
	\large


	Het is jammer dat we elkaar niet persoonlijk leren kennen. Met deze presentatie zal ik proberen mijzelf voor te stellen en de levenservaringen die mij naar Nederland hebben gebracht, te delen.  \\
	
	Ik ben geboren in Verona. Verona ligt in het noorden van Italië, ten westen van Venetië. Mijn geboortestad is een kleine maar mooie stad. Het is gebouwd rond de rivier de Adige.\\
	
	Het historische centrum is beschermd door de UNESCO en bevat een goed bewaarde Romeinse arena en vele middeleeuwse kerken. \\
	
	Mijn familie woont op het westelijke platteland, in Valpolicella. Dit deel van de stad is beroemd om zijn wijnproductie en elk jaar wordt in de stad een van de belangrijkste internationale wijntentoonstellingen gehouden, de Vinitaly.\\
	
	Vlakbij Valpolicella, ligt het grootste Italiaanse meer, het Gardameer. Veel Duitse en Nederlandse toeristen komen elk jaar naar het Gardameer om te kamperen en te genieten van de prachtige landschappen en oude dorpjes rond het meer. Ik ging eens met een vriend naar een club aan het meer en toen ontdekten we dat er alleen maar Nederlandse toeristen waren. \\
	
	Na de middelbare school ben ik voor mijn studie naar Milaan gegaan. Milaan is echt anders dan Verona.  Het is de op één na dichtstbevolkte Italiaanse stad en een van de dichtstbevolkte metropolen van Europa. Het is ook het belangrijkste economische centrum van Italië. \\
	
	Tijdens mijn masterstudie had ik de mogelijkheid om in het buitenland te studeren en ik besloot om naar Toulouse te gaan. Toulouse ligt in het zuiden van Frankrijk, op 100 kilometer afstand van de Pyreneeën en tussen de Middellandse Zee en de Atlantische Oceaan. \\
	
	Als je van fietsen houdt (zoals veel Nederlanders doen), dan kun je zonder stoppen langs het kanaal fietsen van Toulouse tot aan de Middellandse Zee. \\
	
	Ik heb erg genoten van Toulouse. Het weer is zonnig en er zijn veel evenementen. Elk jaar worden er 56 festivals georganiseerd in Toulouse. De stad biedt een bruisend nachtleven voor jongeren: er is 's avonds altijd wat te doen. Om deze redenen besloot ik in Toulouse te blijven voor mijn doctoraalstudie. \\
	
	Tijdens mijn PHD heb ik vier maanden in Brazilië doorgebracht. Ik verbleef in São José dos Campos, op een uur rijden van Sao Paulo. Brazilië is heel anders dan Europa. De stedelijke infrastructuur is echt chaotisch. In Sao José loopt de snelweg door het centrum van de stad en daardoor is het soms moeilijk om je door de stad te verplaatsen. \\
	
	Brazilië staat bekend om het carnaval. Historisch gezien bestond het carnaval uit groepen die door de lanen van de stad paradeerden en optraden op instrumenten en dansten. Tegenwoordig staan ze bekend als carnaval "blocos". Overal in de steden zijn er veel groepen mensen die dansen en het is echt heel leuk. \\
	
	Brazilië heeft een ecosysteem dat meer aanwezig is in de steden zelf. Op de foto's zie je de tropische bossen en heuvels rond rio de Janeiro en het tropische strand van Trinidade. \\
	
	Ik doe nu een post-doc in Enschede. Ik hou van de manier van leven hier: de stad is klein maar alle voorzieningen zijn dichtbij. De mensen zijn erg vriendelijk en bereid om te helpen. \\
	
	Het is nu tijd om de presentatie af te sluiten. Ik heb erg genoten van deze cursus en ik ga waarschijnlijk in september verder.     
	

\end{document}