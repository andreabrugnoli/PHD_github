\documentclass{article}
\usepackage[top=1.5cm, bottom=1.5cm, left=3cm, right=3cm]{geometry}
\usepackage{comment}

\title{A port-Hamiltonian formulation of flexible structures: modeling and symplectic finite element discretization \\ \vspace{3pt}
\Large Thesis résumé}

\author{Andrea Brugnoli}
\begin{document}
\maketitle
\section*{English version}
This thesis aims at extending the port-Hamiltonian (pH) approach to continuum mechanics in higher geometrical dimensions (particularly in 2D). The pH formalism has a strong multiphysics character and represents a unified framework to model, analyze and control both finite- and infinite-dimensional systems. Despite the large literature on this topic, elasticity problems in higher geometrical dimensions have almost never been considered.  This work establishes the connection between port-Hamiltonian distributed systems and elasticity problems. The originality resides in three major contributions. First, the novel pH formulation of plate models and coupled thermoelastic phenomena is presented. The use of tensor calculus is mandatory for continuum mechanical models and the inclusion of tensor variables is necessary to obtain an intrinsic, i.e. coordinate free, and equivalent pH description. Second, a finite element based discretization technique, capable of preserving the structure of the infinite-dimensional problem at a discrete level, is developed and validated. The discretization of elasticity problems requires the use of non-standard finite elements because of the symmetric nature of the tensorial variable under consideration. Nevertheless, the numerical implementation is performed thanks to well-established open-source libraries, providing external users with an easy to use tool for simulating flexible systems in pH form. Third, flexible multibody systems are recast in pH form by making use of a floating frame description valid under small deformations assumptions. This reformulation include all kinds of linear elastic models and exploits the intrinsic modularity of pH systems.   \\\\

\section*{Version française}

Cette thèse vise à étendre l'approche port-hamiltonienne (pH) à la mécanique des milieux continus dans des dimensions géométriques plus élevées (en particulier on se focalise sur la dimension deux). Le formalisme pH, avec son fort caractère multiphysique, représente un cadre unifié pour modéliser, analyser et contrôler les systèmes de dimension finie et infinie. Malgré l'abondante littérature sur ce sujet, les problèmes d'élasticité en deux ou trois dimensions géométriques n'ont presque jamais été considérés. Dans ces travaux de thèse la connexion entre problèmes d'élasticité et systèmes distribués port-Hamiltoniens est établie. L'originalité apportée réside dans trois contributions majeures. Tout d'abord, la nouvelle formulation pH des modèles de plaques et des phénomènes thermoélastiques couplés est présentée. L'utilisation du calcul tensoriel est obligatoire pour modéliser les milieux  continus et l'introduction de variables tensorielles est nécessaire pour obtenir une description pH équivalente qui soit intrinsèque, c'est-à-dire indépendante des coordonnées choisies. Deuxièmement, une technique de discrétisation basée sur les éléments finis et capable de préserver la structure du problème de la dimension infinie au niveau discret est développée et validée. La discrétisation des problèmes d'élasticité nécessite l'utilisation d'éléments finis non standard à cause de tenseurs symétriques qui apparaissent dans la formulation pH. Néanmoins, l'implémentation numérique est réalisée grâce à des bibliothèques open source bien établies, fournissant aux utilisateurs externes un outil facile à utiliser pour simuler des systèmes flexibles sous forme pH. Troisièmement, une nouvelle formulation pH de la dynamique multicorps flexible est dérivée. Cette reformulation, valable sous de petites hypothèses de déformations, inclut toutes sortes de modèles élastiques linéaires et exploite la modularité intrinsèque des systèmes pH.

\end{document}