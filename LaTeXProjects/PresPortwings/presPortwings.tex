% WORKAROUND (an update of the geometry package broke beamer...)
\makeatletter\let\ifGm@compatii\relax\makeatother
% end of WORKAROUND


\documentclass{beamer}
% \usetheme[uttitlepage=false]{ut}
\usetheme[euler=false,titlepage=B]{ut}
%\usetheme[titlepage=C,debug]{ut}
\usetheme{ut}

\usepackage{tikz-cd}
\usepackage[most]{tcolorbox}

\usepackage[backend=bibtex, style=authoryear, doi=false,isbn=false,url=false]{biblatex}

\usepackage{bm}
\usepackage{color}
\definecolor{theme}{RGB}{0,73,114}


\usepackage{graphicx}
\usepackage{diffcoeff}
\usepackage[most]{tcolorbox}
\usepackage{mathtools}

% Math macros
\DeclareMathOperator*{\grad}{grad}
\DeclareMathOperator*{\Grad}{Grad}
\DeclareMathOperator*{\Div}{Div}
\renewcommand{\div}{\operatorname{div}}
\DeclareMathOperator*{\Hess}{Hess}
\DeclareMathOperator*{\curl}{curl}
\DeclareMathOperator{\Tr}{Tr}
\DeclareMathOperator{\Dom}{Dom}
\DeclareMathOperator*{\esssup}{ess\,sup}

\newcommand{\bbR}{\mathbb{R}}
\newcommand{\bbF}{\mathbb{F}}
\newcommand{\bbA}{\mathbb{A}}
\newcommand{\bbB}{\mathbb{B}}
\newcommand{\bbS}{\mathbb{S}}

\newcommand*{\norm}[1]{\ensuremath{\left\|#1\right\|}}
\newcommand{\where}{\qquad \text{where} \qquad}
\newcommand{\inner}[3][]{\ensuremath{\left\langle #2, \, #3 \right\rangle_{#1}}}
\newcommand{\bilprod}[2]{\left\langle \left\langle \, #1, #2 \, \right\rangle \right\rangle}
\newcommand{\pder}[2]{\ensuremath{\partial_{#2} #1}}
\newcommand{\dder}[2]{\ensuremath{\delta_{#2} #1}}
\newcommand{\secref}[1]{\S\ref{#1}}
\newcommand{\energy}[1]{\frac{1}{2} \int_{\Omega} \left\{ #1 \right\} \d\Omega}
\newcommand{\crmat}[1]{\ensuremath{\left[#1\right]_\times}}
\newcommand{\fenics}{\textsc{FEniCS}\xspace}
\newcommand{\firedrake}{\textsc{Firedrake}\xspace}

\DeclareMathOperator*{\argmax}{arg\,max}
\DeclareMathOperator*{\argmin}{arg\,min}

\newtheorem{proposition}{Proposition}
\newtheorem{remark}{Remark}
\newtheorem{hypothesis}{Hypothesis}
\newtheorem{assumption}{Assumption}
\newtheorem{conjecture}{Conjecture}


\def\onedot{$\mathsurround0pt\ldotp$}
\def\cddot{% two dots stacked vertically
	\mathbin{\vcenter{\baselineskip.67ex
			\hbox{\onedot}\hbox{\onedot}}%
}}

\renewcommand\bibfont{\scriptsize}


\makeatletter \renewcommand\d[1]{\ensuremath{%
		\;\mathrm{d}#1\@ifnextchar\d{\!}{}}}
\makeatother


\graphicspath{{./images/}}

\bibliography{biblioPortwings}


\title{Portwings Internal Meeting\\
Challenges and outlook for the numerics}
% \subtitle[Short subtitle]{I am not using any subtitles}
\author[A.~Brugnoli]{Andrea Brugnoli} %  
\institute[UT]{Department of Robotics and Mechatronics,\\ University of Twente}
\date[29-04-2021]{April 29, 2021}
\footlinetext{[A.~Brugnoli]}
% \titlegraphic{\includegraphics[height=1cm]{titlegraphic}}

% \utbeamerset{tpboxbx=10}

\begin{document}

\maketitle

\begin{frame}{Overview}
	\tableofcontents[currentsubsection, sectionstyle=show/show, subsectionstyle=show/shaded/show]
\end{frame}

\section{What numerics for portwings?}



\begin{frame}\frametitle{What numerics for the portwings project?}
Methods should preserve the continuous structure at the discrete level. Which structure? 
\begin{enumerate}
	\item Cohomology: $V^0(\bbR) \xrightarrow{\nabla} V^1(\bbR^3) \xrightarrow{\nabla \times} V^2(\bbR^3) \xrightarrow{\nabla \cdot} V^3(\bbR)$;
	\item Variational structure $\delta \int l = 0$, ($l$ Lagrangian density);
	\item Hamiltonian structure $\dot{\mathcal{F}} = \{F, H\}$, $\{\cdot, \cdot\}$ Poisson brackets.
	\item \dots
\end{enumerate}
\vspace{.1cm}
Recent developments:
\begin{itemize}
\item splitting of topological and metric operators (\cite{bauer2018split});
\item Lie group structure and underlying variational formulation (\cite{gawlik2020variational});
\item connection with algebraic topology, i.e. de Rham complex and more general Hilbert complexes, e.g. elasticity (\cite{bochev2006mimetic,arnold2006acta,palha2014compatible}); 
\end{itemize}

\end{frame}

\subsection{Split discretization}

\begin{frame}[fragile]{Principle behind split discretization}
\begin{itemize}
	\item Fluid equation written in covariant form (exterior calculus);
	\item Split Hamiltonian form $\dot{\mathcal{F}}= \{\mathcal{F}, \mathcal{H}\}$.	
	\begin{itemize}
		\item Topological braket depending on $d$ (exterior derivative) or $\iota_v$ (interior product).
		\item Metric dependent $\mathcal{H}$, since it depends on $*$
	\end{itemize}
\end{itemize}	
\begin{block}{Linear shallow water waves in Hamiltonian form}
	\begin{itemize}
		\item $\mathcal{H} = \frac{1}{2} \int_M\{\bar{h} \norm{\bm{u}}^2 + g h^2\}\d{x}$, with $\diffd{\mathcal{H}}{\bm{u}} = \bar{h}\bm{u}, \; \diffd{\mathcal{H}}{h}=g h$, where $g$ gravity acc. and $\bar{h}$ equilibrium fluid height.
		\item $\{\mathcal{F}, \mathcal{G}\} = -(\diffd{\mathcal{F}}{\bm{u}}, \nabla \diffd{\mathcal{G}}{h})_{L^2} - (\diffd{\mathcal{F}}{h}, \nabla \cdot \diffd{\mathcal{G}}{\bm{u}})_{L^2}$.
		\begin{equation*}
			\begin{aligned}
				\mathcal{F} &= \int \bm{u} \d{\Omega}: \; \dot{\mathcal{F}} = - \left(\diffd{\mathcal{F}}{\bm{u}}, \nabla \diffd{\mathcal{H}}{h}\right) \rightarrow \partial_t \bm{u} = -g \nabla h. \\
				\mathcal{F} &= \int h \d{\Omega}: \;\dot{\mathcal{F}} = - \left(\diffd{\mathcal{F}}{h}, \nabla \cdot \diffd{\mathcal{H}}{\bm{u}}\right) \rightarrow \partial_t h = -\bar{h} \nabla \cdot \bm{u}.
			\end{aligned}
		\end{equation*}
		
	\end{itemize}
\end{block}

\end{frame}

\begin{frame}[fragile]\frametitle{Split and weak (or mixed) form}
	De Rham complex: $V^0(\bbR) \xrightarrow{\nabla} V^1(\bbR^3) \xrightarrow{\nabla \times} V^2(\bbR^3) \xrightarrow{\nabla \cdot} V^3(\bbR).$
	\addtolength{\textfloatsep}{-0.8in}
	
	\begin{tcbraster}[raster columns=2, raster equal height]
		\begin{tcolorbox}[width=0.5\textwidth, nobeforeafter, colframe=theme,title=Split form]%%
				\begin{tikzcd}
					h \in V^0 \arrow{r}{\nabla}\arrow{d}[swap]{\tilde{h} = \tilde{*} h} & V^1 \ni \bm{u} \arrow{d}{\bm{u} = \tilde{*} \bm{u}} \\
					\tilde{h} \in V^3  & \arrow{l}[swap]{\nabla \cdot} V^2 \ni \tilde{\bm{u}} \\
				\end{tikzcd}
		\begin{itemize}
			\item Assume full $\tilde{*}$ in metric eqs.
			\item Both $\nabla, \nabla \cdot$ are imposed strongly.
			\item Both diff. eqs exact 
		\end{itemize}
		\end{tcolorbox} 
		\begin{tcolorbox}[width=0.5\textwidth, nobeforeafter,  colframe=theme,title=Weak form (Mixed FE)]%%
			\begin{tikzcd}
				h \in V^3 \arrow{r}{\widehat{\nabla}}\arrow{d}[swap]{\tilde{h} = h} & V^2 \ni \bm{u} \arrow{d}{\tilde{\bm{u}} = \bm{u}} \\
				\tilde{h} \in V^3  & \arrow{l}[swap]{\nabla \cdot} V^2 \ni \tilde{\bm{u}} \\
			\end{tikzcd}
		\begin{itemize}
			\item Assume $\tilde{*}=\text{Id}$, i.e. $\tilde{\bm{u}}=\bm{u}, \, \tilde{h}=h$.
			\item Weak gradient $\widehat{\nabla}$.
			\item Moment Eq. weak
		\end{itemize}
		\end{tcolorbox}
	\end{tcbraster}
	
\end{frame}

\begin{frame}\frametitle{Split decomposition in practice}
\begin{enumerate}
\item First projection of the strong form.  For the 1D case:
\begin{equation*}
	\begin{aligned}
		\partial_t \mathbf{u}_e^{(1)} + g \mathbf{D}^{en} \mathbf{h}_n^{(0)} = \mathbf{0}, \qquad \mathbf{u}_e^{(1)}: \text{1 form}, \quad \mathbf{h}_n^{(0)}: \text{0 form}, \\
		\partial_t \widetilde{\mathbf{h}}_e^{(1)} + \bar{h} \mathbf{D}^{en} \widetilde{\mathbf{u}}_n^{(0)} = \mathbf{0}, \qquad \widetilde{\mathbf{h}}_e^{(1)}: \text{1 form}, \quad \widetilde{\mathbf{u}}_n^{(0)}: \text{0 form}.
	\end{aligned}
\end{equation*}
$\mathbf{D}^{en}$ metric free approximation of exterior derivative. \vspace{.3cm}
\item Project the metric closure relations:
\begin{itemize}
	\item High accuracy $CG_1^u-CG_1^h$ spaces: 
	$$\mathbf{M}^{nn} \widetilde{\mathbf{u}}_n^{(0)}=\mathbf{P}^{ne}\mathbf{u}_e^{(1)}, \qquad \mathbf{M}^{nn} \mathbf{h}_n^{(0)}=\mathbf{P}^{ne}\widetilde{\mathbf{h}}_e^{(1)};$$
	\item Low accuracy $DG_0^u-DG_0^h$ spaces: 
	$$\mathbf{M}^{en} \widetilde{\mathbf{u}}_n^{(0)}=\mathbf{I}^{ee}\mathbf{u}_e^{(1)}, \qquad \mathbf{M}^{en} \mathbf{h}_n^{(0)}=\mathbf{I}^{ee}\widetilde{\mathbf{h}}_e^{(1)};$$
	\item Medium accuracy $CG_1^u-DG_0^h$ spaces: 
	$$\mathbf{M}^{nn} \widetilde{\mathbf{u}}_n^{(0)}=\mathbf{P}^{ne}\mathbf{u}_e^{(1)}, \qquad \mathbf{M}^{en} \mathbf{h}_n^{(0)}=\mathbf{I}^{ee}\widetilde{\mathbf{h}}_e^{(1)};$$
\end{itemize}
$\mathbf{M}^{nn}, \; \mathbf{M}^{en}$ metric dependent, $\mathbf{P}^{ne}$ metric free averaging op.	
\end{enumerate}

\end{frame}


\begin{frame}{Weak (mixed) form}
Weak formulation: find $\bm{u} \in V^2(\bbR^3), \; h \in V^3(\bbR)$
\begin{equation*}
	\begin{aligned}
(\bm{v}_u, \partial_t \bm{u})_{L^2} &= -(\bm{v}_u, g \widehat{\nabla} h)_{L^2} = (\nabla \cdot \bm{v}_u, g h)_{L^2}, \\
(v_h, \partial_t h)_{L^2} &= -(v_h, \bar{h} \nabla \cdot \bm{u})_{L^2},
\end{aligned} \qquad 
\begin{aligned}
	&\forall\bm{v}_u \in V^2, \\ &\forall v_h \in V^3.
\end{aligned}
\end{equation*}

After projection
\begin{equation*}
\begin{bmatrix}
	\mathbf{M}^{nn} & \mathbf{0} \\
	\mathbf{0} & \mathbf{M}^{ee}
\end{bmatrix}
\diffp{}{t}
\begin{pmatrix}
	\mathbf{u}_n \\ \mathbf{h}_e
\end{pmatrix} = 
\begin{bmatrix}
	\mathbf{0} & g \mathbf{D}^{ne} \\
	 -\bar{h}\mathbf{D}^{en} & \mathbf{0}
\end{bmatrix}
\begin{pmatrix}
	\mathbf{u}_n \\ \mathbf{h}_e
\end{pmatrix}, \qquad \mathbf{D}^{ne} = \mathbf{D}^{en\, \top}.
\end{equation*}
\vspace{.3cm}\\
Another formulation with weak divergence: find $\bm{u} \in V^1(\bbR^3), \; h \in V^0(\bbR)$
\begin{equation*}
	\begin{aligned}
		(\bm{v}_u, \partial_t \bm{u})_{L^2} &= -(\bm{v}_u, g \nabla h)_{L^2} , \\
		(v_h, \partial_t h)_{L^2} &= -(v_h, \bar{h}\widehat{\nabla \cdot} \bm{u})_{L^2} = (\nabla v_h,  \bar{h}\bm{u})_{L^2},
	\end{aligned} \qquad 
	\begin{aligned}
		&\forall\bm{v}_u \in V^1, \\ &\forall v_h \in V^0.
	\end{aligned}
\end{equation*} 
\end{frame}

\subsection{Algebraic topology: from continuous to discrete}

\begin{frame}[fragile]\frametitle{Connection with algebraic topology (n=2)}
		\begin{tikzcd}[row sep=small, column sep=small]
			& C^0(D) \arrow[bend right=25]{dl}[', near end, xshift=5pt, yshift=-2pt]{\mathcal{I}^0}\arrow{rr}{\delta}\arrow[leftrightarrow]{dd}[yshift=-10pt]{*_h} & & C^1(D)\arrow{rr}{\delta} \arrow[bend right=15]{dl}[', near end,xshift=5pt, yshift=-2pt]{\mathcal{I}^1}\arrow[leftrightarrow]{dd}[yshift=-10pt]{*_h} & & C^2(D)\arrow[leftrightarrow]{dd}[yshift=-10pt]{*_h} \arrow[bend right=25]{dl}[', near end, xshift=5pt, yshift=-2pt]{\mathcal{I}^2} \\
			\Lambda^0(M) \arrow[bend right=15]{ur}[xshift=5pt]{\mathcal{R}^0} \arrow[crossing over]{rr}[xshift=10pt]{d}\arrow[leftrightarrow]{dd}[yshift=3pt]{*} & & \Lambda^1(M)\arrow[bend right=25]{ru}[xshift=5pt]{\mathcal{R}^1}\arrow{rr}[xshift=10pt]{d} & & \Lambda^2(M)\arrow[bend right=15]{ur}[xshift=5pt]{\mathcal{R}^2} \\
			& \widetilde{C}^2(D) \arrow[bend right=25]{dl}[', near end, xshift=5pt, yshift=-3pt]{\widetilde{\mathcal{I}}^2} &  & \widetilde{C}^1(D) \arrow{ll}[xshift=-15pt]{\delta} \arrow[bend right=25]{dl}[', near end, xshift=5pt, yshift=-3pt]{\widetilde{\mathcal{I}}^1} & & \widetilde{C}^0(D) \arrow{ll}[xshift=-15pt]{\delta} \arrow[bend right=25]{dl}[', near end, xshift=5pt, yshift=-3pt]{\widetilde{\mathcal{I}}^0} \\
			\widetilde{\Lambda}^2(M)\arrow[bend right=15]{ur}[xshift=5pt]{\widetilde{\mathcal{R}}^2}   & & \widetilde{\Lambda}^1(M)\arrow{ll}{d}\arrow[crossing over, leftrightarrow]{uu}[yshift=3pt]{*}\arrow[bend right=15]{ur}[xshift=5pt]{\widetilde{\mathcal{R}}^1} & & \widetilde{\Lambda}^0(M) \arrow{ll}{d}\arrow[crossing over, leftrightarrow]{uu}[yshift=3pt]{*}\arrow[bend right=15]{ur}[xshift=5pt]{\widetilde{\mathcal{R}}^0}\\
		\end{tikzcd}
	Diagram taken from \cite{palha2014compatible}.
	\begin{itemize}
		\item $C^k(D)$ space of cochains on the primal grid $D$;
		\item $\widetilde{C}^k$ space of cochains on the dual grid $\widetilde{D}$;
		\item $\delta$ coboundary operator (dual of $\partial$ on chains);
		\item $\mathcal{I}^k, \widetilde{\mathcal{I}}^k$ interpolation operators;
		\item $\mathcal{R}^k, \widetilde{\mathcal{R}}^k$ reduction operators;
	\end{itemize}
\end{frame}

\begin{frame}{Connection with Finite Elements and Finite Volumes}
	How to discretize the Hodge Laplacian in  in physics-compatible fashion?
	The exterior derivative can be discretize exactly on one grid, but not the codifferential $d^* = (-1)^{n(k+1)+1}* d *$. \\
	
	
	Two different solutions:
	\begin{itemize}
		\item Dual grid discretization: the Hodge star is treated numerically by considering a dual grid and leads to the explicit construction of the Hodge matrix (analogously to finite volumes);
		\item Single grid discretization: the Hodge operator is incorporated implicitly by using an inner product. The codifferential is converted into an exterior derivative by using the integration by parts (equivalent to mixed finite element method).
	\end{itemize}
\end{frame}

\section{What about mechanics?}

\begin{frame}\frametitle{Continuum mechanics and differential geometry}
	In \cite{arnold2006acta} the authors present linear elasticity using vector-valued forms and its associated complex. \\
	\vspace{.5cm}
	
	 The wing can be modeled as a thin shell structure.  For thin structures (Euler-Bernoulli beam and Kirchhoff plate) the metric enters the interconnection operator.\\
	\vspace{.5cm}
	Open question for Kirchhoff plate: $\Hess \cong  \nabla \circ d$, where $\nabla$ covariant derivative? 

\end{frame}


\begin{frame}\frametitle{Tools available}
	Firedrake: \url{https://www.firedrakeproject.org/}. \\
	\vspace{.1cm}
	
	FEniCS: \url{https://fenicsproject.org/}. \\
	\vspace{.1cm}
	
	Fluid Structure Interaction in Fenics: \cite{bergersen2020fenics}. \\
	\vspace{.1cm}
	Mesh morphing in FEniCS: \url{https://bitbucket.org/Epoxid/femorph/src/c7317791c8f00d70fe16d593344cb164a53cad9b/?at=dokken\%2Frestructuring} \\
	\vspace{.1cm}	
	
	PyDec: \url{https://github.com/hirani/pydec} \\
	\vspace{.1cm}
	
	Learning Python for scientific computing \url{https://faculty.math.illinois.edu/~hirani/cbmg/index.html}
\end{frame}


\section{Outlook for elasticity models}







\end{document}

