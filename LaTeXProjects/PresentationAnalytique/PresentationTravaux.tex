\documentclass[french]{article}
\usepackage{graphicx}
\usepackage{caption}
\usepackage[T1]{fontenc}
\usepackage[utf8]{inputenc}
\usepackage{lmodern}
\usepackage{geometry}
\geometry{
	a4paper,
	left=20mm,
	right=20mm,
	top=30mm,
	bottom=30mm,
}
\usepackage{babel}




\usepackage[unicode=true,pdfusetitle,bookmarks=true,bookmarksnumbered=false,bookmarksopen=false,
breaklinks=false,pdfborder={0 0 0},backref=false,colorlinks=true,urlcolor=blue]{hyperref}

\graphicspath{{images/}}

\author{Andrea Brugnoli \\ 
	\hspace{2.8pt} Docteur ISAE-Supaéro 2020}
\title{Présentation analytique}

\date{}

\begin{document}
	
	\maketitle
	
Dans ce document vous pourriez trouver les résumés de mes travaux plus significatifs.

\paragraph{Thèse de doctorat: Une formulation port-Hamiltonienne des structures flexibles. Modélisation et discrétisation symplectique par éléments finis \cite{brugnoli2020phd}\\} 

Cette thèse vise à étendre l’approche port-Hamiltonienne (pH) à la mécanique des milieux continus dans des dimensions géométriques plus élevées (en particulier on se focalise sur la dimension 2). Le formalisme pH, avec son fort caractère multi-physique, représente un cadre unifié pour modéliser, analyser et contrôler les systèmes de dimension finie et infinie. Malgré l’abondante littérature sur ce sujet, les problèmes d’élasticité en deux ou trois dimensions géométriques n’ont presque jamais été considérés. Dans ce travail de thèse la connexion entre problèmes d’élasticité et systèmes distribués port-Hamiltoniens est établie. L’originalité apportée réside dans trois contributions majeures. Tout d’abord, une nouvelle formulation pH des modèles de plaques et des phénomènes thermoélastiques couplés est présentée. L’utilisation du calcul tensoriel est obligatoire pour modéliser les milieux continus et l’introduction de variables tensorielles est nécessaire pour obtenir une description pH équivalente qui soit intrinsèque, c’est-à-dire indépendante des coordonnées choisies. Deuxièmement, une technique de discrétisation basée sur les éléments finis mixtes et capable de préserver la structure du problème de la dimension infinie au niveau discret est développée et validée. Cette méthodologie repose sur une formule d’intégration par parties abstraite et peut être appliquée aux systèmes hyperboliques et paraboliques linéaires et non linéaires. Plusieurs éléments finis pour les structures minces (poutres et plaques) sont proposés et testés. L’implémentation numérique est réalisée grâce à des bibliothèques open source bien établies, fournissant aux utilisateurs externes un outil facile à utiliser pour simuler des systèmes flexibles sous forme pH. Troisièmement, une nouvelle formulation pH de la dynamique multicorps flexible est dérivée. Cette reformulation, valable sous de petites hypothèses de déformations, inclut toutes sortes de modèles élastiques linéaires
et exploite la modularité intrinsèque des systèmes pH.

\paragraph{Article de revue: Formulation port-hamiltonienne et discrétisation symplectique des modèles de plaques. Partie I : Modèle de Mindlin pour plaques épaisses \cite{brugnoli2019ammmin} \\}

Dans cet article, la formulation port-Hamiltonienne sous forme tensorielle d'une plaque épaisse décrite par le modèle de Mindlin-Reissner est présentée. Le contrôle et l'observation frontière sont pris en compte. Grâce au calcul tensoriel, on peut s'apercevoir que le modèle de plaque de Mindlin imite la structure d'interconnexion de son homologue unidimensionnel, c'est-à-dire le poutre de Timoshenko.

La méthode des éléments finis mixtes est ensuite étendue aux formulations vectorielles et tensorielles afin d'obtenir un système port-Hamiltonien (PHs) de dimension finie approprié, i.e. qui préserve la structure et les propriétés de l'original. système de paramètres distribués. Les conditions aux limites mixtes sont finalement traitées en introduisant des contraintes algébriques. Des exemples numériques sont enfin présentés pour valider cette approche.

\paragraph{Article de revue: Formulation Port-Hamiltonienne et discrétisation symplectique de modèles de plaques. Partie II : Modèle de Kirchhoff pour plaques minces \cite{brugnoli2019ammkir} \\}

Le modèle mécanique d'une plaque mince avec contrôle et observation frontière est présenté comme un système port-Hamiltonien, à la fois sous forme vectorielle et tensorielle : le modèle Kirchhoff-Love pour les plaques fines est décrit en utilisant une structure de Stokes-Dirac et ce représente une nouveauté par rapport à la littérature existante. Cette formulation est réalisée à la fois sous des formes vectorielles et tensorielles. Grâce au calcul tensoriel, ce modèle mime la structure d'interconnexion de son homologue unidimensionnel, c'est-à-dire la poutre d'Euler-Bernoulli.

La méthode des éléments finis mixtes est ensuite utilisée pour obtenir une forme faible appropriée, c'est-à-dire préservant la structure. La procédure de discrétisation, effectuée sur la formulation tensorielle, conduit à un système port-hamiltonien de dimension finie. Cette partie II prolonge la partie I, consacrée au modèle Mindlin pour les plaques épaisses. Le modèle de Kirchhoff-Love s'accompagne de difficultés supplémentaires, en raison de l'ordre supérieur de l'opérateur différentiel considéré.


\paragraph{Article de revue: Dynamique multicorps flexible port-Hamiltonienne \cite{brugnoli2020msd} \\}

Une nouvelle formulation pour la construction modulaire de systèmes multicorps flexibles est présenté. En réarrangeant les équations d'un corps flottant flexible et en introduisant les moments canoniques appropriées, le modèle est formulé comme un système couplé de équations aux dérivées partielles sous forme port-Hamiltonienne (pH). Cette approche repose sur l'approche "floating frame of référence" et est valable sous l'hypothèse de petites déformations. Ceci permet de traiter y compris les modèles mécaniques qui ne peuvent pas être facilement formulés en termes de formes différentielles.

Une méthode basée sur les éléments finis est ensuite introduite pour discrétiser la dynamique d'une manière structurée. Grâce a la modularité de l'approche port-Hamiltonien, des systèmes multicorps complexes peuvent être construits de manière modulaire. Les contraintes sont imposées au niveau de la vitesse. Par conséquent un système quasi-linéaire différentiel-algébrique d'indice 2 est obtenu. Des tests numériques sont effectués pour évaluer la validité de l'approche proposée.

\paragraph{Article de revue: Une formulation port-Hamiltonienne de la thermoélasticité linéaire et sa discrétisation par éléments finis mixtes \cite{brugnoli2021ther} \\}

Une formulation port-Hamiltonienne pour la thermoélasticité linéaire couplée et pour la flexion thermoélastique de structures minces est présentée. La construction exploite la modularité intrinsèque des systèmes port-Hamiltoniens pour obtenir une formulation de la thermoélasticité linéaire en tant qu'interconnexion des équations de l'élastodynamique et de la chaleur. Le modèle dérivé peut être facilement discrétisé en utilisant des éléments finis mixtes. La discrétisation préserve la structure, puisque les principales caractéristiques du système sont conservées à un niveau discret. Le modèle et la stratégie de discrétisation proposés sont validés par rapport à un problème de référence de thermoélasticité, le problème de Danilovskaya.

\paragraph{Actes de congrès: Interconnexion de la plaque Kirchhoff dans le cadre port-Hamiltonien \cite{brugnoli2019cdc} \\}

Le modèle de plaque de Kirchhoff est détaillé en utilisant une formulation tensorielle port-hamiltonienne. Une discrétisation préservant la structure de ce modèle est alors réalisée en utilisant des l'élément finis mixtes. Cette méthodologie permet de considérer facilement le contrôle et observation frontière. Le système de dimension finie peut être interconnecté à l'environnement de manière simple et structurée. Les contraintes algébriques à à considérer se déduisent des conditions aux limites, qui peuvent être homogènes ou défini par un un autre système dynamique. La polyvalence de l'approche proposée est évaluée au moyen de
de simulations numériques. Une première illustration considère une plaque rectangulaire serrée sur un côté et interconnectée à une barre rigide tige soudée sur le côté opposé. Un deuxième exemple exploite la passivité des systèmes de pH pour injecter de l'amortissement dans une plaque subissant un forçage externe. 


\paragraph{Actes de congrès: Discrétisation symplectique des modèles de plaques port-Hamiltoniennes~\cite{brugnoli2020mtns} \\}

Les méthodes de discrétisation des systèmes port-hamiltoniens sont intéressantes à la fois pour la simulation et le contrôle. Malgré l'abondante littérature sur les éléments finis mixtes, aucune analyse rigoureuse des connexions entre éléments mixtes et systèmes port-Hamiltoniens n'a été réalisée. Dans cet article, nous démontrons comment les méthodes existantes peuvent être utilisées pour discrétiser les problèmes de plaques dynamiques de manière à préserver la structure physique. Sur la base des résultats de convergence des schémas existants, de nouvelles estimations d'erreurs sont conjecturées ; les simulations numériques confirment les comportements attendus.

\paragraph{Article de revue: Décoder et réaliser le vol battu avec la théorie port-Hamiltonien \cite{califano2021} \\}

Dans cet article, nous envisageons comment aborder un problème particulièrement difficile qui présente des caractéristiques hautement interdisciplinaires, allant de la biologie à l'ingénierie : la description dynamique et la réalisation technologique du vol battu. Ce document explique pourquoi, afin d'acquérir de nouvelles connaissances sur ce sujet, nous avons choisi d'utiliser la théorie port-Hamiltonienne. Nous discutons de la façon dont le caractère physiquement unificateur du cadre est capable de décrire la dynamique de battement dans tous ses aspects importants. Les défis technologiques et théoriques du vol en battement sont discutés en considérant l'interaction entre différents sujets. Tout d'abord, la conceptualisation formelle du problème est analysée. Deuxièmement, les caractéristiques et les capacités du cadre port-Hamiltonien en tant que langage mathématique sous-jacent sont présentées. Par la suite, la discrétisation du modèle résultant au moyen de stratégies de préservation de la structure est abordée. Une fois qu'un modèle numérique fiable est disponible, nous discutons de la façon dont les actions de contrôle peuvent être calculées sur la base de spécifications de haut niveau visant à augmenter les performances de vol. Dans la dernière partie, les outils technologiques nécessaires pour valider expérimentalement les modèles et pour équiper un prototype d'oiseau robotisé des dispositifs de détection et d'actionnement nécessaires sont discutés.

\bibliographystyle{unsrt}
\bibliography{biblio}
	
\end{document}
