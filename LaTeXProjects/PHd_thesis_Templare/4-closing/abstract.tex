\begin{vcenterpage}

%\noindent\rule[2pt]{\textwidth}{0.5pt}
\thispagestyle{empty}
{\large\textbf{Résumé ---}}
  L’imagerie satellitaire aide à la surveillance des terrains agricoles, à la cartographie,  à la sylviculture et à la surveillance des eaux. Dans des situations de crise, l’imagerie peut contribuer à la conduite d’opérations de sauvetage  et à l’atténuation des effets des catastrophes naturelles. 
Les missions d’observation de la Terre actuelles requièrent de hautes performances en pointage fin qui justifient les nombreuses recherches sur l’atténuation des  micro-vibrations et les systèmes de stabilisation de la ligne de visée. Les micro-vibrations sont une classe particulière de vibrations structurelles de petite amplitude (micrométrique) et avec un large contenu spectral. Elles se propagent depuis les sources aux éléments flexibles du satellite et provoquent la dégradation des images collectées et la perte d’une partie de l’information vitale. Dans les plateformes satellitaires, elles sont causées par différentes sources. Parmi les plus importantes se trouvent les roues à réaction utilisées pour le contrôle d’attitude et les moteurs d’entrainement des panneaux solaires. 
Ce travail de thèse présente une méthodologie complète pour la modélisation  d’un satellite flexible avec ses sources principales de micro-vibrations et la synthèse d’une nouvelle stratégie de contrôle actif pour réduire la dégradation de la ligne de visée.   
Un banc d’essai expérimental développé à l’Agence Spatiale Européenne a permis de valider l'architecture de contrôle proposée.

{\large\textbf{Mots clés :}}
    Micro-vibrations, dynamique flexible, contrôle r\^{o}buste, fusion senseurs.
\\
\noindent\rule[2pt]{\textwidth}{0.5pt}


%\newpage
%\thispagestyle{empty}
%\noindent\rule[2pt]{\textwidth}{0.5pt}
%\begin{center}
%{\large\textbf{Title in english\\}}
%\end{center}
{\large\textbf{Abstract ---}}  
Satellite imagery provides contributions to land surveying, cartography, forestry and water management. In crisis management scenarios imagery can help to carry out rescue operations and to mitigate the results of natural catastrophes.
Modern Space observation and Science missions demand stringent pointing requirements that motivate a significant amount of research on the topic of microvibration mitigation and line-of-sight stabilization systems. Microvibrations are a particular class of structural vibrations with small amplitude (micrometer and sub-micrometer) and wide frequency content. They propagate from the sources to the flexible elements of the spacecraft and cause the degradation of the collected images and the loss of a part of the vital information. On spacecraft they are generated by different sources. Among them the most important sources are the reaction wheels used for attitude control and the solar array drive mechanisms used to orient rotating solar panels.
The present PhD thesis aims to present an end-to-end methodology to model a flexible spacecraft with its main sources of microvibrations and to implement a novel active control strategy to mitigate the line-of-sight degradation. 
An experimental test bench developed at the European Space Agency has enabled the validation of the proposed control architecture.  

{\large\textbf{Keywords:}}
    Microvibrations, flexible dynamics, robust control, hybrid sensing.
\\
\noindent\rule[2pt]{\textwidth}{0.5pt}
\begin{center}
  Commande des Sytèmes et Dynamique du Vol (CSDV) - \'Equipe d'accueil ISAE-ONERA\\
 10, Avenue \'Edouard Belin\\
 31400 Toulouse
\end{center}
\end{vcenterpage}

%%% Local Variables: 
%%% mode: latex
%%% TeX-master: "../phdthesis"
%%% End: 
