\chapter*{Ringraziamenti}
\addstarredchapter{Ringraziamenti}
\chaptermark{Ringraziamenti}

Quando si arriva a scrivere i ringraziamenti vuol dire che, forse, il peggio \`e passato ed \`e ora di riconciliarsi con tutti. \`E il momento del perdono e della riconoscenza. Tre anni densi come il resto della vita sono passati. Si ha l'impressione di essere stati investiti da una navicella spaziale a tutta velocit\`a, ma di essere miracolosamente sopravvissuti. Ci si ritrova ad un tratto nello Spazio profondo, dopo un'attivit\`a extraveicolare andata male, con la fune spezzata come un cordone ombelicale, nel vuoto della disoccupazione. 

Hai tempo per riflettere, ora, guardando la Terra che si allontana con tutti i suoi abitanti. Ripensi ai primi giorni del tuo dottorato, quando le porte dell'ufficio si sono aperte fagocitandoti.
Una nuova famiglia che ti accoglie, i dottorandi (Flavio L.C.R., Pierrick R., Antoine S., Marie L., Vincent L., Nabil K., Anass S., Luca M. \textit{et al.}), donne e uomini di estremo coraggio o di stravagante incoscienza, con cui ridere sornioni delle comuni sventure e con cui condividere i successi. Durante le pause pomeridiane, sorseggiando insieme il caff\'e delle macchinette, avvolti dal fumo rassicurante delle sigarette di Laurent e Mich\`ele, ho perfezionato il francese e scoperto che in alcune regioni di Francia si pu\`o essere "soli tutti e due".

E ora che nuoto nel vuoto dello Spazio, mi guardo intorno per cercare l'angolo in cui \`e nascosto il paradiso, per fare un saluto al mio caro nonno 'Ntino, che appoggiato sul manico di una zappa mi fa da lontano un occhiolino, come per dirmi: "Apri gli occhi vagn\`o!". 

Dallo Spazio si vede un numero infinito di stelle, ce ne sono talmente tante che si perde l'orientamento. Su Terra le stelle sono poche, ma sono servite a imboccare la giusta rotta in un mare a volte tempestoso. Di esse ne ricordo il nome. Una costellazione, chiamata "Famiglia maggiore", \`e stata sempre la prima da individuare, cos\`i lontana ma cos\`i luminosa da poterne ascoltare la calda voce al telefono. L\`i dove gli ulivi crescono rigogliosi in un'arida terra, queste stelle mi hanno insegnato la tenacia nel credere di poter fare grandi cose con poche gocce d'acqua, con l'umilt\`a e il silenzio d'un albero che cresce. 

Quando fui recrutato per il programma spaziale i capitani erano Val\'erie P.B., Daniel A. e Fabrice B.. Mi mostrarono il libretto d'istruzioni della navicella e mi aiutarono a correggere la rotta quando fui al timone. Mi trasmisero le loro conoscenze e la loro esperienza in campo tecnico, ma avemmo tempo anche di discutere delle nostre vite su Terra. 

Durante il viaggio intergalattico atterrai per qualche mese su un pianeta, chiamato "The bubble", dove incontrai Diego N.T., Giordana B., Ozgun Y. e Marina M.. Con loro trascorsi momenti indimenticabili ed una forza speciale si istaur\'o tenendoci connessi dopo esserci separati. 

Talvolta tornavo nella terra degli ulivi per reincontrare gli amici dell'infanzia: Gabriele R., Daniele L., Cinzia D., Paolo R., Valentina G., Marco Z. e Paola P.. Parlavamo di come il mondo cambiava intorno a noi, mentre cercavamo di usare le vecchi leggi per interpretarlo. Siamo difatti cambiati anche noi col mondo e ognuno ha fatto la sua rivoluzione Copernicana. Nonostante ci\'o continuiamo a riconoscerci anche se con molti meno capelli.  

Sul "pianeta degli artisti" incontrai un gruppo di deliranti creature, Andrea G., Andrea B., Paolo P., Andrea A. e Mina L.. Con loro ci capimmo dal primo istante senza pronunciare parola. Poi ne pronunciammo e furono sempre straordinarie.

Una coppia di vecchi amici, Marco S. e Geraldina G., aprirono la porta della loro casa tutte le volte che avevo bisogno di un ristoro o di un sorriso.

Condivisi il dormitorio durante l'addestramento con Simone U. e Aurora B.. Col primo condividemmo la stessa amara sorte di cosmonauta. La seconda, con serafica calma, sopport\`o tutto il mio disordine.

Imparai a scalare le montagne sul pianeta in cui non ci sono montagne, l'Olanda, con Alberto M., Karine Z., Silvia Z. e Chiara M.. Sullo stesso pianeta incontrai Robert A., Martina M. e Levin G.. Con tutti loro condividemmo momenti di indimenticabile ilarit\`a. Nello stesso periodo soggiornava sullo stesso pianeta un vecchio cosmonauta, che avevo gi\`a incontrato in passato, Valentin P.. Mi spieg\`o i segreti dell'iperspazio e come avrei potuto sopravvivere al viaggio intergalattico.  

Come tutti i cult spaziali holliwodiani c'\`e anche la storia d'amore, l'incontro fatidico con Soizic G., la breto-parigina, che sconvolse la vita dell'eroe distraendolo dall'obiettivo della missione, ma rendendo gradevole il viaggio. 
