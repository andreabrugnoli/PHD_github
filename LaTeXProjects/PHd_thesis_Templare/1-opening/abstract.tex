\chapter*{Abstract}
\addstarredchapter{Abstract}
\chaptermark{Abstract}

 Satellite imagery provides contributions to land surveying, cartography, forestry and water management. In crisis management scenarios imagery can help to carry out rescue operations and to mitigate the results of natural catastrophes.
Modern Space observation missions demand stringent pointing requirements that motivated a significant amount of research on the topic of microvibration isolation and line-of-sight stabilization systems. Microvibrations are a particular class of structural vibrations of small amplitude (micrometer) and with a large spectral content. They propagate from the sources to the flexible elements of the spacecraft and they cause the degradation of the gathered images and the loss of part of the vital information. There are many sources of micro-vibrations in the spacecraft. Reaction wheels are generally the main ones. Also other components, such as solar array drive mechanisms, antenna stepper motors, antenna trimming mechanisms or chemical thrusters, can have a huge impact.
In order to have better pointing performances, the disturbances can be reduced by mounting some of the noisy equipment on various isolation platforms. However these solutions result ineffective at low frequency. Residual vibrations can still propagate through the spacecraft and be amplified by the flexible structure of the satellite. The present PhD thesis relies on an architecture based on an actuator that directly acts at payload level and on an additional rate sensor integrated in the line-of-sight. The conventional LOS control architectures use low frequency sensors. This work proposes to integrate an additional rate sensor on the most sensitive structural elements in the optical path to raise the bandwidth of the active control system.
The first contribution relates to the flexible structure modeling. The multi-body modeling methodology called “Two-Input Two-Output Port (TITOP) approach” is extended to the case of flexible structures with N ports (N-Input N-Output Port: NINOP). This methodology is applied to Space systems to develop models of deployable solar arrays, reaction wheels and solar array drive mechanisms.  The versatility of this framework allows the co-design of a structure/control problem by simple connection of elementary models associated to each sub-structure. All possible parametric uncertainties can be easily taken into account in order to perform robust performance analysis with the modern structured singular value techniques. These models permit the analysis of the pointing performances under the effect of microvibrations and their interactions with the gyroscopic effects of the reaction wheels and the spacecraft structural modes that vary with the solar panels rotation.
The second contribution of this work is about the control laws synthesis. Following a comprehensive model identification and uncertainty quantification part, the robust control strategy is designed in the structured-$\mathcal{H}_{\infty}$ framework to account for plant uncertainty and provide formal worst-case performance guarantees.  
The last contribution is experimental. The comparison of the theoretical predictions and the experimental results obtained on the test bench developed at the European Space Agency permits the validation of the proposed approach. 
The results of this thesis, both on the modeling and experimental aspect, are generalized to be scaled to different observation missions. 
