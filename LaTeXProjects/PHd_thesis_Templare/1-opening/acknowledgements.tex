\chapter*{Aknowledgements}
\addstarredchapter{Aknowledgements}
\chaptermark{Aknowledgements}

When you come to write the acknowledgements it means that maybe the worst is over and it is time to reconcile with everyone. It is the moment of forgiveness and gratitude. Three dense years have passed. You have the feeling of having been hit by a spaceship at full speed and of having miraculously survived. You suddenly find yourself in deep Space after a failed extravehicular activity with the rope broken, like an umbilical cord in the vacuum of unemployment. 

You have time to think whilst looking at the Earth that moves away with all its inhabitants. You think back to your first days of your PhD, when the doors of your office opened and phagocytized you.

A new family welcomed you: the PhD candidates (Flavio L.C.R., Pierrick R., Antoine S., Marie L., Vincent L., Nabil K., Anass S., Luca M. \textit{et al.}). These are women and men known for their extreme bravery or the extravagant recklessness with which you can laugh at your common misfortunes or share successes. During afternoon breaks, shrouded in the reassuring smoke of Laurent and Michèle’s cigarettes as we sipped coffee together, I improved my French and discovered that in some regions of France you can be “alone with someone else”.

And now that I am swimming in the emptiness of Space, I look around to find the corner where paradise is hidden and say hello to my dear grandfather ‘Ntino. Leaning on the handle of a hoe, he winks at me from a distance as if to to tell me: “Open your eyes, man!”

From Space you can see an infinite number of stars. There are so many that you lose direction. On Earth stars are few, but sometimes they help you to navigate in a stormy sea. I remember their names. A constellation, called "Family Major", was always the first one to be identified. It was so distant but so bright that you could hear its warm voice on the phone. Where the olive trees flourish in arid land, these stars have taught me the tenacity to believe that I can make great things with a few drops of water, with the humility and silence of a growing tree.

When I was recruited for the Space program the captains were Valérie P.B., Daniel A. and Fabrice B.. They showed me the instruction manual for the spaceship and helped me to correct the course when I was at the helm. They provided me with their technical knowledge and experience, but we also had time to discuss our lives on Earth. 

During our intergalactic journey I landed for a few months on a planet called "the Bubble", where I met Diego N.T., Giordana B., Ozgun Y. and Marina M.. With them I spent unforgettable moments and a special force developed and kept us connected after we were separated.

Sometimes I returned to the land of olive trees to meet my childhood friends: Gabriele R., Daniele L., Cinzia D., Paolo R., Valentina G., Marco Z. and Paola P.. We talked about how the world was changing around us and how we tried to use the old laws to interpret it. Indeed, we too have changed with the world and everyone has made his own Copernican Revolution. Nevertheless we continue to recognize ourselves, although with much less hair.

On the "Planet of the Artists" I met a group of delirious creatures, Andrea G., Andrea B., Paolo P., Andrea A. and Mina L.. We understood each other from the first moment without saying a word. Then we uttered some words and they were always extraordinary.

A couple of old friends, Marco S. and Geraldina G., opened the front door of their home whenever I needed refreshment or a smile.

I shared my dormitory during training with Simone U. and Aurora B.. With the former we shared the same hard fate of astronaut. The latter, with seraphic calm, bore all my untidiness.

I learned to climb mountains on the planet where there are no mountains, The Netherlands, with Alberto M., Karine Z., Silvia Z. and Chiara M.. On the same planet I met Robert A., Martina M. and Levin G.. With all of them I shared moments of unforgettable hilarity. At the same time an old astronaut whom I had already met in the past, Valentin P., was staying on the same planet. He explained to me all the secrets of hyperspace and how I could survive the intergalactic journey.

Like all Hollywood cult space films, there is also a love story, the fateful encounter with Soizic G., the Breton-Parisian girl, who disrupted the hero's life by distracting him from the mission's goal while making his journey all the greater for it.
