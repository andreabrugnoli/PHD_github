\chapter*{Remerciements}
\addstarredchapter{Remerciements}
\chaptermark{Remerciements}

Quand on arrive \`a \'ecrire les remerciements, cela veut dire que, peut-\^etre, le pire est pass\'e et c'est le moment de se r\'econcilier avec tout le monde. C'est le moment du pardon et de la reconnaissance.  Trois ann\'ees denses comme le reste de la vie ont pass\'e. On a l'impression d'avoir \'et\'e percut\'es par un vaisseau spatial \`a toute vitesse mais d'avoir surv\'ecu miraculeusement. On se retrouve tout d'un coup dans l'Espace profond, apr\`es une activit\'e extrav\'ehiculaire qui s'est mal pass\'ee, avec la corde tranch\'ee comme un cordon ombilical, dans le vide du ch\^omage.

T'as le temps pour r\'efl\'echir maintenant en regardant la Terre qui s'\'eloigne avec tous ses habitants. Tu repenses aux premiers jours de ton doctorat, quand les portes du bureau se sont ouvertes en t'engloutissant. Une nouvelle famille qui t'accueille, les doctorants (Flavio L.C.R., Pierrick R., Antoine S., Marie L., Vincent L., Nabil K., Anass S., Luca M. \textit{et al.}), femmes et hommes de courage extr\^eme ou d'extravagante inconscience, avec qui rire sournoisement des m\'esaventures communes et avec qui partager les succ\'es. Pendant les pauses de l'apr\'es-midi, en sirotant le caf\'e des machines, entour\'es par la fum\'ee rassurante des cigarettes de Laurent et Mich\`ele, j'ai perfectionn\'e mon français et d\'ecouvert qu'en certaines r\'egions de France on peut \^etre «seuls tous les deux».

Et maintenant que je nage dans le vide de l'Espace, je regarde autour de moi pour trouver le coin o\`u se cache le paradis, pour saluer mon cher grand-p\`ere 'Ntino, qui pench\'e sur le manche d'une houe me fait un clin d'œil de loin, comme pour dire : «Ouvre les yeux petit!».

Depuis l'Espace on voit un nombre infini d'\'etoiles, il y en a tellement qu'on perd l'orientation. Sur la Terre il y a peu d'\'etoiles, mais elles m'ont servi \`a prendre la bonne route dans une mer parfois orageuse. Je m'en rappelle les noms. Une constellation, appel\'ee « Grande famille», a \'et\'e toujours la premi\`ere que j'identifiai, si distante mais si lumineuse qu'on en peut entendre la voix chaude au t\'el\'ephone. L\`a o\`u les oliviers poussent luxuriants dans une terre aride, ces \'etoiles m'ont appris la t\'enacit\'e de croire qu'on peut faire des choses grandes avec quelques gouttes d'eau, avec l'humilit\'e et le silence d'un arbre qui pousse. 

Quand je fus recrut\'e pour le programme spatial les capitaines \'etaient Val\'erie P.B., Daniel A. et Fabrice B.. Ils me montr\`erent le livret d'instructions du vaisseau et m'aid\`erent \`a corriger la route quand je fus au timon. Ils me transmirent leurs connaissances et leur exp\'erience dans le domaine technique, mais nous e\^umes aussi le temps de discuter de nos vies sur Terre. 

Pendant le voyage intergalactique j'atterris pour quelque mois sur une plan\`ete appel\'ee «The bubble», o\^u je rencontrai Diego N.T., Giordana B., Ozgun Y. et Marina M.. Avec eux je passai des moments inoubliables et une force sp\'eciale s'\'etablit en nous laissant connect\'es apr\`es notre s\'eparation. 

Parfois je rentrais \`a la terre des oliviers pour retrouver les amis d'enfance : Gabriele R., Daniele L., Cinzia D., Paolo R., Valentina G., Marco Z. et Paola P.. On parlait du monde qui changeait autour de nous, alors qu'on cherchait \`a utiliser les vieilles lois pour l'interpr\'eter. En fait nous avons aussi chang\'e avec le monde et chacun a fait sa r\'evolution Copernicienne. Malgr\'e tout on continue encore \`a se reconnaitre m\^eme avec beaucoup moins de cheveux. 
  
Sur la «plan\`ete des artistes» je rencontrai un groupe de cr\'eatures d\'elirantes, Andrea G., Andrea B., Paolo P., Andrea A. et Mina L.. Avec eux on se comprit depuis le premier instant sans prononcer un mot. Apr\`es on en pronon\c{c}a et ils furent toujours extraordinaires.

Une couple de vieux amis, Marco S. et Geraldina G., ouvrirent la porte de leur maison toutes les fois que j'avais besoin d'un rafraichissement ou d'un sourire.

Je partageai le dortoir pendant la formation avec Simone U. et Aurora B.. Avec le premier on partagea le m\^eme sort amer de cosmonaute. La deuxi\`eme, avec un calme s\'eraphique, tol\'era tout mon d\'esordre.   

J’appris \`a escalader les montagnes sur la plan\`ete o\`u il n'y a pas de montagnes, les Pays-Bas, avec Alberto M., Karine Z., Silvia Z. et Chiara M.. Sur la m\^eme plan\`ete je rencontrai Robert A., Martina M. et Levin G.. Avec eux on partagea des moments d'inoubliable hilarit\'e. A cette p\'eriode vivait sur la m\^eme plan\`ete un vieux cosmonaute, que j'avais d\'ej\`a rencontr\'e dans le pass\'e, Valentin P.. Il m'expliqua les secrets de l'hyperespace et comment survivre au voyage intergalactique. 

Comme dans tous les films cultes hollywoodiens sur l'Espace il y a aussi l'histoire d'amour, la rencontre fatidique avec Soizic G., la  breto-parisienne, qui bouleversa la vie du h\'eros en le distrayant de l'objectif de sa mission, mais en rendant agr\'eable le voyage. 
