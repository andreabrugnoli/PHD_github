\chapter*{R\'esum\'e}
\addstarredchapter{R\'esum\'e}
\chaptermark{R\'esum\'e}

 L’imagerie satellitaire aide à la surveillance des terrains agricoles, à la cartographie,  à la sylviculture et à la surveillance des eaux. Dans des situations de crise, l’imagerie peut contribuer à la conduite d’opérations de sauvetage et à l’atténuation des effets des catastrophes naturelles. 
Les missions modernes d’observation de la Terre actuelles requièrent de hautes performances en pointage fin qui justifient les nombreuses recherches sur l’atténuation des  micro-vibrations et les systèmes de stabilisation de la ligne de visée. Les micro-vibrations sont une classe particulière de vibrations structurelles de petite amplitude (micrométrique) et avec un large contenu spectral. Elles se propagent depuis les sources aux éléments flexibles du satellite et provoquent la dégradation des images collectées et la perte d’une partie de l’information vitale. Dans les plateformes satellites, elles sont causées par différentes sources. Les roues à inertie sont généralement la source principale des micro-vibrations.  D’autres composants comme les moteurs d’entrainement des panneaux solaires et des antennes ou les tuyères chimiques peuvent avoir un impact non négligeable.
Pour augmenter la précision de pointage, les perturbations peuvent être réduites de manière passive en isolant les sources de vibrations, mais ces solutions sont inefficaces en basse fréquence. Des micro-vibrations résiduelles peuvent se propager dans le satellite et être amplifiées par la flexibilité des divers éléments de structures du satellite.  Ce travail de thèse s’appuie sur une architecture basée sur un actionneur qui agit directement au niveau de la charge utile et sur un capteur intégré dans la ligne de visée.  Les architectures conventionnelles de contrôle de ligne de visée utilisent des senseurs basse fréquence. Cette thèse propose d’intégrer un senseur de vitesse additionnel sur les éléments structurels les plus sensibles dans le chemin optique afin d’augmenter la bande passante du système actif de contrôle. 
La première contribution concerne la modélisation de structures flexibles. La méthodologie de modélisation multi-corps appelée « approche Two-Input Two-Output Port (TITOP) » est étendue aux cas des structures flexibles avec N ports (N-Input N-Output Port: NINOP). Cette méthodologie est appliquée aux systèmes spatiaux, notamment pour développer un modèle des panneaux solaires déployables, des roues à inertie et des moteurs d’entrainement des panneaux solaires. La versatilité de cette approche permet le co-design de problèmes structure/contrôle à travers l’assemblage de modèles élémentaires associés à chaque substructure. Toutes les incertitudes paramétriques peuvent être prises en compte pour mener une analyse de performance robuste avec les techniques modernes fondées sur la valeur singulière structurée. Ces modèles permettent l’analyse des performances de pointage sous l’effet des micro-vibrations et de leurs interactions avec les effets gyroscopiques des roues à réaction et les modes structuraux du satellite variant selon la rotation des panneaux solaires.
La deuxième contribution de cette thèse concerne la synthèse des lois de commande. Après une partie dédiée à la modélisation et l’identification des sous-systèmes de l’architecture de contrôle et à la quantification des incertitudes, une stratégie de contrôle robuste est présentée dans le cadre $\mathcal{H}_{\infty}$-structuré pour tenir en compte des incertitudes du système et fournir une preuve formelle des performances garanties dans le pire-cas.  
La dernière contribution est expérimentale. La confrontation des  prédictions théoriques avec les résultats expérimentaux obtenus sur le banc d’essai développé à l’Agence Spatiale Européenne permet de  valider l'approche proposée.  
Les résultats de cette thèse, tant sur le plan des modèles et des matériels expérimentaux, sont suffisamment génériques pour être transposés à diverses missions d'observation de la terre.
