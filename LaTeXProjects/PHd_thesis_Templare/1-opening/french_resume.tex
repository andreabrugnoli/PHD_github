\setcounter{chapter}{18}% Equivalent to "letter O"
\renewcommand{\thechapter}{\Alph{chapter}}%

\chapter*{R\'esum\'e en fran\c{c}ais}
\addstarredchapter{R\'esum\'e en fran\c{c}ais}
\chaptermark{R\'esum\'e en fran\c{c}ais}

\minitoc

\section{Contexte et motivations}

L'humanit\'e a b\'en\'efici\'e depuis longtemps de l'imagerie spatiale pour ses d\'ecouvertes scientifiques. Par exemple: les d\'ecouvertes apport\'ees par le \textit{Hubble Space Telescope} (HST) de la NASA depuis son lancement en 1990 jusqu'\`a aujourd'hui comme le raffinement de la constante d'Hubble \cite{FBM01}, qui mesure la vitesse d'expansion de l'univers. Le HST sera remplac\'e par le \textit{James Webb Telescope} (JWST) (Fig. \ref{fig:JWSTfr}) en 2021 pour l'\'etude de la formation des \'etoiles et des galaxies et pour prendre des images d'exoplan\`etes et de novas. La mission \textit{Euclid} (Fig. \ref{fig:euclidfr}) de l'agence spatiale europ\'eenne (ESA), qui sera lanc\'ee en 2021, mesurera avec son t\'elescope Korsch \cite{K77} la forme des galaxies \`a diff\'erentes distances de la Terre pour mieux comprendre l'\'energie noire et la mati\`ere noire. 

\begin{figure}[!h]%
    \centering
    \subfloat[]{{\includegraphics[width=0.45\columnwidth]{images/french_resume/JWST.pdf} }\label{fig:JWSTfr}}%
    \qquad
    \subfloat[]{\includegraphics[width=0.45\columnwidth]{images/french_resume/Euclid.pdf}\label{fig:euclidfr}}%
    \caption{Deux missions astronomiques de nouvelle g\'en\'eration : (a) James Webb Space Telescope. (b) Euclid.}%
\end{figure}
 
Une autre raison de l'importance de l'imagerie spatiale est importante est l'observation de la Terre. Les informations r\'ecolt\'ees par les satellites sont utilis\'ees pour la surveillance des terrains agricoles (Fig. \ref{fig:agricoltura_spain_fr}), la cartographie, la gestion des for\^ets et des eaux. En cas de gestion de crises, les images peuvent devenir indispensables pour faciliter les op\'erations de sauvetage et pour neutraliser ou att\'enuer les d\'eg\^ats de catastrophes naturelles.

\begin{figure}[!h]
	\centering
	\includegraphics[width=0.9\linewidth]{images/french_resume/agricoltura_spain.pdf}
	\caption{Surveillance des terrains agricoles en Espagne: les images du satellite Sentinel-2A de f\'evrier \`a octobre 2016 montrent le changement du paysage dans le parc naturel Brazo de Este et autour de la ville de Los Palacios y Villafranca. Copyright: contains modified Copernicus Sentinel data (2016), modifié par ESA, CC BY-SA 3.0 IGO.}
	\label{fig:agricoltura_spain_fr}
\end{figure}

Les performances en termes de pr\'ecision d'image sont devenues de plus en plus exigeantes en atteignant le niveau des nano radiants pour les missions d'observation de l'avenir. 
Les micro-vibrations ont \'et\'e identifi\'ees comme la cause principale de la d\'egradation des images. Les micro-vibrations, comme d\'efinies par le Spacecraft Mechanical Loads Analysis Handbook \cite{CRS13}, sont des vibrations de basse amplitude sur une large bande passante (jusqu'\`a plusieurs centaines d'Hz) produites par diff\'erents syst\`emes en mouvement sur le satellite. Elles peuvent \^etre amplifi\'ees par la transmissibilit\'e mécanique depuis la source jusqu'\`a la charge utile de la mission. Cette amplification est due \`a la flexibilit\'e des structures, li\'ee \`a la tendance \`a concevoir des structures en m\^eme temps plus l\'eg\`eres et plus larges pour les futures missions spatiales. 

Les micro-vibrations peuvent corrompre les images de deux façons : les perturbations de basse fr\'equence causent des distorsions g\'eom\'etriques comme montr\'e dans la Fig. \ref{fig:distortion_fr}, alors que celles haute fr\'equence provoquent des images floues. Les d\'efauts en basse fr\'equence peuvent être corrig\'es dans certains cas sur Terre en post-traitement avec des algorithmes d'ortho-rectification \cite{LBAA07,MOF15}. Par contre on a la perte totale de l'information dans le cas de micro-vibrations en haute fr\'equence et basse amplitude. La limite entre basse et haute fr\'equence est impos\'ee par le temps d'int\'egration de la charge utile. Cette fr\'equence correspond grossi\`erement \`a 0.1 fois l'inverse du temps d'int\'egration.
   
\begin{figure}[!h]%
    \centering
    \subfloat[]{{\includegraphics[width=0.35\columnwidth]{images/french_resume/distortion1.pdf} }\label{fig:distortion1_fr}}%
    \qquad
    \subfloat[]{\includegraphics[width=0.5\columnwidth]{images/french_resume/distortion2.pdf}\label{fig:distortion2_fr}}%
    \caption{(a) Erreur de pointage sur le plan focal de la charge utile. (b) Effet de l'erreur de pointage sur la qualit\'e de l'image (gauche), avec correction en post-traitement (centre) et image de référence sans vibrations (droite) \cite{CRS13}.}%
    \label{fig:distortion_fr}%
\end{figure}

Les micro-vibrations sont g\'en\'er\'ees principalement \`a l'int\'erieur du satellite. Elles peuvent être classées en deux cat\'egories selon à leur nature fr\'equentielle : p\'eriodiques (harmoniques) ou transitoires. Les micro-vibrations caus\'ees par les roues \`a r\'eaction \cite{miller2007reaction}, les gyros, par les moteurs d'entrainement des panneaux solaires \cite{man1990} (si actionn\'es en continue) et par les refroidisseurs cryog\'eniques \cite{kim2011} font partie de la première catégorie. La deuxi\`eme cat\'egorie est constitu\'ee par les micro-vibrations produites par les moteurs d'entrainement des panneaux solaires (si actionn\'es pendant des phases particuli\`eres de la mission), par les m\'ecanismes de pointage antenne \cite{arnon1997}, par les m\'ecanismes de basculement des miroirs et par les micro-tuy\`eres. 

Les plus grands contributeurs aux micro-vibrations sont les m\'ecanismes tournants. Les micro-vibrations sont alors provoqu\'ees par les balourds statiques et dynamiques des roues, par les imperfections des roulements \`a billes et/ou par les imperfections des moteurs. Ces perturbations d\'ependent strictement de la vitesse de rotation des roues et les balourds statiques et dynamiques jouent le r\^ole le plus important dans la génération des micro-vibrations. 
Si on prend en compte la Fig. \ref{fig:stat_imb_fr}, le balourd statique $U_s$ peut être repr\'esent\'e par une masse $m_s$ avec un d\'ecalage $r$ de l'axe de rotation \cite{CRS13}. Cette masse génère une force $F_s$ qui est sinusoïdale dans un rep\`ere fixe et d'amplitude :
\begin{equation}
F_s = m_s r \Omega^2 = U_s \Omega,
\end{equation}
avec $\Omega$ la vitesse angulaire constante de la roue. 
Le balourd dynamique $U_d$ est mod\'elis\'e comme deux masses \'egales $m_d$ \`a la même distance $r$ de l'axe de rotation et espac\'ees de la distance $d$ comme montr\'e dans la Fig. \ref{fig:dyn_imb_fr}. Un couple $T_d$ est produit quand ces deux masses tournent \`a la vitesse constante $\Omega$ :
\begin{equation}
	T_d = m_drd\Omega^2=U_d\Omega^2.
\end{equation} 

\begin{figure}[!h]%
    \centering
    \subfloat[]{{\includegraphics[width=0.3\columnwidth]{images/french_resume/static_imbal.eps} }\label{fig:stat_imb_fr}}%
    \qquad
    \subfloat[]{\includegraphics[width=0.3\columnwidth]{images/french_resume/dynamic_imbal.eps}\label{fig:dyn_imb_fr}}%
    \caption{(a) Balourd statique, (b) balourd dynamique.}%
    \label{fig:rw_imb_fr}%
\end{figure}

Il existe diff\'erentes façons de r\'eduire les micro-vibrations. Si l'on analyse le sch\'ema de la Fig. \ref{fig:microvib_isolation_fr} trois \'el\'ements contribuent \`a la d\'egradation des performances de la charge utile de la mission : la source des vibrations (\textit{disturbance source}), le chemin de propagation (\textit{propagation path}) constitu\'e par les structures flexibles du satellite entre la source et le r\'ecepteur, et la charge utile (\textit{payload}) elle-même. Des solutions pour augmenter la dissipation des vibrations le long le chemin de propagation sont  d'augmenter la distance entre la source et la charge utile ou de d\'ecoupler la source et la charge utile.

\begin{figure}[!th]
	\centering
	\includegraphics[width=0.7\linewidth]{images/french_resume/microvib_isolation.eps}
	\caption{M\'ethode d'isolation des micro-vibrations \cite{CRS13}.}
	\label{fig:microvib_isolation_fr}
\end{figure}

Une autre m\'ethode consiste à introduire des isolateurs passifs directement en correspondance de la source des perturbations et/ou de la charge utile. Une strat\'egie compl\'ementaire est l'utilisation d'actionneurs qui fournissent directement la compensation des forces et couples communiqu\'ees par les sources \`a la structure. Ces actionneurs peuvent être localis\'es \`a niveau de la source ou de la charge utile. Quand solutions passives et actives sont combin\'ees ensemble, on parle de strat\'egie de contr\^ole hybride. 

\section{Cas d'\'etude}

L'objectif final de ce travail de th\`ese est d'investiguer et valider un syst\`eme de contr\^ole actif pour le pointage fin, capable de rejeter les micro-vibrations au niveau de la charge utile et avec les caractéristiques suivantes: large bande-passante (typiquement jusqu'\`a 100 Hz), erreur r\'esiduelle de l'ordre de $\mu$rad, faible impact en masse et en volume, modulaire et adaptable. 
Le cas d'application est une mission scientifique avec un t\'elescope Korsch, comme celui employ\'e dans la mission \textit{Euclid} (Fig. \ref{fig:euclid_scheme_intro_fr}). On consid\`ere le sch\'ema simplifi\'e de la Fig. \ref{fig:application_case_intro_fr} avec trois miroirs ($M_1$, $M_2$ and $M_3$) et un syst\`eme de contr\^ole d'attitude bas\'e sur des roues \`a r\'eaction. Les performances en pointage de ce syst\`eme sont g\'en\'eralement bas\'ees sur la stabilit\'e de la ligne-de-vis\'ee autour des deux axes perpendiculaires \`a l'axe principal de sym\'etrie du t\'elescope.  Au premier ordre, ces rotations sont une combinaison lin\'eaire de rotations de miroirs et translations du plan focal, parmi lesquelles les rotations du $M_1$ sont les contributeurs les plus significatifs. 

\begin{figure}[!h]%
    \centering
    \subfloat[]{\includegraphics[width=0.35\columnwidth]{images/french_resume/euclid_scheme.eps}\label{fig:euclid_scheme_intro_fr}}%
    \qquad
    \subfloat[]{\includegraphics[width=0.35\columnwidth]{images/french_resume/application.eps}\label{fig:application_case_intro_fr}}%
    \caption{Stabilisation de la ligne-de-vis\'ee pour un t\'elescope spatial: (a) Mission Euclid. Image credit: Airbus Defence and Space. (b) Sch\'ema de principe du syt\`eme de contr\^ole de la ligne-de-vis\'ee propos\'ee dans ce travail de th\`ese.}%
    \label{fig:applic_fr}%
\end{figure}

Le syst\`eme propos\'e dans ce travail de th\`ese est bas\'e sur deux senseurs de vitesse angulaire (ARS), un pour chaque axe, plac\'es \`a la base du miroir $M_1$, un senseur de position CCD et un actionneur pi\'ezo-\'electrique, qui commande le miroir $M_3$ (FSM) en boucle ferm\'ee. Le senseur CCD, plac\'e au niveau de la charge utile, est utilis\'e pour mesurer les translations en basse fr\'equence dans le plan focal et compl\'eter l'information en haute fr\'equence fournie par les deux ARS.

Le syst\`eme exp\'erimental, utilis\'e au sein de l'ESA pour reproduire ce sc\'enario, est montr\'e dans la Fig. \ref{fig:experimental_setup_intro_fr}. Un autocollimateur a le double rôle de source laser et capteur de position (qui imite un capteur CCD). Le faisceau laser est dirig\'e tout d'abord vers un FSM, qui s'occupe de perturber la ligne-de-vis\'ee et qui imite le comportement du miroir $M_1$ de la Fig. \ref{fig:application_case_intro_fr}. Un ARS est install\'e \`a la base de cet FSM, qui imite le comportement du $M_1$ dans la Fig. \ref{fig:application_case_intro_fr}. Enfin un deuxi\`eme FSM est utilis\'e en boucle ferm\'ee (avec les mesures fournies par l' ARS et le capteur de position de l'autocollimateur) pour corriger la ligne-de-vis\'ee.     

\begin{figure}[th!]
	\centering
	\includegraphics[width=0.5\linewidth]{images/french_resume/sanfe1c.eps}
	\caption{Banc d'essai pour la stabilisation de la ligne-de-vis\'ee au sein de l'ESA.}	\label{fig:experimental_setup_intro_fr}
\end{figure}

\section[Mod\'elisation dynamique de syst\`emes multi-corps flexibles]{Contribution \`a la mod\'elisation dynamique de syst\`emes multi-corps flexibles avec points d'attachement multiple}

La m\'ethode utilis\'ee pour la mod\'elisation de syst\`emes flexibles dans ce travail de th\`ese prend le nom de th\'eorie Two-Input Two-Output Port, propos\'ee pour la premi\`ere fois par Alazard \textit{et al.} \cite{ alazard2015two}.  On consid\`ere l'$i$-\`eme appendice flexible $\mathcal{L}_i$ d'un syst\`eme multi-corps dans la Fig. \ref{fig:TITOP_fr}.
Il est discr\'etis\'e par une approche \'el\'ements finis (FEM) et il est connecté \`a la substructure parente $\mathcal{L}_{i-1}$ \`a travers le point $P$ et \`a la substructure enfante $\mathcal{L}_{i+1}$ \`a travers le point $C$. 

\begin{figure}[ht]
\centering
\includegraphics[scale=0.8]{images/french_resume/Fig2.eps} % The printed column width is 8.4 cm.
\caption{$i$-\`eme appendice flexible d'un syst\`eme multi-corps.} 
\label{fig:TITOP_fr}
\end{figure}

Le mod\`ele TITOP $\mathcal{M}_{PC}^{\mathcal{L}_i}(s)$ (o\`u $s$ est la variable de Laplace) est une repr\'esentation d'\'etat avec 12 entrées (six pour chacune des deux ports) :
\begin{enumerate}  
\item Les six composantes dans le rep\`ere $\mathcal{R}_0=(P^0 ;x_0,y_0,z_0)$ du vecteur $\mathbf{W}_{\mathcal{L}_{i+1}/\mathcal{L}_i,C}$ constitu\'e par les vecteurs des forces $\mathbf{F}_C$ (trois composantes) et des couples $\mathbf{T}_C$ (trois composantes) appliqu\'es par $\mathcal{L}_{i+1}$ \`a $\mathcal{L}_i$ au niveau du nœud $C$; 
\item Les six composantes dans le rep\`ere $\mathcal{R}_0$ du vecteur des acc\'el\'erations $\ddot{\mathbf{u}}_{P}$ constitu\'e par les vecteurs des acc\'el\'erations  lin\'eaires $\ddot{\mathbf{a}}_{P}$ (trois composantes) et des acc\'el\'erations rotationnelles $\dot{\bm{\omega}}_{P} $  (trois composantes) appliqu\'es par $\mathcal{L}_{i}$ \`a $\mathcal{L}_{i-1}$ au niveau du nœud $P$;
\end{enumerate}  
Et 12 sorties (6 pour chacune des deux ports) :
\begin{enumerate}
\item Les six composantes dans $\mathcal{R}_0$ du vecteur des acc\'el\'erations $\ddot{\mathbf{u}}_{C}$ au niveau du nœud $C$;
\item Les six composantes dans $\mathcal{R}_0$ du vecteur des forces/couples $\mathbf{W}_{\mathcal{L}_{i}/\mathcal{L}_{i-1},P}$ appliqu\'es par $\mathcal{L}_i$ \`a $\mathcal{L}_{i-1}$ en correspondance du nœud $P$.  
\end{enumerate}

Le diagramme du mod\`ele TITOP $\mathcal{M}_{PC}^{\mathcal{L}_i}$ est montr\'e dans la Fig. \ref{fig:block_titop_fr}.

\begin{figure}[ht]
\centering
\includegraphics[scale=0.7]{images/french_resume/Fig3.eps} % The printed column width is 8.4 cm.
\caption{Diagramme du mod\`ele TITOP $\mathbf{\mathcal{M}}_{PC}^{\mathcal{L}_i}$.} 
\label{fig:block_titop_fr}
\end{figure}

La r\'ealisation d'\'etat du mod\`ele TITOP $\mathcal{M}_{PC}^{\mathcal{L}_i}$ r\'esulte en : 
\begin{equation}
	\left[\begin{array}{c} 
		\dot{\bm{\eta}} \\ \ddot{\bm{\eta}} \\ \hline
		\ddot{\mathbf{u}}_C \\ \mathbf{W}_{\mathcal{L}_{i}/\mathcal{L}_{i-1},P}
	\end{array} \right] =
	\left[\begin{array}{c|c}
		 \mathbf{A} & \mathbf{B} \\
		 \hline
		 \mathbf{C} & \mathbf{D}
	\end{array} \right]
	\left[\begin{array}{c} 
		\bm{\eta} \\ \dot{\bm{\eta}} \\ \hline \mathbf{F}_{\mathcal{L}_{i+1}/\mathcal{L}_{i},C} \\ \ddot{\mathbf{u}}_P
	\end{array} \right],
	\label{eq:titop_model_fr}
\end{equation}

avec
\begin{equation}
	\begin{array}{ll}
		\mathbf{A} = \left[\begin{array}{cc}
\mathbf{0}_{N_i\times N_i} & \mathbf{I}_{N_i} \\
-\mathbf{k} & -\mathbf{c}
\end{array}\right], & \mathbf{B} = \left[\begin{array}{cc}
\mathbf{0}_{N_i\times 6} & \mathbf{0}_{N_i\times 6} \\
\mathbf{\Phi}_{C}^{\mathrm{T}} & -\mathbf{L}_{P}
\end{array}\right], \\
\mathbf{C}= \left[\begin{array}{cc}
	-\mathbf{\Phi}_{C}\mathbf{k} & -\mathbf{\Phi}_{C}\mathbf{c} \\
	\mathbf{L}_{P}^{\mathrm{T}}\mathbf{k} & \mathbf{L}_{P}^{\mathrm{T}}\mathbf{c}
\end{array}\right],  & \mathbf{D} = \left[\begin{array}{cc}
	\mathbf{\Phi}_{C}\mathbf{\Phi}_{C}^{\mathrm{T}} & (\bm{\tau}_{CP}-\mathbf{\Phi}_{C}\mathbf{L}_{P}) \\
	(\bm{\tau}_{CP}-\mathbf{\Phi}_{C}\mathbf{L}_{P})^{\mathrm{T}} & \mathbf{L}_{P}^{\mathrm{T}}\mathbf{L}_{P} - \mathbf{M}_{\mathrm{rr}}
\end{array}\right],
\end{array}.
\end{equation}
où $N_i$ est le nombre total des d\'egr\'es de libert\'e, $\bm{\eta}$ est le vecteur des coordonn\'ees g\'en\'eralis\'ees, $\mathbf{k}=\mathrm{diag}(\omega_k^2)$ avec $\omega_k $ la $k$-\`eme pulsation de résonance, $\mathbf{c}=\mathrm{diag}(2\zeta_k\omega_k)$ avec $\zeta_k $ l'amortissement associ\'e au $k$-\`eme mode, $\bm{\Phi}_{C}$ déformée modale, $\mathbf{L}_{P}$ matrice des facteurs de participation par rapport au nœud $P$ et $\mathbf{M}_{rr}$ matrice de masse rigide.
Ce mod\`ele, conçu comme encastr\'e en $P$ et libre en $C$, peut être utilisé pour \'etudier n'importe quelle autre condition limite, comme prouv\'e par \cite{chebbi2016linear}, gr\^ace \`a l'invertibilit\'e des 12 entr\'ees-sorties.
Le mod\`ele \eqref{eq:titop_model_fr} peut être \'etendu au cas où le corps flexible $\mathcal{L}_i$ est connect\'e \`a plusieurs substructures enfant  comme dans l'exemple montr\'e dans la Fig. \ref{fig:NINOP_fr}. 

\begin{figure}[ht]
\centering
\includegraphics[scale=0.7]{images/french_resume/Fig4.eps} % The printed column width is 8.4 cm.
\caption{$i$-\`eme appendice d'un syst\`eme multi-corps connect\'ee \`a deux structures enfants.} 
\label{fig:NINOP_fr}
\end{figure}

Dans ce cas le mod\`ele $\mathcal{M}_{PC_1C_2}^{\mathcal{L}_i}$ s'\'ecrit :
\begin{equation}
	\left[\begin{array}{c} 
		\dot{\bm{\eta}} \\ \ddot{\bm{\eta}} \\ \hline
		\ddot{\mathbf{u}}_{C_1} \\ \ddot{\mathbf{u}}_{C_2} \\ \mathbf{W}_{\mathcal{L}_{i}/\mathcal{L}_{i-1},P}
	\end{array} \right] =
	\left[\begin{array}{c|c}
		\mathbf{A} & \mathbf{B} \\ \hline
		\mathbf{C} & \mathbf{D} 
	\end{array} \right]
	\left[\begin{array}{c} 
		\bm{\eta} \\ \dot{\bm{\eta}} \\ \hline \mathbf{W}_{\mathcal{L}_{i+1}^{'}/\mathcal{L}_{i},C_1} \\ \mathbf{W}_{\mathcal{L}_{i+1}^{''}/\mathcal{L}_{i},C_2} \\ \ddot{\mathbf{u}}_P
	\end{array} \right],	
	\label{eq:ninop_model_fr}
\end{equation}

avec

\begin{equation*}
\begin{footnotesize}
\begin{array}{lll}
	\mathbf{A} = \left[\begin{array}{cc}
\mathbf{0}_{N_i\times N_i} & \mathbf{I}_{N_i} \\
-\mathbf{k} & -\mathbf{c}
\end{array}\right], 
	& \mathbf{B} =  \left[\begin{array}{ccc}
	\mathbf{0}_{N_i\times 6} & \mathbf{0}_{N_i\times 6} & \mathbf{0}_{N_i\times 6} \\
	\mathbf{\Phi}_{C_1}^{\mathrm{T}} & \mathbf{\Phi}_{C_2}^{\mathrm{T}} & -\mathbf{L}_{P}
	\end{array}\right], &
	\mathbf{C} = \left[\begin{array}{cc}
		-\mathbf{\Phi}_{C_1}\mathbf{k} & -\mathbf{\Phi}_{C_1}\mathbf{c} \\
		-\mathbf{\Phi}_{C_2}\mathbf{k} & -\mathbf{\Phi}_{C_2}\mathbf{c}  \\
		\mathbf{L}_{P}^{\mathrm{T}}\mathbf{k} & \mathbf{L}_{P}^{\mathrm{T}}\mathbf{c}
	\end{array}\right]
\end{array},
\end{footnotesize}
\end{equation*}
\begin{equation}
\begin{small}
		\mathbf{D} = \left[\begin{array}{ccc}
		\mathbf{\Phi}_{C_1}\mathbf{\Phi}_{C_1}^{\mathrm{T}} & \mathbf{\Phi}_{C_1}\mathbf{\Phi}_{C_2}^{\mathrm{T}} & (\bm{\tau}_{C_1P}-\mathbf{\Phi}_{C_1}\mathbf{L}_{P}) \\
		\mathbf{\Phi}_{C_2}\mathbf{\Phi}_{C_1}^{\mathrm{T}} & \mathbf{\Phi}_{C_2}\mathbf{\Phi}_{C_2}^{\mathrm{T}} & (\bm{\tau}_{C_2P}-\mathbf{\Phi}_{C_2}\mathbf{L}_{P}) \\
		(\bm{\tau}_{C_1P}-\mathbf{\Phi}_{C_1}\mathbf{L}_{P})^{\mathrm{T}} & (\bm{\tau}_{C_2P}-\mathbf{\Phi}_{C_2}\mathbf{L}_{P})^{\mathrm{T}} & \mathbf{L}_{P}^{\mathrm{T}}\mathbf{L}_{P} - \mathbf{M}_{\mathrm{rr}}
\end{array}\right].
\end{small}
\end{equation}

Gr\^ace \`a l'extension de l'approche TITOP aux corps flexibles avec points d'attachement multiples, un mod\`ele de plaque de Kirchhoff a \'et\'e synth\'etis\'e. Ce mod\`ele a \'et\'e compar\'e avec les r\'esultats fournis par le logiciel MSC/\textsc{Nastran} pour sa validation et a permis l'\'etude de l'op\'eration d'inversion des ports de l'approche TITOP. Les valeurs propres normalis\'ees obtenues pour la plaque dans la Fig. \ref{fig:plate_mesh_fr} sont pr\'esent\'ees dans le Tableau \ref{tab:plate_com_fr} pour le mod\`ele direct (encastr\'e en $P$ et libre en $C$) et le mod\`ele inverse (encastr\'e en $C$ et libre en $P$). La sym\'etrie des valeurs singuli\`eres du mod\`ele TITOP dans la Fig. \ref{fig:sigma_plate_fr} confirme l'exactitude de l'op\'eration d'inversion.

 \begin{minipage}{\columnwidth}
  \begin{minipage}[b]{0.3\columnwidth}
    \centering
    \includegraphics[width=\linewidth]{images/chapter1/Fig12.eps}
    \captionof{figure}{Mesh de la plaque  encastr\'ee en $P$ et libre en $C$.}
    \label{fig:plate_mesh_fr}
  \end{minipage}
  \hfill
  \begin{minipage}[b]{0.64\columnwidth}
    \centering
    \captionof{table}{Valeurs propres des mod\`eles TITOP direct et inverse obtenues par l'\'element Kirchhoff et l'\'element \textsc{MSC/Nastran} \textit{CQUAD4}. Les fr\'equences sont exprim\'ees en $\unit[]{Hz}$.}
    \begin{tabular}{ccccc}
\hline
\multicolumn{1}{c}{} & \multicolumn{2}{c}{$\mathbf{\mathcal{M}}_{\mathrm{TITOP}}$} & \multicolumn{2}{c}{$\mathbf{\mathcal{M}}_{\mathrm{MSC}}$} \\ \hline
 Mode & Direct & Inverse & Direct & Inverse \\ \hline
1 & 1.3358 & 1.3358 & 1.3192 & 1.4521 \\
2 & 2.7115 & 2.7115 & 2.6995 & 2.9098 \\
3 & 8.5304 & 8.5304 & 8.4979 & 8.7739 \\
4 & 17.2851 & 17.2851 & 17.6446 & 18.2240 \\
5 & 22.1254 & 22.1254 & 22.6662 & 23.5798 \\
6 & 26.9616 & 26.9616 & 27.5835 & 27.7454 \\
7 & 32.0298 & 32.0298 & 32.6737 & 33.5411 \\
8 & 49.6710 & 49.6710 & 52.0400 & 54.1891 \\
9 & 56.4259 & 56.4259 & 59.8442 & 61.6793 \\
10 & 58.6866 & 58.6866 & 63.1416 & 63.1010 \\ 
 \hline 
\label{tab:plate_com_fr} 
\end{tabular}
    \end{minipage}
  \end{minipage}

\begin{figure}[!ht]
\centering
\includegraphics[width=0.65\columnwidth]{images/chapter1/Fig14.pdf} % The printed column width is 8.4 cm.
\caption{Valeurs singulieres des mod\`eles TITOP obtenus par l'\'element Kirchhoff et l'\'element \textsc{MSC/Nastran} \textit{CQUAD4}.} 
\label{fig:sigma_plate_fr}
\end{figure}

Le modèle N-Input N-Output Port (NINOP) de la plaque de Kirchhoff a été aussi validé expérimentalement grâce au banc d’essai dans la Fig. \ref{fig:beam_bench_fr} développé au sein des laboratoires de l’ESA/ESTEC. 

\begin{figure}[!ht]
\centering
\includegraphics[width=0.99\columnwidth]{images/chapter1/beam_bench.eps} % The printed column width is 8.4 cm.
\caption{Banc d'essai pour plaque flexible.} 
\label{fig:beam_bench_fr}
\end{figure}

Une plaque flexible $\mathcal{P}$ est encastrée sur un bout et présente un miroir $\mathcal{L}$ à l’autre extrémité. Deux actionneurs inertiels (PMAs), $\mathcal{A}_1$ et $\mathcal{A}_2$ sont connectés à la plaque en deux locations différentes. Quand ils sont activés ils provoquent un déplacement de la plaque qui est mesuré par une lunette autocollimatrice, localisée en face du miroir. Le problème se réduit à l’assemblage de quatre modèles avec l’approche TITOP. 
Le schéma-bloc du système entier est décrit dans la Fig. \ref{fig:titop_plate_exp_fr}. La validation du modèle théorique avec les résultats expérimentaux est présentée dans la Fig. \ref{fig:exp_fig_fr}.
\begin{figure}[!ht]
\centering
\includegraphics[width=0.6\columnwidth]{images/chapter1/titop_plate_exp.pdf} % The printed column width is 8.4 cm.
\caption{Modéle de plaque encastrée avec deux actionneurs inertiels et un miroir.} 
\label{fig:titop_plate_exp_fr}
\end{figure}
\begin{figure*}[!th]
    \centering
    \def\svgwidth{\textwidth}
    {
    \small
    \import{images/chapter1/}{exp_fig.pdf_tex}
    }
    \caption{Comparaison entre modèles expérimental et analytique : $u_1\rightarrow \mathbf{q}_{S}(2)$ (à gauche), $u_2\rightarrow \mathbf{q}_{S}(2)$ (à droite).}
\label{fig:exp_fig_fr}
\end{figure*}

\section[Mod\'elisation sources de micro-vibrations]{Contribution \`a la mod\'elisation des sources de \\ micro-vibrations}

La modélisation TITOP a été utilisée pour la synthèse des modèles d’un satellite flexible avec ses sources de micro-vibrations principales : un système de roue à réaction et un moteur d’entrainement de panneaux solaires. 
Le système de roue à réaction présenté dans la Fig. \ref{fig:reaction_wheel_scheme_fr} a été modélisé à travers l’assemblage de trois blocs élémentaires : une roue tournante à la vitesse constante $\Omega_w$, un système raideur-amortisseur et un corps rigide. Le modèle LPV résultant, variable avec $\Omega_w$, avec sa représentation LFT est montré dans la Fig. \ref{fig:rwa_titop_fr}.

\begin{figure}[!ht]
\centering
\includegraphics[width=0.9\columnwidth]{images/chapter2/reaction_wheel_scheme.eps} % The printed column width is 8.4 cm.
\caption{Reaction Wheel Assembly schematic view.} 
\label{fig:reaction_wheel_scheme_fr}
\end{figure}

Le couplage de ce système de roue à réaction avec un satellite flexible et la modélisation analytique des micro-vibrations produites par les balourds de la roue, comme illustré dans la Fig. \ref{fig:sat_rwa_lpv_fr}, a permis l’étude de la robustesse des performances de pointage fin du système entier avec les outils de la valeur singulière structurée $\mu$. Un exemple de l’étude de pire-cas, qui peut être conduit à travers ce modèle est illustré dans la Fig. \ref{fig:mu_wheel_sat_fr}. La flexibilité de l’approche TITOP permet d’introduire facilement n’importe quelle incertitude paramétrique et/ou non-paramétrique associée à chaque sous-système en permettant l’analyse directe de l’impact des variations de ce paramètre sur les performances souhaitées.    

\begin{figure}[!ht]
\centering
\includegraphics[width=.9\columnwidth]{images/chapter2/rwa_titop.pdf} % The printed column width is 8.4 cm.
\caption{Modèle TITOP de la roue à réaction avec diagramme LFT correspondent.} 
\label{fig:rwa_titop_fr}
\end{figure}

\begin{figure}[!h]
\centering
\includegraphics[width=.9\columnwidth]{images/chapter2/sat_rwa_lpv.pdf} % The printed column width is 8.4 cm.
\caption{Diagramme d'assemblage d'un satellite flexible avec roues à réaction et son diagramme LFT.} 
\label{fig:sat_rwa_lpv_fr}
\end{figure}

\begin{figure}[!h]
\centering
\includegraphics[width=0.9\columnwidth]{images/chapter2/mu_wheel_sat.png} % The printed column width is 8.4 cm.
\caption{Borne supérieure $\mu_{\Delta}$ du transfer $\tilde{\mathbf{d}}_{\mathrm{hd}}\rightarrow\mathbf{e}_{\theta_{G}}(1)$.} 
\label{fig:mu_wheel_sat_fr}
\end{figure}


À travers la même procédure le modèle TITOP d’un moteur pas-à-pas d’entrainement des panneaux solaires a été élaboré. La réduction des micro-vibrations induites par ce système, dues surtout au fonctionnement en micro-pas imposé par le profil de commande transmise au rotor, est fondamental pour ne pas interrompre le fonctionnement du système pendant les phases sensibles de prise d’image et aussi optimiser l’énergie récoltée. 
Le modèle synthétisé est obtenu par linéarisation de l’équation de la dynamique le long de l’axe rotor en ayant comme entrées la commande des micro-pas $i$, le vecteur des accélérations communiquées par le satellite au système d’entrainement du panneau solaire et le vecteur des forces/couples imposés par le panneau solaire au rotor, et comme sorties, la rotation du rotor, le vecteur des accélérations à la sortie du système d’entrainement et le vecteur des forces/couples de réaction imposés par le moteur au satellite (Fig. \ref{fig:sadm_titop_block_diagram_fr}). 

\begin{figure}[!ht]
\centering
\includegraphics[width=\columnwidth]{images/chapter2/sadm_titop_block_diagram.pdf} % The printed column width is 8.4 cm.
\caption{Diagramme du modèle TITOP d'un moteur d'entrainement de panneau solaire.} 
\label{fig:sadm_titop_block_diagram_fr}
\end{figure}

L’assemblage de ce système avec le modèle d’un panneau solaire flexible permet l’étude de l’évolution de la dynamique en fonction de l’angle de révolution $\theta_r$ du rotor. La paramétrisation introduite par \cite{guy2014dynamic}, qui permet de linéariser la matrice de rotation entre les deux repères associés à chaque corps flexible, en exprimant les fonctions trigonométriques de $sin$ et $cos$ en fonction de $\tau = \tan(\theta_r/4)$ :
\begin{equation}
\cos\theta_r = \frac{\left( 1+\tau^2\right)^2-8\tau^2}{\left( 1+\tau^2\right)^2}, \quad \sin\theta_r = \frac{4\tau\left( 1-\tau^2\right)}{\left( 1+\tau^2\right)^2}, \quad \mathrm{pour\,tout}\quad\tau\in\left]-1;1\right],
\label{eq:parametrization_fr}
\end{equation}

permet en fait d’obtenir un modèle LPV basé sur $\tau$ (Fig. \ref{fig:lpv_sadm_fr}). Ce modèle englobe du coup toutes les possibles configurations atteintes par le système en tenant en compte de toutes ses incertitudes paramétriques pour mener des analyses pire-cas sur les performances souhaitées.

\vspace{-10pt}
\begin{figure}[!ht]
\centering
\includegraphics[width=\columnwidth]{images/chapter2/lpv_sadm.pdf} % The printed column width is 8.4 cm.
\caption{Modèle LPV d'un moteur pas-à-pas qui entraine un panneau solaire flexible.} 
\label{fig:lpv_sadm_fr}
\end{figure}
\vspace{-10pt}

La Fig. \ref{fig:sadm_only_fr} montre l’évolution des valeurs singulières de la configuration nominale des deux transferts entre la commande du micro-pas $i$ et deux composantes du couple de réaction transférées par le moteur qui entraine les panneaux solaires flexibles à la structure où le moteur est fixé.

\begin{figure}[!ht]
\centering
\includegraphics[width=\columnwidth]{images/chapter2/sadm_only.png} % The printed column width is 8.4 cm.
\caption{Valeurs singulières des transferts (a) $i\rightarrow \mathbf{W}_{\mathcal{Q}/\mathcal{S},P}(4)$ et (b) $i\rightarrow \mathbf{W}_{\mathcal{Q}/\mathcal{S},P}(6)$.} 
\label{fig:sadm_only_fr}
\end{figure}

La caractérisation analytique du signal générateur du micro-pas $i$ permet d’étudier analytiquement l’impact de la source de micro-vibrations sur le pointage d’un satellite doté d’un moteur d’entrainement des panneaux solaires. Le signal $i$ peut être en fait décomposé en trois signaux comme montré dans la Fig. \ref{fig:composition_i_fr} : une rampe de pente égale à l’inverse de la période du micro-pas $T_{\mu}$, un biais et  un signal en dents de scie renversé. 

\begin{figure}[!ht]
\centering
\includegraphics[width=\columnwidth]{images/chapter2/composition_i.eps} % The printed column width is 8.4 cm.
\caption{Décomposition du signal générateur de micro-pas.} 
\label{fig:composition_i_fr}
\end{figure}

Des trois signaux le principale responsable des micro-vibrations est le dernier que l'on nomme $i_3$. Son expression analytique s'écrit:
\begin{equation}
i_3 = \left\lfloor\frac{1}{2}+\frac{t}{T_\mu}\right\rfloor-\frac{t}{T_\mu},
\label{eq:sawtooth_fr}
\end{equation}

et peut être approximé par la série de Fourier :
\begin{equation}
i_3 = \frac{1}{\pi}\sum_{j=1}^{\infty} \frac{1}{h_j}\sin\left(\frac{2\pi h_j}{T_\mu}t\right).
\label{eq:sadm_harmonics_fr}
\end{equation}

Ce signal peut être approximé par un nombre $N$ fini d’harmoniques significatives. L’objet est de trouver les filtres :
\begin{equation}
\mathbf{W}_{\mathrm{hd}} = \left[W_{\mathrm{hd}_1},\,\dots,\,W_{\mathrm{hd}_j},\,\dots,\,W_{\mathrm{hd}_N}\right],
\end{equation}
qui génèrent ces $N$ harmoniques en régime permanent pour des signaux d'entrée harmoniques normalisés $\mathbf{d}_{i_3}=\left[{d}_{i_{3_1}},\,\dots,\,{d}_{i_{3_j}},\,\dots,\,{d}_{i_{3_N}}\right]^{\mathrm{T}}$, avec ${d}_{i_{3_j}} = \sin\left(\frac{2\pi h_j}{T_\mu} t\right)$.

Le $j$-ème filtre $W_{\mathrm{hd}_j}$ prend la forme :

\begin{centering}
\fcolorbox{black}{blue!2}{%
    \minipage[t]{0.98\columnwidth}
\begin{equation}
	W_{\mathrm{hd}_j} = \frac{1}{\pi h_j}F_{\mathrm{peak}} = \frac{1}{\pi h_j}\left[ 
		\begin{array}{cc|c}
			-\beta & -h_j\Omega_q z p n_\mu & \beta \\
			h_j\Omega_q z p n_\mu & 0 & 0 \\ \hline
			1 & 0 & 0
 		\end{array}
	\right],
\end{equation}
\endminipage}
\end{centering}

où $h_j$ est le nombre d’harmoniques, $\Omega_q$ est la vitesse de rotation souhaitée pour le panneau solaire, $z$ est le nombre de dents du stator, $p$ est le nombre de paires de pôles, $n_\mu$ est le nombre de micro-pas et $\beta$ est un paramètre qui contrôle la bande passante du filtre :
\begin{equation}
\beta = \left\{ \begin{array}{l}
	1\,\quad \mathrm{si}\quad \frac{4\pi h_j}{T_\mu}>0.1\mathrm{rad/s} \\
	10^{-2}\quad\mathrm{autrement}
\end{array}\right.
\end{equation}

La $\mu$-analyse permet de mener une étude de sensibilité paramétrique pour vérifier la robustesse en performance d’un système qui respecte toutes les contraintes imposées par le cahier de charge en état nominal. Ce type d’analyse peut révéler si, en cas d’incertitudes, ces contraintes sont dépassées et quels sont les paramètres responsables. Un exemple de ce type d’analyse est montré dans les figures \ref{fig:sadm_particular_satellite_fr} et \ref{fig:mu_sadm_sensitivity_fr}. La borne supérieure de $\mu$ montre par exemple comment l’incertitude sur une des fréquences de résonance ($\omega_4^{\mathrm{SA}}$) du panneau solaire agit sur le dépassement des performances à cause du couplage de ce mode flexible avec l’excitation générée par le micro-pas du moteur d’entrainement. 

\begin{figure}[!ht]
\centering
\includegraphics[width=\columnwidth]{images/chapter2/sadm_particular_satellite.png} % The printed column width is 8.4 cm.
\caption{Détail de l'interaction entre l'excitation produite par le moteur d'entrainement et les modes flexibles structurelles pour le transfert $\left[\mathbf{d}_{i_3}^{\mathrm{T}}\,\mathbf{d}_{T_d}^{\mathrm{T}}\right]^{\mathrm{T}}\rightarrow \mathbf{e}_{\theta_G}(3)$ avec $\theta_r\in\left[0,180\right]^{\circ}$.} 
\label{fig:sadm_particular_satellite_fr}
\end{figure}

\begin{figure}[!th]
\centering
\includegraphics[width=\columnwidth]{images/chapter2/mu_sadm_sensitivity.png} % The printed column width is 8.4 cm.
\caption{Diagramme de sensibilité : borne supérieure de $\mu_{\Delta}$ pour le transfert $\left[\mathbf{d}_{i_3}^{\mathrm{T}}\,\mathbf{d}_{T_d}^{\mathrm{T}}\right]^{\mathrm{T}}\rightarrow \mathbf{e}_{\theta_G}(3)$ for $\theta_r\in\left[0,180\right]^{\circ}$.} 
\label{fig:mu_sadm_sensitivity_fr}
\end{figure}

\section{Contribution au contr\^ole robuste de la ligne-de-vis\'ee}
Après la caractérisation et la modélisation des sources principales de micro-vibrations, un banc d’essai, développé au sein de l’ESA/ESTEC, a été développé avec pour objectifs : la reproduction fidèle des micro-vibrations et leur contrôle actif. 
Le dispositif clé identifié par l’ESA dans le cadre de ce projet de thèse est un senseur avec large bande passante, basé sur le principe magnétohydrodynamique. Le capteur pris en compte dans cette étude est le ARS-14 produit par Applied Technology Associates (ATA).
Un banc d’essai dédié à la caractérisation de ce système (montré dans la Fig. \ref{fig:banco_ars}) a été réalisé pour l’identification de la fonction transfert de l’ARS-14 et pour tester les limites de performances atteintes par ce senseur sur la bande de fréquence typique de manifestation des micro-vibrations.
\begin{figure}[!h]
\centering
    \subfloat[]{\includegraphics[width=0.7\columnwidth]{images/chapter3/mech_design2.pdf}}%
    \quad
   \subfloat[]{\includegraphics[angle=90,width=0.25\columnwidth]{images/chapter3/CONEX_table.pdf}}   
\caption{Banc d'essai pour la caractérisation de l'ARS-14 : (a) design, (b) banc d'essai expérimental.}
  \label{fig:banco_ars}
\end{figure}

L’estimation non-paramétrique du modèle est faite avec la méthode d’analyse spectrale développé par \cite{ljung1987system}. Le modèle final est obtenu par résolution d’un problème d’optimisation non-linéaire au moindres carrés. Le modèle d’ARS choisi résulte du meilleur compromis entre précision et complexité. L’inévitable erreur produite par l’opération de paramétrisation du modèle est après prise en compte pour établir le niveau d’incertitude sur le modèle nominal, qui est modélisée comme une incertitude multiplicative. Le modèle incertain est montré dans la Fig. \ref{fig:nominal_uncertain_bode_fr}.

\begin{figure*}
    \centering
        \def\svgwidth{0.5\textwidth}
    {
    \import{images/chapter3/}{nominal_uncertain_bode.pdf_tex}
    }
   \caption{Réponse en fréquence du modèle nominal idéntifié de l'ARS-14 et du modèle incertain due à l'erreur de paramétrisation.}
         \label{fig:nominal_uncertain_bode_fr}
\end{figure*}

Ce senseur est ensuite intégré dans le banc d’essai montré dans la Fig. \ref{fig:experimental_setup_intro_fr}. Deux contrôleurs ont été synthétisés pour ce système comme montré dans la Fig. \ref{fig:bench_scheme_fr} :
\begin{itemize}
\item un contrôleur (DFSM Controller) pour l’injection d’un profil de micro-vibrations souhaité pour la perturbation de la ligne-de-visée;
\item un contrôleur (CFSM Controller) dédié au contrôle de la ligne-de-visée basé sur la fusion des mesures fournies par le senseur de position de la lunette autocollimatrice et l’ARS.
\end{itemize}

Les deux contrôleurs sont obtenus par la technique moderne de contrôle optimal robuste $\mathcal{H}_\infty$-structuré \cite{apkarian2011,apkarian2015}. 
Les deux schémas de synthèse et analyse sont montrés dans les figures \ref{fig:control_scheme_DFSM_fr} et \ref{fig:control_scheme_bench_fr}.

Grâce à la valeur singulière structurée $\mu$, il est aussi possible faire une étude pire-cas des performances garanties de manière robuste du système de stabilisation de la ligne-de-visée conçu dans le cadre de cette thèse. Une synthèse de résultats obtenus pour trois scénarios différents est illustrée dans la Fig. \ref{fig:comparison_hybridization_fr} :
\begin{enumerate}
\item Seules les mesures de la lunette autocollimatrice sont disponibles ;
\item Seules les mesures de l’ARS sont disponibles ;
\item Les mesures de la lunette autocollimatrice et de l’ARS sont disponibles et sont fusionnées.
\end{enumerate}

\begin{figure}[!th]
\centering
     \includegraphics[width=0.8\linewidth]{images/chapter3/sanfe2.pdf}
 \caption{Architecture \textit{hardware-in-the-loop} du banc d’essai pour le contrôle robuste de la ligne-de-visée.}
 \label{fig:bench_scheme_fr}
\end{figure}

\begin{figure*}[!ht]%
    \centering
    \subfloat{{\includegraphics[width=0.8\linewidth]{images/chapter3/sanfe4a.pdf} }}%
    \\
    \subfloat{{\includegraphics[width=0.35\linewidth]{images/chapter3/sanfe4b.pdf} }}%
    \caption{Diagramme de synthèse et analyse pour le Disturbing Fast Steering Mirror (DFSM) et diagramme LFT correspondent.}%
    \label{fig:control_scheme_DFSM_fr}%
\end{figure*}
\begin{figure*}[!ht]%
    \centering
    \subfloat{{\includegraphics[width=\linewidth]{images/chapter3/sanfe6a.pdf} }}%
    \\
    \subfloat{{\includegraphics[width=0.35\linewidth]{images/chapter3/sanfe6b.pdf} }}%
    \caption{Diagramme de synthèse et analyse pour le Control Fast Steering Mirror (CFSM) et diagramme LFT correspondant.}%
    \label{fig:control_scheme_bench_fr}%
\end{figure*}

\begin{figure*}[!h]
    \centering
    \includegraphics[width=\linewidth]{images/chapter3/sanfe13.png}
    \caption{Analyse pire-cas et résultats expérimentaux pour la réjection des micro-vibrations.}
    \label{fig:comparison_hybridization_fr}
\end{figure*}

\newpage
\section{Conclusions and perspectives}

Les différentes contributions résumées dans ce document ont permis de développer l'expertise du laboratoire d'accueil dans le domaine de la modélisation et du contrôle des grandes structures spatiales flexibles. Elles ont donné lieu à des publications:
\begin{itemize}
\item Sanfedino, Francesco and Alazard, Daniel and Pommier-Budinger, Valérie and Boquet, Fabrice and Falcoz, Alexandre. \textit{Dynamic modeling and analysis of micro-vibration jitter of a spacecraft with solar arrays drive mechanism for control purposes}. (2017) In: 10th International ESA Conference on Guidance, Navigation \& Control Systems (GNC 2017), 29 May 2017 - 2 June 2017 (Salzburg, Austria).
\item Sanfedino, Francesco and Alazard, Daniel and Pommier-Budinger, Valérie and Falcoz, Alexandre and Boquet, Fabrice. \textit{Finite element based N-Port model for preliminary design of multibody systems}. (2018) Journal of Sound and Vibration, 415. 128-146. ISSN 0022-460X
\item Sanfedino, Francesco and Alazard, Daniel and Pommier-Budinger, Valérie and Boquet, Fabrice and Falcoz, Alexandre. \textit{A novel dynamic model of a reaction wheel assembly for high accuracy pointing space missions}. (2018) In: ASME 2018 Dynamic Systems and Control Conference (DSCC2018), 30 September 2018 - 3 October 2018 (Atlanta, United States). 
\end{itemize}
Les principales perspectives à ces travaux sont:
\begin{itemize}
\item la synthèse LPV à partir des modèles de roues paramétrées en fonction de la vitesse de rotation et du SADM paramétré en fonction de la configuration angulaire;
\item modélisation TITOP de la structure de la charge utile optique;
\item Extension aux miroirs déformables.
\end{itemize}
