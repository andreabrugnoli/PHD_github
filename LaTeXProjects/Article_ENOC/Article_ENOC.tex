%% June 2019
%% ENOC2020 template file for LaTeX modified by Sebastien Baguet, University of Lyon, France 
%% based on 
%% ENOC2017 template file for LaTeX modified by Gabor Csernak, Budapest University of Technology and Economics
%% ENOC2014 template file for LaTeX modified by Horst Ecker, Vienna University of Technology
%% ENOC2011 template file for LaTeX generated by Andrea Arena, Sapienza University of Rome
%% This file was generated with a LateX2e version.
%%%%%%%%%%%%%%%%%%%%%%%%%%%%%%%%%%%%%%%%%%%%%%%%%%%%%%%%%%%%
\documentclass{enoc2020}
\usepackage{natbib}
%\usepackage{diffcoeff} % désolé chez moi ça n'a jamais fonctionné correctement
\usepackage{comment}

\bibliographystyle{unsrtnat}


% Math macros
\DeclareMathOperator*{\grad}{grad}
\DeclareMathOperator*{\Grad}{Grad}
\DeclareMathOperator*{\Div}{Div}
\renewcommand{\div}{\operatorname{div}}
\DeclareMathOperator*{\Hess}{Hess}
\DeclareMathOperator*{\curl}{curl}
\DeclareMathOperator{\Tr}{Tr}
\DeclareMathOperator{\Dom}{Dom}
\DeclareMathOperator*{\esssup}{ess\,sup}

\newcommand{\bbR}{\mathbb{R}}
\newcommand{\bbF}{\mathbb{F}}
\newcommand{\bbA}{\mathbb{A}}
\newcommand{\bbB}{\mathbb{B}}
\newcommand{\bbS}{\mathbb{S}}

\newcommand*{\norm}[1]{\ensuremath{\left\|#1\right\|}}
\newcommand{\where}{\qquad \text{where} \qquad}
\newcommand{\inner}[3][]{\ensuremath{\left\langle #2, \, #3 \right\rangle_{#1}}}
\newcommand{\bilprod}[2]{\left\langle \left\langle \, #1, #2 \, \right\rangle \right\rangle}
\newcommand{\pder}[2]{\ensuremath{\partial_{#2} #1}}
\newcommand{\dder}[2]{\ensuremath{\delta_{#2} #1}}
\newcommand{\secref}[1]{\S\ref{#1}}
\newcommand{\energy}[1]{\frac{1}{2} \int_{\Omega} \left\{ #1 \right\} \d\Omega}
\newcommand{\crmat}[1]{\ensuremath{\left[#1\right]_\times}}
\newcommand{\fenics}{\textsc{FEniCS}\xspace}
\newcommand{\firedrake}{\textsc{Firedrake}\xspace}

\DeclareMathOperator*{\argmax}{arg\,max}
\DeclareMathOperator*{\argmin}{arg\,min}

\newtheorem{proposition}{Proposition}
\newtheorem{proof}{Proof}
\newtheorem{remark}{Remark}
\newtheorem{hypothesis}{Hypothesis}
\newtheorem{assumption}{Assumption}
\newtheorem{conjecture}{Conjecture}


\def\onedot{$\mathsurround0pt\ldotp$}
\def\cddot{% two dots stacked vertically
	\mathbin{\vcenter{\baselineskip.67ex
			\hbox{\onedot}\hbox{\onedot}}%
}}

%\renewcommand\bibfont{\scriptsize}


\makeatletter \renewcommand\d[1]{\ensuremath{%
		\;\mathrm{d}#1\@ifnextchar\d{\!}{}}}
\makeatother


\title{A port-Hamiltonian formulation for the full von-K\'arm\'an plate model}

\author{{\underline{Andrea Brugnoli}}$^\ast$, Denis Matignon$^{\ast}$}

\address{$^\ast$ISAE-SUPAERO, Universit\'e de Toulouse, France 
}

\abstract{In this contribution, a port-Hamiltonian reformulation of the full von-K\'arm\'an  dynamical model for geometrically non-linear plates is detailed, including the collocated boundary control and observation. Starting from the canonical equations, a set of variables is chosen so as to  make the total energy quadratic. The model, reformulated in these variables, highlights a port-Hamiltonian structure ruled by a state-modulated interconnection operator. }

\begin{document}

\section{Classical model}

The classical full  von-K\'arm\'an dynamical model is presented in \cite{bilbao2015conservative}. The problem, defined on an open connected set $\Omega \subset \bbR^2$, takes the dimensionless form

\begin{equation}\label{eq:class}
\begin{aligned}
\ddot{\bm{u}} &= \Div{\bm{N}}, \\
\ddot{w} &= -\div\Div{\bm{M}} + \div{(\bm{N}\grad w)}, \\
\end{aligned} \qquad \qquad 
\begin{aligned}
\bm{N} &= \bm{\Phi} (\bm{\varepsilon}), \\
\bm{M} &= \bm{\Phi} (\bm{\kappa}), \\
\end{aligned} \qquad \qquad
\begin{aligned}
\bm{\varepsilon} &= \Grad \bm{u} + 1/2 \grad w \otimes \grad w, \\
\bm{\kappa} &= \Grad \grad w,
\end{aligned}
\end{equation}
where $\bm{u} \in \bbR^2$ is the in-plane displacement, $w$ is the vertical displacement, $\bm{\varepsilon}$ is the in-plane strain tensor, $\bm{\kappa}$ is the curvature tensor, $\bm{N}$ is the in-plane stress resultant and $\bm{M}$ is the bending stress resultant. The notation $\bm{a} \otimes \bm{b} = \bm{a} \bm{b}^\top$ denotes the dyadic product of two vectors.  The $\div$ operator  is the divergence of a vector field, and $\grad$ the gradient of a scalar field. The  operator $\mathrm{Grad} = \frac{1}{2} \left(\nabla + \nabla^\top \right)$ designates the symmetric part of the gradient (i. e. the deformation gradient in continuum mechanics). For a tensor field $\bm{U}: \Omega \rightarrow \bbR^{2\times 2}$, with components $U_{ij}$, the divergence $\Div(\bm{U})$ is a vector, defined column-wise as
\begin{equation*}
\Div(\bm U) := \sum_{i = 1}^2 \partial_{x_i} U_{ij}, \qquad \forall j = \{1, 2\}.
\end{equation*}
The linear tensor mapping $\bm{\Phi}$ is positive and preserves symmetry:
\begin{equation*}
	\bm\Phi (\bm{A}) = \nu \Tr(\bm{A})\bm{1} + (1 - \nu) \bm{A}, \qquad \bm{A} = \bm{A}^\top \implies \bm{\Phi}(\bm{A}) = \bm{\Phi}(\bm{A})^\top, \where \bm{1} = \mathrm{Diag}(1,1).
\end{equation*}

\begin{comment}
Its inverse is given by 
\begin{equation}
\bm{\Phi}^{-1}(\bm{A}) = \frac{1}{1 - \nu} \bm{A} - \frac{\nu}{1 - \nu^2} \Tr(\bm{A}) \bm{1}.
\end{equation}
\end{comment}

The total energy of the model (Hamiltonian functional)
\begin{equation}
	H = \frac{1}{2} \int_{\Omega}\left\{ \norm{\dot{\bm{u}}}^2 + \dot{w}^2 + \bm{N} \cddot \bm{\varepsilon} + \bm{M} \cddot \bm{\kappa} \right\} \d\Omega, \qquad \where \bm{A} \cddot \bm{B} = \Tr(\bm{A}^\top\bm{B})
\end{equation}
consists of the kinetic energy and both membrane and bending deformation energies.
This model proves conservative, see \cite{bilbao2015conservative}. Indeed, this implies that a port-Hamiltonian realization of the system exists. We shall demonstrate how to construct a port-Hamiltonian realization, equivalent to \eqref{eq:class}.


\section{The equivalent port-Hamiltonian system (pHs)}
To find a suitable port-Hamiltonian system, we first select a set of new energy variables to make the Hamiltonian functional quadratic. The selection is the same as for both the linear plate problems in \cite{brugnoli2019mindlin,brugnoli2019kirchhoff}:
\begin{equation}
\bm{\alpha}_u = \dot{\bm{u}}, \qquad \alpha_w = \dot{w}, \qquad \bm{A}_\varepsilon = \bm{\varepsilon}, \qquad \bm{A}_\kappa = \bm{\kappa}.
\end{equation}
The energy is quadratic in these variables
\begin{equation}
 H = \frac{1}{2} \int_{\Omega} \left\{\norm{\bm{\alpha}_u}^2 + \alpha_w^2 + \bm{\Phi}(\bm{A}_\varepsilon) \cddot \bm{A}_\varepsilon + \bm{\Phi}(\bm{A}_\kappa) \cddot \bm{A}_\kappa \right\}.
\end{equation}
By computing the variational derivative of the Hamiltonian, one obtains the so-called co-energy variables:
\begin{equation}
	\bm{e}_u := \delta_{\bm\alpha_u} H = \dot{\bm{u}}, \qquad \bm{e}_w := \delta_{\alpha_w} H = \dot{w}, \qquad \bm{E}_\varepsilon := \delta_{\bm{A}_\varepsilon} H = \bm{\Phi}(\bm{A}_\varepsilon), \qquad \bm{E}_\kappa := \delta_{\bm{A}_\kappa} H = \bm{\Phi}(\bm{A}_\kappa).
\end{equation}
Before stating the final formulation, consider the operator $\mathcal{C}(w)(\cdot): L^2(\Omega, \bbR^{2\times 2}_{\text{sym}}) \rightarrow L^2(\Omega)$ acting on symmetric tensors 
\begin{equation}
	\mathcal{C}(w)(\bm{T}) = \div(\bm{T} \grad w).
\end{equation}
\begin{proposition}\label{prop:adjC}
The formal adjoint of the $\mathcal{C}(w)(\cdot)$ is given by
\begin{equation}
\mathcal{C}(w)^*(\cdot) = -\frac{1}{2} \left[\grad (\cdot) \otimes \grad(w) + \grad(w) \otimes \grad(\cdot) \right] .
\end{equation}
\end{proposition}
\begin{proof}
Consider a smooth scalar field  $v \in C^\infty_0(\Omega)$ and a smooth symmetric tensor field $\bm{U} \in C^\infty_0(\Omega, \bbR^{2\times 2}_{\text{sym}})$ with compact support. The formal adjoint of $\mathcal{C}(w)(\cdot)$ satisfies the relation
\begin{equation}
\inner[L^2(\Omega)]{v}{\mathcal{C}(w)(\bm{U})} = \inner[L^2(\Omega, \bbR^{2\times 2}_{\text{sym}})]{\mathcal{C}(w)(v)^*}{\bm{U}}.
\end{equation}
The proof follows from the computation
\begin{equation}
\begin{aligned}
\inner[L^2(\Omega)]{v}{\mathcal{C}(w)(\bm{U})} &= \inner[L^2(\Omega)]{v}{\div(\bm{U} \grad w)}, \qquad \text{Integration by parts,}\\
&= \inner[L^2(\Omega, \bbR^2)]{-\grad v}{\bm{U} \grad w}, \qquad \text{Dyadic product properties,}\\
&= \inner[L^2(\Omega, \bbR^{2\times 2}_{\text{sym}})]{-\grad v \otimes \grad w}{\bm{U}}, \qquad \text{Symmetry of $\bm{U}$,}\\
&= \inner[L^2(\Omega, \bbR^{2\times 2}_{\text{sym}})]{-1/2(\grad v \otimes \grad w + \grad w \otimes \grad v)}{\bm{U}}.
\end{aligned} 
\end{equation}
This means
\begin{equation}
\mathcal{C}(w)^*(\cdot) = -\frac{1}{2} \left[\grad (\cdot) \otimes \grad(w) + \grad(w) \otimes \grad(\cdot) \right],
\end{equation}
leading to the final result.
\end{proof}

The pH realization is then given by the following system
\begin{equation}\label{eq:pHsys}
  %\diffp{}{t}
   \frac{\partial}{\partial t}
\begin{pmatrix}
\bm{\alpha}_u \\
\bm{A}_\varepsilon \\
\alpha_w \\
\bm{A}_\kappa
\end{pmatrix} = 
\begin{bmatrix}
\bm{0} & \Div & \bm{0} & \bm{0} \\
\Grad & \bm{0} & -\mathcal{C}(w)^* & \bm{0} \\
0 & \mathcal{C}(w) & 0 & -\div\Div \\
\bm{0} & \bm{0} & \Grad\grad & \bm{0} \\ 
\end{bmatrix}
\begin{pmatrix}
\delta_{\bm\alpha_u} H \\
\delta_{\bm{A}_\varepsilon} H \\
\delta_{\bm\alpha_w} H \\
\delta_{\bm{A}_\kappa} H
\end{pmatrix},
\end{equation}
The second line of system \eqref{eq:pHsys} represents the time derivative of the membrane strain tensor. To close the system, variable $w$ has to be accessible. For this reason, its dynamics has to be included. The augmented system reads
\begin{equation}\label{eq:pHsys_aug}
%\diffp{}{t}
  \frac{\partial}{\partial t}
  \begin{pmatrix}
\bm{\alpha}_u \\
\bm{A}_\varepsilon \\
w \\
\alpha_w \\
\bm{A}_\kappa
\end{pmatrix} = 
\underbrace{\begin{bmatrix}
\bm{0} & \Div & \bm{0} & \bm{0} & \bm{0}\\
\Grad & \bm{0} & \bm{0} & -\mathcal{C}(w)^* & \bm{0} \\
0 & 0 & 0 & 1 & 0 \\
0 & \mathcal{C}(w) & -1 & 0 & -\div\Div \\
\bm{0} & \bm{0} & \bm{0} & \Grad\grad & \bm{0} \\ 
\end{bmatrix}}_{\mathcal{J}}
\begin{pmatrix}
\delta_{\bm\alpha_u} H \\
\delta_{\bm{A}_\varepsilon} H \\
\delta_{w} H \\
\delta_{\alpha_w} H \\
\delta_{\bm{A}_\kappa} H
\end{pmatrix}.
\end{equation}
Given the results in \cite{brugnoli2019mindlin,brugnoli2019kirchhoff} and Proposition \ref{prop:adjC}, the operator $\mathcal{J}$ is formally skew-adjoint. If only the kinetic and deformation energies are considered, it holds $\delta_{w} H=0$. In general this terms allows accommodating other potentials, for example the gravitational one. Suitable boundary variables are then obtained considering the power balance
\begin{equation}
\dot{H} = \inner[\partial\Omega]{\gamma_0\bm{e}_u}{\gamma_\perp \bm{E}_\varepsilon} + \inner[\partial\Omega]{\gamma_0\bm{e}_w}{\gamma_{\perp\perp, 1} \bm{E}_\kappa + \gamma_0(\bm{E}_\varepsilon \bm{n} \cdot \grad w)} + \inner[\partial\Omega]{\gamma_1\bm{e}_w}{\gamma_{\perp\perp} \bm{E}_\kappa},
\end{equation}
where $\gamma_0\bm{e}_u= \bm{e}_u\vert_{\partial\Omega}$ is the Dirichlet trace, $\gamma_\perp \bm{E}_\varepsilon = \bm{E}_\varepsilon \bm{n}\vert_{\partial\Omega}$ is the normal trace ($\bm{n}$ is the outward normal vector), $\gamma_{\perp\perp, 1} \bm{E}_\kappa = - \bm{n} \cdot \Div \bm{E}_\kappa - \partial_{\bm{s}}(\bm{n}^\top \bm{E}_\kappa \bm{s})\vert_{\partial\Omega}$ is the effective shear force at the boundary ($\bm{s}$ is the tangent versor at the boundary), $\gamma_1 \bm{e}_w = \partial_{\bm{n}} \bm{e}_w\vert_{\partial\Omega}$ is the normal derivative trace and $\gamma_{\perp\perp} \bm{E}_\kappa = \bm{n}^\top \bm{E}_\kappa \bm{n}$ is the normal to normal trace. The boundary conditions are consistent with the ones assumed in \cite{puel1996} for deriving a global existence result for this model.



\section{Conclusions}
We have presented a pHs formulation of the full von-K\'arm\'an model. The dynamics of the system exhibits a state modulated interconnection operator, while the energy remains quadratic in the chosen variables. Of particular interest is the discretization of such a model for simulation and control purposes. %
The Partitioned Finite Element Method (PFEM), an extension of mixed finite elements to pHs,  seems to be particularly suitable to achieve a structure-preserving discretization of this model, as in \cite{cardoso2020pfem} for the 2D  Shallow Water Equation, which exhibits the same kind of polynomial nonlinearity.
 
\footnotesize{\bibliography{biblio_ENOC}}

 \end{document}
