\documentclass[11t]{article}
\usepackage[T1]{fontenc}
\usepackage[utf8]{inputenc}
\usepackage[cyr]{aeguill}
\usepackage[english]{babel}
\usepackage{amsmath,amssymb,amsthm}
\usepackage{bm}
\usepackage{graphicx}

\usepackage{authblk}	
\usepackage{geometry}
\geometry{top=3cm,bottom=3cm,left=2.5cm,right=2.5cm}

\usepackage{mathrsfs}

\usepackage{multirow}
\usepackage[justification=centering]{caption}
\usepackage{subfig}
\usepackage{xcolor,colortbl}
\usepackage{diffcoeff}

\usepackage[colorlinks=true,linkcolor=black, citecolor=blue, urlcolor=blue]{hyperref}

\DeclareMathOperator*{\argmax}{arg\,max}
\DeclareMathOperator*{\argmin}{arg\,min}
\DeclareMathOperator{\Tr}{Tr}
\DeclareMathOperator*{\grad}{grad}
\DeclareMathOperator*{\Grad}{Grad}
\DeclareMathOperator*{\dive}{div}
\DeclareMathOperator*{\Div}{Div}

\newtheorem{theorem}{Theorem}
\newtheorem{remark}{Remark}
\newtheorem{proposition}{Proposition}
\newtheorem{definition}{Definition}
\newtheorem{corollary}{Corollary}

\graphicspath{{./Figures/}}

\def\onedot{$\mathsurround0pt\ldotp$}
\def\cddot{% two dots stacked vertically
	\mathbin{\vcenter{\baselineskip.67ex
			\hbox{\onedot}\hbox{\onedot}}%
}}

\makeatletter \renewcommand\d[1]{\ensuremath{%
		\;\mathrm{d}#1\@ifnextchar\d{\!}{}}}
\makeatother

\title{Port-Hamiltonian Formulation of Thermoelasticity}	

\author[1]{Andrea Brugnoli\thanks{andrea.brugnoli@isae.fr}}
\author[1]{Daniel Alazard\thanks{daniel.alazard@isae.fr}}
\author[1]{Val\'erie Pommier-Budinger \thanks{valerie.budinger@isae.fr}}
\author[1]{Denis Matignon \thanks{denis.matignon@isae.fr}}
\affil[1]{ISAE-SUPAERO, Universit\'e de Toulouse, France. \\
	10 Avenue Edouard Belin, BP-54032, 31055 Toulouse Cedex 4.}

\begin{document}
\maketitle

\section{Classical model}
The classical model is detailed in \cite{Carlson1973}. The equations read
\begin{equation}
\begin{cases}
\displaystyle \rho \diffp[2]{\bm{u}}{t} = \Div(\mathcal{S}_{\mathcal{E}T}) , \\
\displaystyle \rho c_V \diffp{T}{t} = -\dive(\bm{j}_Q) - T_0 \mathcal{C} \mathcal{A} \cddot \displaystyle\diffp{\mathcal{E}}{t}, \\
\mathcal{S}_{\mathcal{E}T} = \mathcal{S_E} + \mathcal{S}_{T}, \\
\mathcal{S_E} = \mathcal{C} \mathcal{E}, \\
\mathcal{S}_{T} = - \mathcal{C} \mathcal{A} (T - T_0),  \\
\mathcal{E} = \Grad(\bm{u}), \\
\bm{j}_Q = -\mathcal{K} \, \mathrm{grad}(T).
\end{cases}
\end{equation}

Here $\rho, c_v$ are the mass density and the specific heat density at constant strain, $\bm{u}$ is the displacement, $T, T_0$ are the temperature and the constant reference temperature, $\mathcal{E}$ is the strain tensor, $\mathcal{S}_E, \mathcal{S}_T$ are the stress tensor contribution due to mechanical deformation and a thermal field, $\bm{j}_Q$ is the heat flux. $\mathcal{C}$ is the elasticity tensor, $\mathcal{K}$ the thermal diffusivity tensor and $\mathcal{A}$ the thermal expansion tensor. 

\section{PH Elasticity}
The port-Hamiltonian elasticity formulation reads

\begin{equation}
\begin{aligned}
\diffp{}{t}
\begin{pmatrix}
\bm{p} \\
\mathcal{E} \\
\end{pmatrix} &= 
\begin{bmatrix}
0 & \Div \\
\Grad & 0 \\
\end{bmatrix}
\begin{pmatrix}
\bm{v} \\
\mathcal{S_E} \\
\end{pmatrix} + 
\begin{bmatrix}
\bm{I} \\
0 \\
\end{bmatrix}\bm{u}_E, \\
\bm{y}_E &= \begin{bmatrix}
\bm{I} & 0 \\
\end{bmatrix}\begin{pmatrix}
\bm{v} \\
\mathcal{S_E} \\
\end{pmatrix}
\end{aligned}
\end{equation} 

where $\bm{v}=\diffp{\bm{u}}{t}, \; \bm{p} = \rho \bm{v}$ and $\bm{u}_E$ is a distributed force  in the domain $\Omega$, meaning that 
\begin{equation}
\begin{pmatrix}
\bm{v} \\
\mathcal{S_E} \\
\end{pmatrix}= 
\begin{bmatrix}
\rho^{-1} & 0\\
0 & \mathcal{C} \\
\end{bmatrix}
\begin{pmatrix}
\bm{p} \\
\mathcal{E} \\
\end{pmatrix}.
\end{equation}

\section{PH Heat equation}
Taking as Hamiltonian the Lyapunov function 
\begin{equation*}
H = \frac{1}{2} \int_{\Omega} \rho c_v \frac{T^2}{T_0} \d\Omega.
\end{equation*}
This Lyapunov function doesn't represent the actual internal energy. Nevertheless it has the dimension of an energy. From this functional can be equivalently written introducing as energy variable the actual internal energy
\begin{equation}
	\alpha_u := u =  \rho c_v T,
\end{equation}
and the corresponding coenergy 
\begin{equation}
	e_u := \diffd{H}{\alpha_u} = \frac{T}{T_0}.
\end{equation}
It is then possible to express the heat equation as a pHDAE
\begin{equation}
\begin{aligned}
\begin{pmatrix}
1 & 0 \\
0 & 0 \\
\end{pmatrix}
\diffp{}{t}
\begin{pmatrix}
\alpha_u \\
\bm{j}_Q \\
\end{pmatrix} &= 
\begin{bmatrix}
0 & -\dive \\
-\grad & - (T_0 \mathcal{K})^{-1} \\
\end{bmatrix}
\begin{pmatrix}
e_u \\
\bm{j}_Q \\
\end{pmatrix} + 
\begin{bmatrix}
1 \\
0 \\
\end{bmatrix}u_T, \\
y_T &= \begin{bmatrix}
1 & 0 \\
\end{bmatrix}\begin{pmatrix}
e_u \\
\bm{j}_Q \\
\end{pmatrix}
\end{aligned}
\end{equation} 
The input $u_T$ represents a heat source. The operator 
\[
\mathfrak{A} = \begin{bmatrix}
0 & -\dive \\
-\grad & - (T_0 \mathcal{K})^{-1} \\
\end{bmatrix}
\]
can be decomposed in a skew-symmetric and a non negative definite symmetric part
\[
\mathfrak{J} = \begin{bmatrix}
0 & -\dive \\
-\grad & 0 \\
\end{bmatrix}, \qquad
\mathfrak{R} = \begin{pmatrix}
0 & 0 \\
0 & (T_0 \mathcal{K})^{-1} \\
\end{pmatrix}
\]
The first one accounts for the energy conservation, the second one for the dissipation.

\section{Coupling}
The linear thermoelastic problem can be expressed as a coupled port-Hamiltonian system.  Consider the following power preserving interconnection

\begin{align}
\bm{u}_E &= - \Div(T_0 \, \mathcal{C A} \, y_T) \\
u_T &= - T_0 \, \mathcal{C A} \cddot\Grad(\bm{y}_E) 
\end{align}

Using the expression of $y_T, \bm{y}_E$, considering that $T_0$ is constant in space and time and applying Schwarz theorem for smooth function, the inputs are equal to

\begin{align*}
\bm{u}_E &=  \Div(\mathcal{S}_T) \\
u_T &= - T_0 \, \mathcal{C A} \cddot  \Grad(\bm{v}) = - T_0 \, \mathcal{C A} \cddot  \diffp{\mathcal{E}}{t} 
\end{align*}
The connection is power preserving as the interconnection can be compactly written as 
\begin{align}
\bm{u}_E &= \mathfrak{D}(y_T) \\
u_T &= - \mathfrak{D}^*(\bm{y}_E) 
\end{align}
were the operator $\mathfrak{D} := - \Div(T_0 \, \mathcal{C A} \, \cdot) : L^2(\Omega) \rightarrow L^2(\Omega, \mathbb{R}^3)$ has formal adjoint $ \mathfrak{D}^* =  T_0 \, \mathcal{C A} \cddot  \Grad(\cdot)$. Consider the inner product of

\bibliographystyle{unsrt}
\bibliography{thermoelasticity} 
	
\end{document}
\endinput