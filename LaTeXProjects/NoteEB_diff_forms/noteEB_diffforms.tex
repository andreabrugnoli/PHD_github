\documentclass{article}
\usepackage{amsmath,amssymb}
\usepackage[left=2cm, right=2cm]{geometry}
\usepackage{bm}
\usepackage{diffcoeff}
\usepackage{scrextend}
\changefontsizes[14pt]{12pt}

\usepackage{natbib}



\DeclareMathOperator*{\grad}{grad}
\DeclareMathOperator*{\Grad}{Grad}
\DeclareMathOperator*{\Div}{Div}
\renewcommand{\div}{\operatorname{div}}
\DeclareMathOperator*{\Hess}{Hess}
\DeclareMathOperator*{\curl}{curl}
\DeclareMathOperator{\Tr}{Tr}

\newtheorem{proof}{Proof}

\newcommand{\inner}[3][]{\ensuremath{\left\langle #2, \, #3 \right\rangle_{#1}}}
\newcommand{\energy}[1]{\frac{1}{2} \int_{\Omega} \left\{ #1 \right\} \d\Omega}

\makeatletter \renewcommand\d[1]{\ensuremath{%
		\;\mathrm{d}#1\@ifnextchar\d{\!}{}}}
\makeatother

% inner products
\def\onedot{$\mathsurround0pt\ldotp$}
\def\cddot{% two dots stacked vertically
	\mathbin{\vcenter{\baselineskip.67ex
			\hbox{\onedot}\hbox{\onedot}}%
}}

\title{The Euler-Bernoulli beam in differential form}
\author{Andrea Brugnoli}
\begin{document}
	\maketitle
	
\section{Classical formulation}

The Euler-Bernoulli beam model consists of one PDE, describing the vertical displacement along the beam length:
\begin{equation}
\rho A(x) \diffp[2]{w}{t}(x,t) + \displaystyle \diffp[2]{}{x} \left( EI(x) \diffp[2]{w}{x} \right) = 0, \quad x \in \Omega=\{0, L\},
\end{equation}
where ${w}(x,t)$ is the transverse displacement of the beam. The coefficients $\rho(x), A(x) E(x)$ and $I(x)$  are the mass density, cross section, Young's modulus of elasticity and the moment of inertia of a cross section. The energy variables are then chosen as follows:
\begin{equation}\label{eq:alpha_EB}
\alpha_{w}= \rho A(x) \diffp{w}{t}(x,t),  \quad \text{Linear Momentum},\quad
\alpha_{\kappa} = \diffp[2]{w}{x}(x,t), \quad \text{Curvature}. \
\end{equation}

Those variables are collected in the vector $\bm{\alpha} = (\alpha_{w}, \, \alpha_{\kappa})^T $, so that the Hamiltonian can be written as a quadratic functional in the energy variables: 
\begin{equation}
H = \energy{\frac{1}{\rho A} \alpha_{w}^2 + EI \alpha_\kappa^2}
\end{equation}

The co-energy variables are found by computing the variational derivative of the Hamiltonian:
\begin{equation}\label{eq:e_EB}
\begin{aligned}
e_{w} &:= \diffd{H}{\alpha_w} = \diffp{w}{t}(x,t) ,  &\quad \text{Vertical velocity}, \\
e_{\kappa} &:= \diffd{H}{\alpha_{\kappa}} =EI(x) \diffp[2]{w}{x}(x,t),  &\quad \text{Flexural momentum}. \\
\end{aligned}
\end{equation}
The underlying interconnection structure is then found to be:
\begin{equation}\label{eq:pHsysEB}
\diffp{}{t}\begin{pmatrix}
\alpha_w \\ \alpha_\kappa \\
\end{pmatrix} = \begin{bmatrix}
0 & -\partial_{xx} \\
\partial_{xx} & 0 \\
\end{bmatrix} \begin{pmatrix}
e_w \\ e_\kappa \\
\end{pmatrix}.
\end{equation}

The power flow gives access to the boundary variables:
\begin{equation}\label{eq:powrateEB}
\begin{aligned}
\dot{H} &= \int_{\Omega} \left\{e_w \partial_t \alpha_w + e_\kappa \partial_t \alpha_\kappa \right\}  \d{\Omega}, \\
&= \int_{\Omega} \left\{-e_w \partial_{xx}e_\kappa + e_\kappa \partial_{xx} e_w \right\} \d{\Omega}, \qquad \text{Integration by parts}, \\
&= \int_{\partial\Omega}\{-e_w \partial_{x}e_\kappa + e_\kappa \partial_{x} e_w \} \d{s}= \inner[\partial\Omega]{-e_w}{\partial_{x}e_\kappa} + \inner[\partial\Omega]{e_\kappa}{\partial_{x}e_w}
\end{aligned}
\end{equation}
Since the system is of differential order two, two pairing appears, giving rise to four combination of uniform boundary causality
\begin{itemize}
	\item {First case} ${u}_{\partial, 1} = e_w, \quad {u}_{\partial, 2} = \partial_x e_w, \quad {y}_{\partial, 1} = -\partial_{x}e_\kappa, \quad {y}_{\partial, 2} = e_\kappa$. \\
	This imposes the vertical $e_w:= \partial_t w$ and angular velocity $\partial_x e_w:= \partial_{xt} w$ as boundary inputs. If the inputs are null a clamped boundary condition is obtained.
	\item {Second case} ${u}_{\partial, 1} = e_w, \quad {u}_{\partial, 2} = e_\kappa, \quad {y}_{\partial, 1} = -\partial_{x}e_\kappa, \quad {y}_{\partial, 2} = \partial_x e_w$. \\
	This imposes the vertical velocity and flexural momentum $e_\kappa:= EI \partial_{xx} w$ as boundary inputs.  Zero inputs lead to a simply supported condition is found.
	\item {Third case} ${u}_{\partial, 1} = -\partial_{x}e_\kappa , \quad {u}_{\partial, 2} = e_\kappa, \quad {y}_{\partial, 1} = e_w, \quad {y}_{\partial, 2} = \partial_x e_w$. \\
	This imposes the shear force $\partial_{x}e_\kappa:= \partial_{x}(EI \partial_{xx}w)$ and flexural momentum as boundary inputs. Null inputs correspond to a free condition.
	\item {Forth case} ${u}_{\partial, 1} = -\partial_{x}e_\kappa , \quad {u}_{\partial, 2} = \partial_{x}e_w, \quad {y}_{\partial, 1} = e_w, \quad {y}_{\partial, 2} = e_\kappa$. \\
	This imposes the shear force and angular velocity as boundary inputs.
\end{itemize} 

\section{Differential forms formulation}

The co-energy now are 0-forms $e_w, e_\kappa \in \Lambda^0(\Omega)$ whereas the flows are 1-forms $f_w= \partial_t \alpha_w, f_\kappa= \partial_t \alpha_\kappa \in \Lambda^1(\Omega)$. To recast \eqref{eq:pHsysEB} using the exterior derivative, the Hodge star operator is needed.
\begin{equation}
	*: \Lambda^k(\Omega) \rightarrow \Lambda^{n-k}(\Omega), \qquad \Omega \subset \mathbb{R}^n.
\end{equation}
For one dimensional domains $\Omega \subset \mathbb{R}$ and using Euclidian coordinates, this operator can be either used on 1-forms or 0-forms
\begin{equation}
\begin{aligned}
*: \Lambda^1(\Omega) &\longrightarrow \Lambda^0(\Omega), \\
 f(x)\d{x} &\longrightarrow f(x)
\end{aligned}
\end{equation}
 or 
\begin{equation}
\begin{aligned}
*: \Lambda^0(\Omega) &\longrightarrow \Lambda^1(\Omega), \\
f(x) &\longrightarrow f(x)\d{x}
\end{aligned}
\end{equation}
Then the equivalent system reads 

\begin{equation}\label{eq:pHsysEB_diff}
\diffp{}{t}\begin{pmatrix}
\alpha_w \\ \alpha_\kappa \\
\end{pmatrix} = \begin{bmatrix}
0 & -\mathrm{d}{*}\mathrm{d}\\
\mathrm{d}{*}\mathrm{d} & 0 \\
\end{bmatrix} \begin{pmatrix}
e_w \\ e_\kappa \\
\end{pmatrix}.
\end{equation}
\begin{proof}
	The operator $\d * \d : \Lambda^0(\Omega) \rightarrow \Lambda^1(\Omega)$ is a composition of operators that reads in Euclidean coordinates
	\begin{equation}
		\begin{aligned}
		\mathrm{d}{*}\mathrm{d} e &= \mathrm{d}{*} (\diffp{e}{x} \d{x}), \qquad e \in \Lambda^0(\Omega)\\
		         &= \d (\diffp{e}{x}), \\
		         &= \diffp[2]{e}{x} \d{x} \in \Lambda^1(\Omega) \\
		\end{aligned}
	\end{equation}
\end{proof}
The Hamiltonian energy is then 
\begin{equation}
\begin{aligned}
H = \frac{1}{2}  \int_{\Omega} e_w \wedge \alpha_w   + e_\kappa \wedge \alpha_\kappa
\end{aligned}
\end{equation}
To find the appropriate power balance, consider the integration by parts formula for smooth differential forms $\lambda \in \Lambda^k(\Omega)$ and $\mu \in \Lambda^{n-k-1}(\Omega)$,
\begin{equation}
	 \inner{\mathrm{d}{\lambda}}{\mu} = \inner{\Tr\lambda}{\Tr \mu} - (-1)^k \inner{\lambda}{\mathrm{d}\mu}
\end{equation}
For uni-dimensional domains $n=1$, and we take $\lambda \in \Lambda^0(\Omega)$ and $\mu \in \Lambda^{0}(\Omega)$, implying
\begin{equation}
\begin{aligned}
\inner{\mathrm{d}{\lambda}}{\mu} = \inner{\Tr\lambda}{\Tr \mu} - \inner{\lambda}{\mathrm{d}\mu}, \\
\inner{\lambda}{\mathrm{d}\mu} = \inner{\Tr\lambda}{\Tr \mu} - \inner{\mathrm{d}\lambda}{\mu}.
\end{aligned}
\end{equation}
Then we can state
\begin{equation}\label{eq:int1}
\inner{-e_w}{\mathrm{d}({*}\mathrm{d} e_\kappa)} = \inner{-\Tr e_w}{\Tr({*}\mathrm{d} e_\kappa)} + \inner{\mathrm{d} e_w}{{*}\mathrm{d} e_\kappa}.
\end{equation}
and
\begin{equation}\label{eq:int2}
	\inner{e_\kappa}{\mathrm{d}({*}\mathrm{d} e_w)} = \inner{\Tr e_\kappa}{\Tr ({*}\mathrm{d} e_w)} - \inner{\mathrm{d} e_\kappa}{{*}\mathrm{d} e_w},
\end{equation}

The power rate then reads 
\begin{equation}
\begin{aligned}
\dot{H} &= \int_{\Omega} e_w \wedge \partial_t \alpha_w   + e_\kappa \wedge \partial_t \alpha_\kappa, \\
&= \int_{\Omega} e_w \wedge \partial_t \alpha_w   + e_\kappa \wedge \partial_t \alpha_\kappa, \\
&=\int_{\Omega} -e_w \wedge (\mathrm{d}{*}\mathrm{d} e_\kappa)   + e_\kappa \wedge (\mathrm{d}{*}\mathrm{d} e_w), \qquad \text{From \eqref{eq:int1}, \eqref{eq:int2}}\\
&=\inner{-\Tr e_w}{\Tr({*}\mathrm{d} e_\kappa)} + \inner{\mathrm{d} e_w}{{*}\mathrm{d} e_\kappa} + \inner{\Tr e_\kappa}{\Tr ({*}\mathrm{d} e_w)} - \inner{\mathrm{d} e_\kappa}{{*}\mathrm{d} e_w}
\end{aligned}
\end{equation}

The wedge product is such that for $\lambda \in \Lambda^k(\Omega)$ and $\mu \in \Lambda^l(\Omega)$ it holds
\begin{equation}
\lambda \wedge \mu = (-1)^{kl} \mu \wedge \lambda
\end{equation}
Furthermore, the Hodge star is such that 
\begin{equation}
(\alpha, \beta) := \inner{\alpha}{*\beta} =  \inner{*\alpha}{\beta} = (\beta, \alpha), \qquad \alpha, \beta \in \Lambda^k(\Omega)
\end{equation}

Then for $e_w, e_\kappa \in \Lambda^0$ it holds
\begin{equation}
\begin{aligned}
\inner{\mathrm{d} e_w}{{*}\mathrm{d} e_\kappa} &= \inner{{*} \mathrm{d} e_w}{\mathrm{d} e_\kappa}, \\
 &= \int_{\Omega} ({*} \mathrm{d} e_w) \wedge (\mathrm{d} e_\kappa), \\
 &= \int_{\Omega} (\mathrm{d} e_\kappa) \wedge ({*} \mathrm{d} e_w), \\
 & = \inner{\mathrm{d} e_\kappa}{{*} \mathrm{d} e_w}
\end{aligned}
\end{equation}
It can be then stated
\begin{equation}
 \inner{\mathrm{d} e_w}{{*}\mathrm{d} e_\kappa} - \inner{\mathrm{d} e_\kappa}{{*}\mathrm{d} e_w} = 0
\end{equation}
The power balance is then
\begin{equation}
\dot{H}=\inner{-\Tr e_w}{\Tr({*}\mathrm{d} e_\kappa)}+ \inner{\Tr e_\kappa}{\Tr ({*}\mathrm{d} e_w)}
\end{equation}
In vector calculus notation it reads
\begin{equation}
\dot{H}=\inner{-\Tr e_w}{\Tr\diffp{e_\kappa}{x}}+ \inner{\Tr e_\kappa}{\Tr \diffp{e_w}{x}} = \inner[\partial\Omega]{-e_w}{\partial_{x}e_\kappa} + \inner[\partial\Omega]{e_\kappa}{\partial_{x}e_w}
\end{equation}
Completely equivalent to Eq. \eqref{eq:powrateEB}.
\nocite*
\bibliographystyle{unsrt}
\bibliography{biblio} 
\end{document}


