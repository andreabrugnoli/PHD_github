\documentclass[11pt]{letter}
\usepackage[utf8]{inputenc} % un package
\usepackage[T1]{fontenc} % un second package
\usepackage[english]{babel} % un troisième package
\usepackage{textcomp}
\usepackage{comment}
\usepackage{ifpdf}
\ifpdf
\usepackage[pdftex]{graphicx}
\else
\usepackage[dvips]{graphicx}\fi
\pagestyle{empty}
\usepackage[top=0.5cm, bottom=0.5cm, left=1.5cm, right=1.5cm]{geometry}
\setlength{\parindent}{0pt}
\addtolength{\parskip}{6pt}
\renewcommand{\ttdefault}{pcr}
\begin{document}
	\sffamily
	%\hfill
	%
	\begin{flushleft}
		{\bfseries Andrea \textsc{BRUGNOLI}}\\[.35ex]
		% \small\itshape
		% 1 Avenue de Rangueil\\
		% 31400, Toulouse\\[.35ex]
		% 0033 7 50 39 47 27 \\
		Andrea.BRUGNOLI@utwente.nl ou andrea.brugnoli92@gmail.com
	\end{flushleft}
	%
	\begin{flushright}
		{\bfseries BERGISCHE UNIVERSITÄT WUPPERTAL}\\[.35ex]
		\small\itshape
		Gau{\ss}stra{\ss}e 20,  \\
		42119 Wuppertal, Germany
	\end{flushright}
	%
	%\hfill
	%
	\begin{flushright}
		Enschede (NL), \today 
	\end{flushright}
	%
	\textbf{Subject}: Cover letter for the vacancy\\
	\textit{Assistant Professorship (W 1) for “Port-Hamiltonian Systems”}
	
	
	To the attention of the recruitment committee.\\
	
	The ever-increasing number of high-impact publications, the expertise of the research staff and the numerous collaborations with industrial and academic partners are undeniable elements of the Bergische Univerisit\"{a}t Wuppertal's reputation. The university, whose image and prestige are recognized worldwide, has always been distinguished by its strong participation in projects with a significant scientific impact. The Institute of Mathematical Modelling, Analysis and Computational Mathematics (IMACM) seeks to improve the understanding required to tackle the many technological, societal and economical challenges of the present and future. Fundamental and applied mathematics are equally important to achieve this goal. Being able to contribute to such far-reaching projects and being surrounded by a stimulating, international and intensely committed environment are a reason of great enthusiasm for me.
	
	I studied mechanical and space engineering during my four years of study at Politecnico di Milano. After my fourth year, I enrolled in a double degree at ISAE-SUPAERO, where I specialized in applied mathematics and advanced automatic control. My strong passion for this latter subject encouraged me to enroll in a Research Master organized by the University Paris-Saclay. In November 2020, I obtained my thesis, titled \textit{A port-Hamiltonian formulation of flexible structures. Modelling and structure-preserving finite element discretization}. This work, was advised by Daniel Alazard, Valérie Pommier-Budinger and Denis Matignon. The thesis was focused on the development of discretization paths preserving the physical structure for the dynamics of thin flexible structures. It is an interdisciplinary subject, at the intersection between system theory, mechanics, scientific computing and numerical analysis. During my thesis, I was able to do an exchange with the Instituto Tecnológico de Aeronáutica (ITA) in Brazil, to work with prof. Cardoso-Ribeiro. Our collaboration continues to this day, with the supervision of an ITA student in double degree at the University of Twente. After my PhD thesis I joined as a post-Doc the Portwings project, funded by an ERC advanced grant and leaded by prof. Stefano Stramigioli. My mission is to set up the numerical algorithms for the simulation of the flapping flight of birds, which represents an extremely complicated example of fluid-structure interaction.
	
	My research is situated between scientific computing and dynamical systems. I wish to continue my work on the structured discretization of port-Hamiltonian systems, to develop a computational library for multiphysics problems. I also plan to deepen my knowledge in model reduction methods. This research topic aims at obtaining reduced models of small dimension, able to retain the dominant dynamics of the problem. This step allows to get rid of extremely expensive simulations when dealing with optimization, real-time control or uncertainty quantification. In my future research I want to explore the preservation of physical invariants in the reduction process. My long-term plan is to develop a computational project to provide a unifying framework for modeling, discretizing and reducing physical models used in engineering. The goal is to improve the industrial status quo for the design of mechanical structures. I invite you to consult the attached file for more details on my research project. 
	
 
	I remain at your disposal for any further information. 
	
	
	
	
	\begin{center}
		\large\textit{Andrea Brugnoli}
	\end{center}
\end{document}