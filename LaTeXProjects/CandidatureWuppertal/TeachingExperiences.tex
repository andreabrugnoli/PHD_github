\documentclass[12pt]{article}
\usepackage{fontspec}
\setmainfont{Arial}
\usepackage{wrapfig}
\usepackage{babel}
\usepackage{multirow}

\usepackage{setspace}
\singlespacing

% normal box
\newcommand{\sqboxs}{1.2ex}% the square size
\newcommand{\sqboxf}{0.6pt}% the border in \sqboxEmpty
\newcommand{\sqbox}[1]{\textcolor{#1}{\rule{\sqboxs}{\sqboxs}}}

%\usepackage[none]{hyphenat}
%\usepackage{hyphenat}

\usepackage{graphicx}
\usepackage{caption}
%\usepackage[T1]{fontenc}
%\usepackage[utf8]{inputenc}
\usepackage{lmodern}
\usepackage{geometry}
\geometry{
	a4paper,
	left=20mm,
	right=20mm,
	top=25mm,
	bottom=25mm,
}

\usepackage{pgfgantt}
\usepackage{eurosym}

\usepackage{pdfpages}


\usepackage{subcaption}

\usepackage[unicode=true,pdfusetitle,bookmarks=true,bookmarksnumbered=false,bookmarksopen=false,
breaklinks=false,pdfborder={0 0 0},backref=false,colorlinks=true,urlcolor=magenta]{hyperref}


\graphicspath{{images/}}

\author{Andrea Brugnoli \\ 
	\hspace{2.8pt} Docteur ISAE-SUPAERO 2020\\
	Ingénieur ISAE-SUPAERO 2017}
\title{Teaching experiences \& Scientific activities}

\date{}

\begin{document}
	
	\maketitle
	
	
	\thispagestyle{empty}
	
	\tableofcontents
	
	
	\section{Teaching activities}
	
	During my thesis, I carried out my teaching activities at the \textit{Institut Supérieur de l'Aéronautique et de l'Espace} (ISAE-SUPAERO), either for the french engineering training or for the international masters. I have taught courses in Automatic Control for 4th year students in the french engineering diploma (state representation, frequency control, root locus, estimation, identification, co-located output feedback, supervision of the student project), and in the international master "Aerospace systems and control" (discrete systems, Z-transform, student project \textit{identification and control of flexible structures}). I have also taught classes in Applied Mathematics for 3rd year students in the french engineering diploma (numerical resolution of PDEs, finite differences, finite elements, numerical methods for convex optimization). The Table \ref{tab:teaching} summarizes the different activities in terms of time volume.
	
	
	\begin{table}[h]
		\centering
		\begin{tabular}{p{\dimexpr.15\linewidth-2\tabcolsep}p{\dimexpr.15\linewidth-2\tabcolsep}p{\dimexpr.15\linewidth-2\tabcolsep}p{\dimexpr.4\linewidth-2\tabcolsep}p{\dimexpr.15\linewidth-2\tabcolsep}}
			\hline
			Year & Level & Class Type  & Subject & Duration  \\
			\hline
			\multirow{2}{*}{2019-2020} & 3rd year & Coding Lab &  Numerical solution of PDEs & 6h \\
			& 3rd year & Coding Lab &  Optimization & 6h \\
			\hline
			\multirow{3}{*}{2018-2019} & 4th year  & Frontal lecture \& Coding Lab & Automatic control (for the french and international curricola) & 40h \\
			& 4th year  & Coding Lab & Control of flexible structures & 8h \\
			\hline
			\multirow{2}{*}{2017-2018} & 4th year  & Frontal lecture \& Coding Lab & Automatic control (for the french and international curricola) & 40h \\			   
			\hline
		\end{tabular}
		\caption{Summary teaching activities at ISAE-SUPAERO}
		\label{tab:teaching}
	\end{table}
	
	
	
	
	\section{Scientific activities}
	
	During my professional career, I have been involved in different scientific activities. During my thesis, I participated in the supervision of the "Engineering and Enterprise Project" for 5th year students in ISAE-SUPAERO (see Table \ref{tab:activites}). This project was focused on the development of numerical algorithms for linear thermoelasticity. I contributed as a reviewer for \textit{Journal of Elasticity} and \textit{Mathematical and Computer Modelling of Dynamical Systems}. I co-organized an invited session for the conference \textit{Lagrangian and Hamiltonian method in non linear control 2021}. Currently, I co-supervise with prof. Marijn Nijenhuis a PhD student. The thesis is focused on the development of interconnected Hamiltonian models for highly flexible structures (in English \textit{flexures}). In particular, the idea is to design a structured computational framework for the digitization of flexible mechanical components for topological design and optimization purposes. I also collaborate with the Instituto Tecnológico de Aeronáutica (Brazil), for the supervision of the final year internship of a Brazilian student in double degree with the University of Twente. The subject of this internship is the reduction of models for flexible multi-body systems.
	
	\begin{table}[h]
		\centering
		\begin{tabular}{p{\dimexpr.15\linewidth-2\tabcolsep}p{\dimexpr.2\linewidth-2\tabcolsep}p{\dimexpr.65\linewidth-2\tabcolsep}}
			\hline
			Année & Lieu & Description  \\
			\hline
			2022 (ongoing) & University of Twente (Enschede) & Supervision of the master thesis of Vitor Borges Santos, in a double degree program  Instituto Tecnológico de Aeronáutica/University of Twente (collaboration with prof. Flavio Cardoso Ribeiro). \\
			2022 (ongoing) & University of Twente (Enschede) & Supervision of the PhD thesis "On the modeling and mechanical design of flexures (compliant mechanisms)" between the Robotics and Mechatronics et le the precision engineering department at the University of Twente (collaboration with prof. Marijn Nijenhuis). \\
			\hline
			2021  & Technical University of Berlin (Berlin) & Organisation of the invited session: "Theoretical and numerical advancements in Hamiltonian formulations of continuum mechanics" for the conference "Lagrangian and Hamiltonian method in non linear control 2021". \\
			\hline
			2022 & --- & Reviewer for \textit{Mathematical and Computer Modelling of Dynamical Systems}. \\
			2020 & --- & Reviewer for \textit{Journal of Elasticity}. \\
			\hline
			2019-2020 & ISAE-SUPAERO (Toulouse) & Organisation and supervision of an Engineering and Enterprise Project titled  "Simulation and control of thermoelastic structures for space applications". \\
			\hline
		\end{tabular}
		\caption{Summary scientific activities.}
		\label{tab:activites}
	\end{table}
	
\end{document}
