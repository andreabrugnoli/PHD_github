\documentclass[aspectratio=169]{beamer}
\usepackage{fontspec}
\usefonttheme{professionalfonts}
\usepackage{amsmath,amssymb,amsthm}
\usepackage{arydshln,mathtools}
\usepackage{bm}
\usepackage{color}
\definecolor{theme}{RGB}{0,73,114}
\usepackage{multicol}
%\usepackage[caption=false]{subfig}
\usepackage{subcaption}

\usepackage{comment}

\usepackage{graphicx}
\usepackage{diffcoeff}
\usepackage{dsfont}
\usepackage{mathrsfs}
\usepackage[most]{tcolorbox}

\usepackage{xspace}
\usepackage{appendixnumberbeamer}


\usepackage{media9}
\usepackage[backend=bibtex, style=verbose]{biblatex}

\bibliography{biblio_DissDyn}
%\renewcommand\bibfont{\scriptsize}

\addtobeamertemplate{footnote}{\vspace{-6pt}\advance\hsize-0.5cm}{\vspace{6pt}}
\makeatletter
% Alternative A: footnote rule
\renewcommand*{\footnoterule}{\kern -3pt \hrule \@width 2in \kern 8.6pt}
% Alternative B: no footnote rule
% \renewcommand*{\footnoterule}{\kern 6pt}
\makeatother


\makeatletter
\g@addto@macro\normalsize{%
	\setlength\abovedisplayskip{5pt}
	\setlength\belowdisplayskip{5pt}
	\setlength\abovedisplayshortskip{5pt}
	\setlength\belowdisplayshortskip{5pt}
}
\makeatother

\graphicspath{{./images/}}



% Math macros
\DeclareMathOperator*{\grad}{grad}
\DeclareMathOperator*{\Grad}{Grad}
\DeclareMathOperator*{\Div}{Div}
\renewcommand{\div}{\operatorname{div}}
\DeclareMathOperator*{\Hess}{Hess}
\DeclareMathOperator*{\curl}{curl}
\DeclareMathOperator{\Tr}{Tr}
\DeclareMathOperator{\Dom}{Dom}
\DeclareMathOperator*{\esssup}{ess\,sup}

\newcommand{\bbR}{\mathbb{R}}
\newcommand{\bbC}{\mathbb{C}}
\newcommand{\bbF}{\mathbb{F}}
\newcommand{\bbA}{\mathbb{A}}
\newcommand{\bbB}{\mathbb{B}}
\newcommand{\bbS}{\mathbb{S}}

\newcommand*{\norm}[1]{\ensuremath{\left\|#1\right\|}}
\newcommand{\where}{\qquad \text{where} \qquad}
\newcommand{\inner}[3][]{\ensuremath{\left\langle #2, \, #3 \right\rangle_{#1}}}
\newcommand{\bilprod}[2]{\left\langle \left\langle \, #1, #2 \, \right\rangle \right\rangle}
\newcommand{\pder}[2]{\ensuremath{\partial_{#2} #1}}
\newcommand{\dder}[2]{\ensuremath{\delta_{#2} #1}}
\newcommand{\secref}[1]{\S\ref{#1}}
\newcommand{\energy}[1]{\frac{1}{2} \int_{\Omega} \left\{ #1 \right\} \d\Omega}
\newcommand{\crmat}[1]{\ensuremath{\left[#1\right]_\times}}
\newcommand{\fenics}{\textsc{FEniCS}\xspace}
\newcommand{\firedrake}{\textsc{Firedrake}\xspace}

\DeclareMathOperator*{\argmax}{arg\,max}
\DeclareMathOperator*{\argmin}{arg\,min}

\newtheorem{proposition}{Proposition}
\newtheorem{remark}{Remark}
\newtheorem{hypothesis}{Hypothesis}
\newtheorem{assumption}{Assumption}
\newtheorem{conjecture}{Conjecture}


\def\onedot{$\mathsurround0pt\ldotp$}
\def\cddot{% two dots stacked vertically
	\mathbin{\vcenter{\baselineskip.67ex
			\hbox{\onedot}\hbox{\onedot}}%
}}


\setbeamertemplate{blocks}[rounded][shadow]

\setbeamercolor{block body alerted}{bg=alerted text.fg!10}
\setbeamercolor{block title alerted}{bg=alerted text.fg!20}
\setbeamercolor{block body}{bg=structure!10}
\setbeamercolor{block title}{bg=structure!20}
\setbeamercolor{block body example}{bg=green!10}
\setbeamercolor{block title example}{bg=green!20}

% Remove navigation bar
\setbeamertemplate{navigation symbols}{}

\addtobeamertemplate{navigation symbols}{}{%
	\usebeamerfont{footline}%
	\usebeamercolor[fg]{footline}%
	\hspace{1em}%
	\insertframenumber/\inserttotalframenumber
}


\makeatletter \renewcommand\d[1]{\ensuremath{%
		\;\mathrm{d}#1\@ifnextchar\d{\!}{}}}
\makeatother


\graphicspath{{./images/}}

\newif\iftocsub
\tocsubtrue
\AtBeginSection[] {
	\begin{frame}[noframenumbering]{Outline}
		\tableofcontents[sectionstyle=show/shaded, subsectionstyle=show/show/hide]
	\end{frame}
	\tocsubfalse
}
\AtBeginSubsection[] {
	\iftocsub
	\begin{frame}[noframenumbering]{Outline}
		\tableofcontents[currentsubsection, sectionstyle=show/shaded, subsectionstyle=show/shaded/hide]
	\end{frame}
	\fi
	\tocsubtrue
}

\newcommand{\beginbackup}{
	\newcounter{framenumbervorappendix}
	\setcounter{framenumbervorappendix}{\value{framenumber}}
}
\newcommand{\backupend}{
	\addtocounter{framenumbervorappendix}{-\value{framenumber}}
	\addtocounter{framenumber}{\value{framenumbervorappendix}} 
}


\begin{document}
	
	
\begin{frame}[plain]
	
	\input{TitleDissSys}
	
\end{frame}
	
	
\begin{frame}{Outline}
	
	\tableofcontents
	
\end{frame}

\section{Introduction}

\begin{frame}{Why dissipative dynamical systems?}

\begin{tcolorbox}[width=0.95\textwidth, nobeforeafter, colframe=theme,title=All engineering systems exhibit dissipation.]
\begin{itemize}
	\item Electrical networks with resistors;
	\item Mechanical systems (viscoelastic or Coulomb friction);
	\item Thermodynamic systems: dissipation leads to an increase in entropy.
\end{itemize}
\end{tcolorbox}
The notion of dissipativity establishes a natural link between the properties of input-output and state-space models. Many modern computational tools for the analysis and synthesis of control systems  are based on it.
\vspace{.3cm}

\footnotesize{
\fullcite{willems1972part1}\\
\fullcite{willems1972part2}\\
\cite{schaft1999l2}}

\end{frame}

\begin{frame}{Some mathematical notation}
$\bbR_+ = [0, \infty)$ denotes the set of positive reals. \\
Let $V$ be a finite dimensional normed liner space with norm $||\cdot||_V$. \\
\vspace{.1cm}
(If $V=\bbR^n$ then the Euclidean norm is denoted by$||x||_2 = \sqrt{x^\top x}$)

\begin{definition}[Local $L^p_{\text{loc}}$ Banach spaces]
	For each positive integer $p \in {1, 2, \dots}$, the set $L^p_{\text{loc}}(\bbR, V)$ consists of all functions $f : \bbR \rightarrow V$, which are measurable and satisfy
	\begin{equation*}
		\int_a^b ||f(t)||^p_V \d t< \infty, \qquad \forall\,  a, b \in \bbR.
	\end{equation*}
	The case $p=\infty$ consists of all bounded measurable functions on compact intervals, i.e. $\sup_{t \in [a, b]} f(t) < \infty.$
	
\end{definition}

\end{frame}



\begin{comment}
	
	\begin{definition}[$L^p$ Banach spaces]
		For each positive integer $p \in {1, 2, \dots}$, the set $L^p(\bbR, V)$ consists of all functions $f : \bbR \rightarrow V$, which are measurable and satisfy
		\begin{equation*}
			\int_{\bbR} ||f(t)||^p_V \d t< \infty,
		\end{equation*}
		The case $p=\infty$ consists of all bounded measurable functions, i.e. $\sup_{t \in \bbR} f(t) < \infty.$
		The $L^p$ spaces are Banach spaces (complete normed linear spaces) w.r.t. the norm
		\begin{equation*}
			||f||_{L^p} = \left(\int_0^\infty ||f(t)||^p_V \d t \right)^{\frac{1}{p}}, \quad q=1,2,\dots \qquad ||f||_{L^\infty} = \sup_{t \in \bbR} |f(t)|, \quad q=\infty.
		\end{equation*} 
		
	\end{definition}
	
	\begin{definition}[Extended $L^p$ Banach spaces]
		For each $T \in \bbR_+$ the function $f_T: \bbR_+ \rightarrow V$ defined by
		\begin{equation*}
			f_T= 
			\begin{cases}
				f(t), \\
				0, 
			\end{cases} \quad
			\begin{aligned}
				0\le t < T, \\
				t \ge T
			\end{aligned}
		\end{equation*}
		is called the truncation of $f$.\\
		For $q=1,2,\dots, \infty$ the set $L^{pe}(\bbR_+, V)$ consists of all measurable functions $f: \bbR_+ \rightarrow V$ such that $f_T \in L^p(\bbR_+, V), \quad  \forall T, \; 0 \le T < \infty$. \\
		The spaces $L^{pe}$ are called the extended $L^p$ spaces. It holds $L^p(\bbR_+, V) \subset L^{pe}(\bbR_+, V)$.
	\end{definition}
	
\end{comment}

\begin{frame}{General setting}
Consider the state-space system  with inputs and outputs
\begin{equation*}
	\Sigma : \quad 
	\begin{aligned}
		\dot{x} &= f(x, u), \\
		y &= h(x,u),
	\end{aligned} \qquad
	\begin{aligned}
		u(t) \in U, \\
		y(t) \in Y,
	\end{aligned}
\end{equation*} 
where $x(t) \in \mathcal{X}$. In general $\mathcal{X}$ is a manifold and $U,\; Y$ vector spaces. \\

For sake simplicity, assume $\mathcal{X}= \bbR^n,\; U=\bbR^m, \; Y=\bbR^p$.

\begin{theorem}
	Suppose $f, h$ to be Lipschitz continuous in $x$ and $u$ jointly. Then system $\Sigma$ has a unique solution $\forall x(t_0) \in \bbR^n, \; u(\cdot) \in L^2_{\text{loc}}(\bbR, U)$ with $x(\cdot) \in L^2_{\text{loc}}(\bbR, \mathcal{X}), \ y(\cdot) \in L^2_{\text{loc}}(\bbR, Y)$.
\end{theorem}

\end{frame}

\begin{frame}{Reachability and controllability}
	Notation:
	$\bbR^2_+:=\{(t_1, t_2) \in \bbR^2 |\;  t_2\ge t_1\}$ (causal triangular sector of $\bbR^2$). \\
	%Given two sets $A, B$ the notation $B^A$ indicates the set of functions $f: A \rightarrow B$. \\
	\begin{definition}[State transition function]
		Given the system $\Sigma$, the state transition function $\phi$ is the map 
		\begin{equation*}
			\phi(t_1, t_0, x(t_0), u(\cdot)) : \bbR_+^2 \times \mathcal{X} \times L^{2}_{\text{loc}}(\bbR, U) \rightarrow \bbR^n
		\end{equation*}
		such that $x(t_1) = \phi(t_1, t_0, x(t_0), u(\cdot))$. 
	\end{definition}
	\begin{definition}[Reachability and controllability]
		The state space $\mathcal{X}$ of system $\Sigma$ is said to be \textbf{reachable}
		from $x_{-1}$ if 
		\begin{equation*}
		\forall\, x \in \mathcal{X}, \; \exists\,  t_{-1} \le 0, \, \exists\, u(\cdot) \in L^{2}_{\text{loc}}(\bbR, U) \text{ such that } x = \phi(0, t_{-1}, x_{-1}, u(\cdot)).
		\end{equation*}
				It is said to be \textbf{controllable} to $x_1$ if 
		\begin{equation*}
			\forall\, x \in \mathcal{X}, \; \exists\, t_1 > 0, \; \exists \, u(\cdot) \in L^{2}_{\text{loc}}(\bbR, U) \text{ such that } x_1 = \phi(t_1, 0, x, u(\cdot)).
		\end{equation*}
	\end{definition}
\end{frame}

\section{Definition and characterization of dissipativity}

\begin{frame}{The mathematical definition of dissipativity}
On the combined space $U × Y$ consider the supply rate function $s : U \times Y \rightarrow \bbR$.

\begin{definition}[Dissipative state space system]
A state space system $\Sigma$ is said to be dissipative w.r.t. the supply rate $s$ if there exists a function $S : \mathcal{X} \rightarrow \bbR_+$ (the storage function), such
that $\forall \, x(t_0) \in \mathcal{X}$ at any time $t_0$, and $\forall\,  u(\cdot)$ and $\forall\, t_1 \ge t_0$ and the following inequality holds
\begin{equation*}\label{eq:diss_ineq}
	S(x(t_1)) \le S(x(t_0)) + \int_{t_0}^{t_1} s(u(t), y(t)) \d t, \qquad \text{Dissipation Inequality}.
\end{equation*}
It equality holds then the system is called conservative (w.r.t. the supply rate $s$). 
\end{definition}
\begin{corollary}[Convexity of the storage functions set]
	Given two storage functions $S_1$ and $S_2$ then any convex combination $\alpha S_1 + (1-\alpha) S_2, \; \alpha=[0,1]$ is also a storage function.
\end{corollary}

\end{frame}

\begin{frame}{Passive systems and $L^2$ finite gain}
\begin{overlayarea}{\textwidth}{\textheight}
	Two important class of supply rate functions:
	\begin{itemize}
		\item passive systems $s(u, y) = u^\top y$;
		\item finite $L^2$ gain $s(u, y) = \frac{1}{2} \gamma ||u||^2_2 - \frac{1}{2} ||y||^2_2, \quad \gamma \ge 0$.
	\end{itemize}

\only<1>{
	\begin{definition}[Passive system]
	$\Sigma$ with $U = Y = \bbR^m$ is \textbf{passive} if it is dissipative w.r.t.
	\begin{equation*}
		s(u, y) = u^\top y.
	\end{equation*}
	$\Sigma$ is \textbf{input strictly passive} if $\exists\, \delta > 0$ such that $\Sigma$ is dissipative w.r.t.
	\begin{equation*}
		s(u, y) = u^\top y − \delta||u||^2_2.
	\end{equation*}  
	$\Sigma$ is \textbf{output strictly passive} if $\exists\, \varepsilon >0$ such that $\Sigma$ is dissipative w.r.t.
	\begin{equation*}
		s(u, y) = u^\top y − \varepsilon ||y||^2_2
	\end{equation*}
	$\Sigma$ is \textbf{lossless} if it is conservative with respect to
	$s(u, y) = u^\top y$.
	\end{definition}
}
\only<2>{
\begin{definition}[$L^2$ finite gain]
	A system $\Sigma$ with $U = \bbR^m, \; Y = \bbR^p$ has $L^2$-gain $\le \gamma$ ($\gamma \ge 0$)
	if it is dissipative w.r.t.
	\begin{equation*}
		s(u, y) = \frac{1}{2}\gamma||u||^2_2 − \frac{1}{2}||y||^2_2.
	\end{equation*}
	 
	The $L^2$-gain of $\Sigma$ is defined as 
	\begin{equation*}
		\gamma(\Sigma) := \inf\{\gamma | \; \Sigma \text{ has } L^2\text{-gain} \le \gamma\}.
	\end{equation*}
	$\Sigma$ is said to have $L^2$-gain $<\gamma$ if $\exists\, \tilde{\gamma} \le \gamma$ such that $\Sigma$ has $L^2$-gain $\le \tilde{\gamma}$. \\
	\vspace{.3cm}
	$\Sigma$ is called {inner} if it is conservative with respect to $s(u, y) = \frac{1}{2}||u||^2_2 − \frac{1}{2}||y||^2_2$.
\end{definition}
}
\end{overlayarea}
\end{frame}




\begin{frame}{How to establish dissipativity? The available storage}
	
\begin{theorem}[Necessary and sufficient conditions for dissipativity]
Consider system $\Sigma$ and supply rate $s(u, y)$. $\Sigma$ is dissipative with respect to $s$ iff
\begin{equation*}
	S_a(x):= \sup_{\substack{u(\cdot) \\ T \ge 0}} - \int_0^T s(u(t), y(t)) \d t, \qquad x(0) = x,
\end{equation*}
is finite $\forall \, x \in \mathcal{X}$. Furthermore, if $S_a$ is finite $\forall \, x \in \mathcal{X}$ then $S_a$ is a storage function, called the \textbf{available storage}, and all other possible storage functions $S$ satisfy
\begin{equation*}
	S_a(x) \le S(x) - \inf_x S(x), \qquad \forall \,  x \in \mathcal{X}
\end{equation*}
Moreover $\inf_x S_a(x)=0.$
\end{theorem}

The available storage is the minimal storage function.
	
\end{frame}

\begin{frame}{Proof}

		\begin{itemize}
			\item ($\implies$) Suppose $S_a$ is finite. Then $S_a \ge 0$ (supremum of a set that contains 0).  Compare $S(x(t_0))$ and $S(x(t_1)) - \int_{t_0}^{t_1} s(u(t), y(t))\d t$ with $s(u, y)$ evaluated on a trajectory generated by $u: [t_0, t_1] \rightarrow \bbR^m$ that drives $x(t_0)$ at $t_0$ to $x(t_1)$ at $t_1$. \\
			Since $S_a$ is the supremum over all $u(\cdot)$ it follows
			\begin{equation*}
				S_a(x(t_0)) \ge S_a(x(t_1)) - \int_{t_0}^{t_1} s(u(t), y(t)) \d t \implies S_a \text{ is a storage function.}
			\end{equation*}
			\item ($\impliedby$) Suppose $\Sigma$ dissipative. Then $\exists\,  S \ge 0$ such that $\forall \, u(\cdot)$
			\begin{equation*}
				S(x(t)) + \int_0^T s(u(t), y(t)) \d t \ge S(x(T)) \ge 0.
			\end{equation*}
		This implies that $$S(x(0)) \ge \sup_{\substack{u(\cdot) \\ T \ge 0}} - \int_0^T s(u(t), y(t)) \d t = S_a(x(0)) \implies S_a(x(0)) < \infty$$
		\end{itemize}
\end{frame}

\begin{frame}{Reachability and Storage functions}
If the system is reachable from $x^*$, the finiteness of $S_a$ needs to be checked only in $x^*$
\begin{theorem}
	Assume that $\Sigma$ is reachable from $x^* \in \mathcal{X}$. Then $\Sigma$ is dissipative
	iff $S_a (x^∗) < \infty$.
\end{theorem}

\textbf{Proof} \\
($\impliedby$) Suppose there exists $x \in \mathcal{X}$ such that $S_a(x) = \infty$. Since
by reachability $x$  can be reached from $x^*$ in finite time, this would imply (by time
invariance) that also $S_a (x^*) = \infty$.


\end{frame}

\begin{frame}{The maximal storage: the required supply}
	If $\Sigma$ is reachable from $x^∗$, there
	exists another canonically defined storage function. 
\begin{theorem}
	Assume that $\Sigma$ is reachable from $x^∗ \in \mathcal{X}$. Define the required supply (from $x^*$) $S_r : \mathcal{X} \rightarrow \bbR \cup \{-\infty\}$ as
\begin{equation*}
	S_r(x) := \inf_{\substack{u(\cdot) \\ T \ge 0}} \int_{-T}^0 s(u(t), y(t)) \d t, \qquad x(-T)=x^*, \quad x(0)=x.
\end{equation*}
Then $S_r$ satisfies the dissipation inequality. Furthermore, $\Sigma$ is dissipative iff $\exists\, K > −\infty$ such that $S_r (x) \ge K, \; \forall \, x \in \mathcal{X}$. Moreover, if $S$ is a
storage function for $\Sigma$, then
\begin{equation*}
S(x) \le S_r(x) + S(x^*), \qquad x \in \mathcal{X},
\end{equation*}
and $S_r(x) + S(x^∗)$ is itself a storage function (and in particular $S_r(x) + S_a(x^*)$). 
\end{theorem}
\end{frame}

\begin{frame}{Proof}
	
	To steer the system from $x^∗$ at $-T$ to $x(t_1)$  consider $u(\cdot) : [-T, t_1] \rightarrow U$ which first
	take $x^∗$ to $x(t_0)$ at time $t_0 \le t_1$, and then equal to a given input $u(\cdot) : [t_0 , t_1] \rightarrow U$ transferring $x(t_0) $ to $x(t_1)$. This is a suboptimal control policy, whence
	\begin{equation*}
		S_r(x(t_0)) + \int_{t_0}^{t_1} s(u(t), y(t)) \d t \ge S_r(x(t_1)).
	\end{equation*}

	For the second claim, note that by definition of $S_a$ and $S_r$
	\begin{equation*}
	S_a(x^∗) = \sup_x −S_r(x),
	\end{equation*}
	
	
	By the previous theorem, $\Sigma$ is dissipative iff $\exists K > -\infty$ such that $S_r(x) \ge -K, \; \forall x$. \\
	Finally, let $S$ satisfy the dissipation inequality. Then for any $u(\cdot) : [-T, 0] \rightarrow U$ transferring $x(-T) = x^∗$ to $x(0) = x$ we have by the dissipation inequality
	\begin{equation*}
		S(x) - S(x^*) \le \int_{-T}^{0} s(u(t), y(t)) \d t.
	\end{equation*}
	
	Taking the infimum on the right-hand side over all $u(\cdot)$ proves the claim. \\
	If $S \ge 0$, then $S_r + S(x^∗) \ge 0$, and also $S_r + S(x^*)$ satisfies the dissipation inequality.
	
\end{frame}



\begin{frame}{Bibliography}
	%\bibliographystyle{unsrt}
	\nocite{*}
	\printbibliography
\end{frame}
	
	
\end{document}