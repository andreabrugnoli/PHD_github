% beautiful title slides in Beamer
% Model 6
% latex-beamer.com

\documentclass[aspectratio=169]{beamer}
\usepackage{color}
\definecolor{theme}{RGB}{0,73,114}
\usepackage{media9}
\usepackage[backend=bibtex, style=authoryear, doi=false,isbn=false,url=false]{biblatex}
\usepackage[most]{tcolorbox}
% Remove navigation bar
\setbeamertemplate{navigation symbols}{}

% Tikz package
\usepackage{tikz}
\usetikzlibrary{positioning}

\graphicspath{{./images/}}

\bibliography{biblio}


%% At begin of each section: show current section and all subsections in the section if any
%% At begin of each subsection except first: show only the current section/subsection
\newif\iftocsub
\tocsubtrue
\AtBeginSection[] {
	\begin{frame}[noframenumbering]{Outline}
		\tableofcontents[sectionstyle=show/shaded, subsectionstyle=show/show/hide]
	\end{frame}
	\tocsubfalse
}
\AtBeginSubsection[] {
	\iftocsub
	\begin{frame}[noframenumbering]{Outline}
		\tableofcontents[currentsubsection, sectionstyle=show/shaded, subsectionstyle=show/shaded/hide]
	\end{frame}
	\fi
	\tocsubtrue
}


\begin{document}

% Title slide frame
\begin{frame}[plain]

%%%%%%%% Title slide details %%%%%%%%%%%%%%


% Background Image
\newcommand{\myBackground}
{
    \includegraphics[height=1.02\paperheight,page=9]{beamerthemeutresources}
}

% Title
\newcommand{\myTitle}
{
    Numerics for the Portwings project
}

% Subtitle
\newcommand{\mySubTitle}
{
   Portwings external meeting
}

% Author
\newcommand{\myAuthor}   
{
    Andrea Brugnoli
}

% Affiliation
\newcommand{\myAffiliate}
{
  
}

% Presentation Date
\newcommand{\myDate}   
{
    \today
}

% Logo
\newcommand{\myLogo}   
{
    \includegraphics[width=3cm]{Logo.png}
}
%%%%%%%%%%%%%%%%%%%%%%%%%%%%%%%%%%%%


%%%%%%%%%% Title slide code %%%%%%%%%%%
\begin{tikzpicture}[remember picture,overlay]

% Background color

\fill[white] (current page.south west) rectangle (current page.north east);
% Background image
\node[above right,inner sep=0pt] at (current page.south west)
    {
        \myBackground
    };
    
% Title & Subtitle
\node
[
    above=0.5cm,
    align=center,
    draw=black!50,
    % rounded corners,
    double,
    double distance=0.1cm,
    double=blue!10,
    fill=theme!10,
    inner xsep=15pt,
    inner ysep=10pt, 
    minimum width=0.7\textwidth,
    text width=0.6\textwidth
] (title) at (current page.center)
{
    \LARGE \myTitle  \\[5pt]
    \small \mySubTitle
};

% Author 
\node[ below=0.5cm] (author) at (title.south){\myAuthor};

% Author 
\node[ below=0.25cm ](affiliate) at (author.south){\small \myAffiliate};

% Date
\node[below=0.25] (date) at (affiliate.south){\large \myDate};

% Logo
\node
[
    below =0.25cm
] at (date.south)
{
    \myLogo
};

\end{tikzpicture}
    
\end{frame}

\begin{frame}{Overview}
	\tableofcontents
\end{frame}

\section{Introduction}

\begin{frame}{Vision for the portwings project}
	Numerical methods for port-Hamiltonian system should:
	\begin{itemize}
		\item reproduce the physical properties of the problem;
		\item preserve the modularity inherithed from the Dirac structure;
		\item work in any spatial dimension;
		\item adaptable to many physical problems (fluid and solid mechanics, electromagnetism, etc.);
	\end{itemize}
\end{frame}

\begin{frame}{Classical methods for computational simulation}
\begin{tcbraster}[raster columns=3, raster equal height]
	\begin{tcolorbox}[width=0.32\textwidth, nobeforeafter, colframe=theme,title=Finite differences]%%
	Pros:
	\begin{itemize}
		\item quadrature-free implementation;
		\item diagonal mass matrices;
	\end{itemize}
	Cons:
	\begin{itemize}
		\item solution is only pointwise;
		\item difficult bcs. implementation;
		\item no Galerkin orthogonality;
	\end{itemize}
	\end{tcolorbox} 
	\begin{tcolorbox}[width=0.32\textwidth, nobeforeafter,  colframe=theme,title=Finite volumes]%%
		Pros:
		\begin{itemize}
			\item Exact conservation laws (CFD);
			\item incorporates discontinous phenomena;
		\end{itemize}
		Cons:
		\begin{itemize}
			\item low order approximation;
			\item requires Voronoi dual mesh;
		\end{itemize}
	\end{tcolorbox}
	\begin{tcolorbox}[width=0.32\textwidth, nobeforeafter,  colframe=theme,title=Finite elements]%%
		Pros:
		\begin{itemize}
			\item Galerkin orthogonality;
			\item geometric flexibility;
			\item mathematical foundation;
		\end{itemize}
		Cons:
		\begin{itemize}
			\item trouble at high aspect ratio;
			\item continuous FEM has trouble with transport;
		\end{itemize}
	\end{tcolorbox}
\end{tcbraster}
\end{frame}


\begin{frame}{Finite difference implementation of Rayleigh-Taylor instability}
	\begin{figure}
		\centering
		\includemedia[
		label=vidNoRod,
		addresource=/home/andrea/Videos/Videos_PW/Rayleight_Taylor_FDM.mp4,
		activate=pageopen,
		width=10cm, height=6cm, 
		flashvars={
			source=/home/andrea/Videos/Videos_PW/Rayleight_Taylor_FDM.mp4
			&loop=true
		}
		]{}{VPlayer.swf}
	\end{figure}
	Source: \url{https://wci.llnl.gov/simulation/computer-codes/miranda}
\end{frame}


\begin{frame}{Finite volume simulations in astrophysics}
	\begin{figure}
		\centering
		\includemedia[
		label=vidNoRod,
		addresource=/home/andrea/Videos/Videos_PW/turbolence_mach10_FVM.mp4,
		activate=pageopen,
		width=10cm, height=6cm, 
		flashvars={
			source=/home/andrea/Videos/Videos_PW/turbolence_mach10_FVM.mp4
			&loop=true
		}
		]{}{VPlayer.swf}
	\end{figure}
	\fullcite{price2010}
\end{frame}

\begin{frame}{Finite elements in biology}
	\begin{figure}
		\centering
		\includemedia[
		label=vidNoRod,
		addresource=/home/andrea/Videos/Videos_PW/aorta_FEM.mp4,
		activate=pageopen,
		width=10cm, height=6cm, 
		flashvars={
			source=/home/andrea/Videos/Videos_PW/aorta_FEM.mp4
			&loop=true
		}
		]{}{VPlayer.swf}
	\end{figure}
	\fullcite{laadhari2017}
\end{frame}


\end{document}