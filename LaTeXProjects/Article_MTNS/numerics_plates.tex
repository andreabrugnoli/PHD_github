%===============================================================================
% $Id: ifacconf.tex 7 2007-11-21 12:50:23Z jpuente $
% Template for IFAC meeting papers
% Copyright (c) 2007 International Federation of Automatic Control
%===============================================================================
\documentclass{ifacconf}
\usepackage[round]{natbib} % you should have natbib.sty
\usepackage{graphicx}      % include this line if your document contains figures

\usepackage{amsmath,amssymb}
\usepackage{url}
\usepackage{diffcoeff}
\usepackage{bm}

\usepackage{mathrsfs}
\usepackage{multirow}

\usepackage{caption, subfig}

\usepackage{xcolor,colortbl}

\newtheorem{remark}{Remark}

\DeclareMathOperator*{\argmax}{arg\,max}
\DeclareMathOperator*{\argmin}{arg\,min}
\DeclareMathOperator{\Tr}{Tr}

\def\onedot{$\mathsurround0pt\ldotp$}
\def\cddot{% two dots stacked vertically
	\mathbin{\vcenter{\baselineskip.67ex
			\hbox{\onedot}\hbox{\onedot}}%
}}

\makeatletter \renewcommand\d[1]{\ensuremath{%
		\;\mathrm{d}#1\@ifnextchar\d{\!}{}}}
\makeatother

\bibliographystyle{agsm}
\graphicspath{{./Figures/}}
%===============================================================================
\begin{document}
\begin{frontmatter}

\title{Numerical discretization of port-Hamiltonian plate models \thanksref{footnoteinfo}} 
% Title, preferably not more than 10 words.

\thanks[footnoteinfo]{This work is  supported by the project ANR-16-CE92-0028,
	entitled {\em Interconnected Infinite-Dimensional systems for Heterogeneous
		Media}, INFIDHEM, financed by the French National
	Research Agency (ANR) and the Deutsche Forschungsgemeinschaft (DFG). Further information is available at {\url{https://websites.isae-supaero.fr/infidhem/the-project}}.
	}

\author[ISAE]{Andrea Brugnoli}
\author[ISAE]{Daniel Alazard} 
\author[ISAE]{Val\'erie Pommier-Budinger}
\author[ISAE]{Denis Matignon}

\address[ISAE]{ISAE-SUPAERO, Universit\'e de Toulouse, France.\\
	10 Avenue Edouard Belin, BP-54032, 31055 Toulouse Cedex 4. \\
	Andrea.Brugnoli@isae.fr,  Daniel.Alazard@isae.fr, \\
	Valerie.Budinger@isae.fr, Denis.Matignon@isae.fr}

\begin{abstract}

\end{abstract}

\begin{keyword}
Port-Hamiltonian systems, Kirchhoff Plate, Mindlin-Reissner Plate, Mixed Finite Element Method, Numerical convergence
\end{keyword}

\end{frontmatter}
%===============================================================================

\section{Introduction}


\section{Problem statement}

In this section the models under consideration are recalled. The details can be found in  \cite{BRUGNOLI2019961,BRUGNOLI2019940}. 

\subsection{Notations}
For a scalar field $u: \mathbb{R}^d \rightarrow \mathbb{R}$ the gradient is defined as 
\begin{equation*}
\mathrm{grad}(u) =  \nabla u := \begin{pmatrix}
\partial_{x_1} u \dots \partial_{x_d} u \\
\end{pmatrix}^\top.
\end{equation*}
For a vector field $\bm{u}: \mathbb{R}^d \rightarrow \mathbb{R}^d$, with components $u_j$, the gradient is defined as
\begin{equation*}
\mathrm{grad}(\bm{u})_{i j}:= (\nabla \bm{u})_{ij} = \partial_{x_i} u_j.
\end{equation*}
The symmetric part of the gradient operator $\mathrm{Grad}$ (i. e. the deformation gradient in continuum mechanics) is given by
\begin{equation*}
\mathrm{Grad}(\bm{u}) := \frac{1}{2} \left(\nabla \bm{u} + \nabla^\top \bm{u} \right).
\end{equation*}
The Hessian operator of $u$ is then computed as follows
\begin{equation*}
\mathrm{Hess}(u) = \nabla^2 u = \mathrm{Grad}(\mathrm{grad}(u)),
\end{equation*}
For a tensor field $\bm{U}: \mathbb{R}^d \rightarrow \mathbb{R}^{d \times d}$, with components $u_{ij}$, the divergence is a vector, defined column-wise as
\begin{equation*}
\mathrm{Div}(\bm U) = \nabla \cdot \bm{U} := \left( \sum_{i = 1}^d \partial_{x_i} u_{ij} \right)_{j = 1, \dots, d}.
\end{equation*}
The double divergence of a tensor field $\bm{U}$ is then a scalar field defined as
\begin{equation*}
\mathrm{div}(\mathrm{Div}(\bm U)):= \sum_{i, j = 1}^d \partial_{x_i} \partial_{x_j} u_{ij} \, .
\end{equation*}

\subsection{Plate models in port-Hamiltonian form}

\paragraph{Mindlin-Reissner plate}
The Mindlin model is a generalization to the 2D case of the Timoshenko beam model and is expressed by a system of three coupled PDEs (\cite{timoshenko1959theory})  
\begin{equation}
\begin{cases}
\displaystyle \rho h \diffp[2]{w}{t} &= \mathrm{div}(\bm{q}),  \vspace{1mm}\\
\displaystyle \frac{\rho h^3}{12} \diffp[2]{\bm \theta}{t} &= \bm{q} + \mathrm{Div}(\bm M), \\
\end{cases}
\end{equation}
where $\rho$ is the mass density, $h$ the plate thickness, $w$ the vertical displacement, $\bm \theta = (\theta_x, \theta_y)^\top$ collects the deflection of the cross section along axes $x$ and $y$ respectively. Variables $\bm{M}, \bm{q}$ represent the momenta tensor and the shear stress. The Hooke law relates those to the curvature tensor and shear deformation vector
\begin{equation*}
\begin{aligned}
\bm{M} &:= \mathcal{D} \bm{K}, \\ \bm{q} &:= \mathcal{C} \bm{\gamma},
\end{aligned} \qquad
\begin{aligned}
\bm{K} &:= \mathrm{Grad}(\bm{\theta}), \\ \bm{\gamma} &:= \mathrm{grad}(w) - \bm{\theta}, 
\end{aligned}
\end{equation*}
where $\mathcal{D}, \mathcal{C}$ are symmetric positive tensors. The kinetic and potential energy density $\mathcal{K}$ and $\mathcal{U}$ read
\begin{equation}
\begin{aligned}
\mathcal{K} &=  \frac{1}{2} \left\{ \rho h \left(\diffp{w}{t} \right)^2 +  \frac{\rho h^3}{12} \diffp{\bm{\theta}}{t} \cdot \diffp{\bm{\theta}}{t}  \right\}, \\
\mathcal{U} &= \frac{1}{2} \left\{ \bm{M} \cddot \bm{K} + \bm{q} \cdot \bm{\gamma}  \right\},
\end{aligned}
\end{equation} 
where $\bm{M} \cddot \bm{K} := \sum_{i,j} m_{ij} \kappa_{ij}$ is the tensor contraction. The Hamiltonian  is easily written as
\begin{equation} 
H = \int_{\Omega} \left( \mathcal{K} + \mathcal{U} \right)   \d\Omega. 
\end{equation}
To get a port-Hamiltonian formulation suitable energy variables must be selected. The appropriate set is the following
\begin{equation}
\begin{aligned}
\alpha_w &= \rho h \diffp{w}{t}, \\
\bm{A}_{\kappa} &= \bm{K}, \\
\end{aligned} \qquad
\begin{aligned}
\bm\alpha_{\theta} &= \frac{\rho h^3}{12} \diffp{\bm{\theta}}{t}, \\
\bm\alpha_{\gamma} &= \bm{\gamma}. \\
\end{aligned}
\end{equation}
The co-energy variables are found by computing the variational derivative of the Hamiltonian
\begin{equation}
\begin{aligned}
e_w &:= \diffd{H}{\alpha_w} = \diffp{w}{t},  \\
\bm{E}_{\kappa} &:= \diffd{H}{\bm{A}_{\kappa}} = \bm{M}, \\
\end{aligned} \qquad
\begin{aligned}
\bm{e}_{\theta} &:= \diffd{H}{\bm\alpha_{\theta}} = \diffp{\bm{\theta}}{t}, \\
\bm{e}_{\gamma} &:= \diffd{H}{\bm{\alpha}_{\bm{\gamma}}} = \bm{q}. \\
\end{aligned}
\end{equation}
Energy and co-energy are relative by a positive symmetric operator $\bm{\alpha} = \mathcal{Q} \bm{e}$
\[ \mathcal{Q} = \mathrm{diag}(\frac{1}{\rho h}, \; \frac{12}{\rho h^3} , \; \mathcal{D}, \; \mathcal{C})
\]

The port-Hamiltonian system is expressed as follows 
\begin{equation}
\label{eq:PH_sys_Min_Ten}
\diffp{}{t}
\begin{pmatrix}
\alpha_w \\
\bm\alpha_\theta \\
\bm{A}_\kappa \\
\bm\alpha_{\gamma} \\
\end{pmatrix} = 
\underbrace{\begin{bmatrix}
	0  & 0  & 0  & \mathrm{div} \\
	0 & 0 &  \mathrm{Div} & \bm{I}_{2 \times 2}\\
	0  & \mathrm{Grad}  & 0  & 0\\
	\mathrm{grad} & -\bm{I}_{2 \times 2} &  0 & 0  \\
	\end{bmatrix}}_{J}
\begin{pmatrix}
e_w \\
\bm{e}_{\theta} \\
\bm{E}_{\kappa} \\
\bm{e}_{\gamma} \\
\end{pmatrix},
\end{equation}
This system defines a Stokes-Dirac structure, therefore, the boundary values can be found by evaluating the time derivative of the Hamiltonian. 

\paragraph{Kirchhoff plate}
The Kirchhoff Model is a generalization to the 2D case of the Euler-Bernoulli beam model. The classical equations for this model \cite{timoshenko1959theory} are 
\begin{equation}
\displaystyle \rho h \diffp[2]{w}{t} = -\mathrm{div}(\mathrm{Div}(\bm{M}))
\end{equation}
The bending moment tensor and the curvature are related as in the Mindlin model $\bm{M} = \bm{D} \bm{K}$. Following the Kirchhoff assumption the curvature tensor is the Hessian of the vertical displacement
\begin{equation*}
\bm{K} := \mathrm{Grad}(\mathrm{grad}(w)).
\end{equation*}
 The kinetic and potential energy densities $\mathcal{K}$ and $\mathcal{U}$ read
\begin{equation}
\mathcal{K} =  \frac{1}{2}\rho h \left(\diffp{w}{t} \right)^2, \quad
\mathcal{U} = \frac{1}{2} \bm{M} \cddot \bm{K},
\end{equation} 
The Hamiltonian  is easily written as
\begin{equation} 
H = \int_{\Omega} \left( \mathcal{K} + \mathcal{U} \right)   \d\Omega. 
\end{equation}
Selecting as energy variables
\begin{equation}
\begin{aligned}
\alpha_w &= \rho h \diffp{w}{t}, \quad &\text{Linear momentum}, \\
\bm{A}_{\kappa} &= \bm{K}, \quad &\text{Curvature tensor}.\\
\end{aligned}
\end{equation}
co-energy variables are found by computing the variational derivative of the Hamiltonian
\begin{equation}
\begin{aligned}
e_w &:= \diffd{H}{\alpha_w} = w_t, \quad &\text{Vertical Velocity}, \\
\bm{E}_{\kappa} &:= \diffd{H}{\bm{A}_{\kappa}} = \bm{M}, \quad &\text{Momenta tensor},\\
\end{aligned} 
\end{equation}
where $w_t := \diffp{w}{t}$ for compactness. The coercive operator linking energy and co-energies reads
\[
\mathcal{Q} = \mathrm{diag}(\frac{1}{\rho h}, \mathcal{D})
\] 
The port-Hamiltonian system is expressed as follows 
\begin{equation}
\label{eq:PH_sys_Kir_Ten}
\diffp{}{t}
\begin{pmatrix}
\alpha_w \\
\bm{A}_\kappa \\
\end{pmatrix} = 
\underbrace{\begin{bmatrix}
	0  & -\mathrm{div} \circ \mathrm{Div} \\
	\mathrm{Grad} \circ \mathrm{grad}  & 0 \\
	\end{bmatrix}}_{\mathcal{J}}
\begin{pmatrix}
e_w \\
\bm{E}_{\kappa} \\
\end{pmatrix},
\end{equation}

\section{Conclusion}

\begin{ack}
The authors would like to thank Michel Sala\"un from ISAE-SUPAERO for the fruitful and insightful discussions.
\end{ack}

\bibliography{biblio_MTNS}             % bib file to produce the bibliography
                                                     % with bibtex (preferred)
                                                   
%\begin{thebibliography}{xx}  % you can also add the bibliography by hand

%\bibitem[Able(1956)]{Abl:56}
%B.C. Able.
%\newblock Nucleic acid content of microscope.
%\newblock \emph{Nature}, 135:\penalty0 7--9, 1956.

%\bibitem[Able et~al.(1954)Able, Tagg, and Rush]{AbTaRu:54}
%B.C. Able, R.A. Tagg, and M.~Rush.
%\newblock Enzyme-catalyzed cellular transanimations.
%\newblock In A.F. Round, editor, \emph{Advances in Enzymology}, volume~2, pages
%  125--247. Academic Press, New York, 3rd edition, 1954.

%\bibitem[Keohane(1958)]{Keo:58}
%R.~Keohane.
%\newblock \emph{Power and Interdependence: World Politics in Transitions}.
%\newblock Little, Brown \& Co., Boston, 1958.

%\bibitem[Powers(1985)]{Pow:85}
%T.~Powers.
%\newblock Is there a way out?
%\newblock \emph{Harpers}, pages 35--47, June 1985.

%\bibitem[Soukhanov(1992)]{Heritage:92}
%A.~H. Soukhanov, editor.
%\newblock \emph{{The American Heritage. Dictionary of the American Language}}.
%\newblock Houghton Mifflin Company, 1992.

%\end{thebibliography}

\appendix

      
                                                          % in the appendices.
\end{document}
