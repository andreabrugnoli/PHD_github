\documentclass[french]{article}
\usepackage{graphicx}
\usepackage{caption}
\usepackage[T1]{fontenc}
\usepackage[utf8]{inputenc}
\usepackage{lmodern}
\usepackage[a4paper]{geometry}
\usepackage{babel}
\usepackage[unicode=true,pdfusetitle,bookmarks=true,bookmarksnumbered=false,bookmarksopen=false,
breaklinks=false,pdfborder={0 0 0},backref=false,colorlinks=true]{hyperref}

\author{Andrea Brugnoli \\ 
\hspace{2.8pt} Docteur ISAE-Supaéro 2020\\
Ingénieur ISAE-Supaéro 2017}
\title{Méthodes numériques pour la révolution digitale des jumeaux numériques: de la modélisation multi-physique haute fidélité aux modèles réduits pour l'ingénierie}

\date{}

\begin{document}

\maketitle

\large{Dossier de candidature au prix de la fondation Jean-Jacques et Félicia
	Lopez-Loreta pour l’excellence académique}


\begin{figure}[h]
	\centering
	\includegraphics[width=.95\textwidth]{3Dplane.jpg}
	\captionsetup{labelformat=empty}
	\caption{Source: \href{http://www.fenics-hpc.org/}{FEniCS-HPC website}}
\end{figure}





\thispagestyle{empty}

\newpage

\section{Contexte et Objectifs du projet}

\subsection{Le candidat}
La technologie,  les sciences et leur impacte sur l'humaine m'ont toujours intéressé. C'est pour cela que j'ai opté pour un baccalauréat littéraire avec option informatique (obtenu en 2011 a Vérone, Italie). Après mon baccalauréat\footnote{En Italie il est possible d'accéder aux universités scientifique après un Bac. L.}, j'ai obtenu une licence en ingénierie mécanique du Politecnico de Milan. Pendant la première année du master en ingénierie Spatiale, j'ai décide de partir a'l'étranger et j'ai choisi d'effectuer un double diplôme a l'ISAE-Supaéro. J'ai pu approfondir mes connaissances en automatique grâce \`a un master recherche en collaboration avec Supélec/Université Paris Saclay, ainsi que mes compétences en mathématiques appliquées \`a travers un parcours specilis\'e. Mon intérêt pour les systèmes dynamiques et la simulation m'a amené au centre national d'études spatiales (CNES)) pour mon stage de fin études, ou j'ai effectué des simulations intensives sur le supercalculateur. \\

J'ai donc décidé de poursuivre un doctorat de recherche dans l'automatique et les mathématiques appliques (calcul numérique). Ma thèse a été finance dans le cadre d'un projet européenne financé par l'agence national de la recherche (ANR) et la Fondation allemande pour la recherche (DFG). Le projet de recherche consiste a utiliser de formulation physique pour les systèmes dynamiques interconnecté.



\subsection{Contexte}

\subsection{Le projet}

\section{Organisation du projet et mise en œuvre}

\subsection{Partenariats académiques et retombées industrielles}

\subsection{Le plan}

\subsection{Budjet}




\bibliographystyle{unsrt}
\nocite*
\bibliography{biblio_articles}

\end{document}
